%Senior Thesis Draft
%Ben Cook '14 (bacook@)
%Adviser: Neta Bahcall

\documentclass{puthesis_undergraduate}
%preamble
%preamble document for 2014 thesis
%Ben Cook '14 (bacook@)
%adviser: Neta Bahcall

\bibliographystyle{apj}
\usepackage{graphicx, multirow, natbib,amsmath,amssymb,afterpage}
\usepackage[bookmarks=true,colorlinks=false,bookmarksopen]{hyperref}

%slightly bigger plots than standard
\newcommand\plottwobig[2]{\centering \leavevmode
\includegraphics[width=.49\linewidth]{#1} \hfil
\includegraphics[width=.49\linewidth]{#2}}


\newcommand\plotonebig[1]{\centering \leavevmode
\includegraphics[width=\linewidth]{#1}}

\newcommand\plotoneman[2]{\centering \leavevmode
\includegraphics[width=#2\linewidth]{#1}}

%commonly used macros
\newcommand\apx{\ensuremath{\sim}}
\newcommand\citeeg[1]{\citep[e.g.,][]{#1}}
\newcommand\gsim{\gtrsim}
\newcommand\lsim{\lesssim}
\newcommand\kms{km~s\ensuremath{^{-1}}}
\newcommand\Lsun{\ensuremath{L_\odot}}
\newcommand\Msun{\ensuremath{M_\odot}}
\newcommand\todo[1]{\textsc{#1}}

\newcommand\rhom{\ensuremath{\rho_m}}
\newcommand\rhob{\ensuremath{\rho_b}}
\newcommand\rhoc{\ensuremath{\rho_c}}
\newcommand\rhog{\ensuremath{\rho_\gamma}}
\newcommand\rhocrit{\ensuremath{\rho_{crit}}}
\newcommand\omegam{\ensuremath{\Omega_m}}
\newcommand\omegab{\ensuremath{\Omega_b}}
\newcommand\omegac{\ensuremath{\Omega_c}}
\newcommand\omegag{\ensuremath{\Omega_\gamma}}
\newcommand\omegal{\ensuremath{\Omega_\Lambda}}

\newcommand\fb{\textrm{f\ensuremath{_b}}}
\newcommand\Ho{\textrm{H\ensuremath{_0}}}


\title{Keep Calm and Baryon: The Distribution of Baryons and Dark Matter on
  Cosmic Scales}
\author{Benjamin A. Cook}
\advisor{Prof. Neta Bahcall}

\abstract{We present a compilation of observational constraints on the
  distribution of baryons, relative to dark matter, in cosmic halos
  ranging from galaxies to massive clusters. These include X-ray and
  SZ measurements of the hot intracluster medium (ICM) in groups and
  clusters, weak lensing and optical constraints on the cluster
  stellar fraction, and absorption measurements of the cool
  circumgalactic medium (CGM) outside galaxies. Using direct
  observations when possible and extrapolations of observed density
  profiles when necessary, we show that the baryon content within the
  virial radius matches the cosmic baryon fraction ($0.164$) for halos
  ranging over three orders of magnitude in mass. The baryonic mass is
  therefore a strong tracer of the underlying dark matter
  distribution, and can be used as a tool in cosmological studies of
  structure formation, although baryons are more extended beyond the
  virial radius in low-mass halos. Using the ratio of stellar
  fractions in galaxies and clusters, we place a lower limit of $40\%$
  on the contributions of individual galaxy halos to the dark matter
  and baryonic mass of clusters.}

\acknowledgements{
Thank Prof. Prochaska, and Lars/Shy and Carlstrom/Plagge if they get
back to me.}

\dedication{Dedicated to my brother, Walter.
}


\begin{document}
%=====================================================================

%Senior Thesis - First Chapter
%Ben Cook '14 (bacook@)
%Adviser: Neta Bahcall

\chapter{Introduction}
\label{c.Intro}
\section{The Cosmic Baryon Fraction}
\label{s.BaryonFraction}
In the beginning, there was the big bang. All of the contributions to
the cosmic energy budget -- all of the forms that matter and energy
take today -- originated around 13.8 billion years ago, when the
universe was unimaginably hot and dense. Various
mechanisms\footnote{most notably the theory of inflation, which will
  surely see a dramatic increase in interest since the detections of
  B-mode polarization in the CMB \citep{BICEP22014}.} have been
offered to explain the genesis of the energy-filled, rapidly expanding
universe which appeared in the briefest fraction of a second after the
big bang. As the universe expanded and cooled, the available energy
distributed itself into various different forms, eventually settling
down into the primary energy components we observe today, including
radiation (photons and neutrinos), baryons (``ordinary'' matter,
comprised of protons, neutrons, and electrons), as well as the
mysterious dark matter and dark energy. 

Energy did not populate these forms in equal proportions; the energy
densities of each population differed by many orders of magnitude,
initially, and their ratios changed continually throughout the
expansion history of the universe. Radiation density -- primarily the
photon density ($\rho_\gamma$), dominant in the earliest periods after
the big bang -- diluted quickly from the combination of expansionary
volume increase and Doppler redshifting. The matter density --
$\rho_m$, comprised of both baryons ($\rho_b$) and cold dark matter
($\rho_c$) -- was initially only a miniscule portion of the cosmic
energy budget, but eventually matter dominated the cosmic scene after
expansion ``cooled'' the photon temperature significantly. Buried far
below the other components originally, dark energy -- $\rho_\nu$,
commonly thought to be a cosmological constant $\Lambda$ -- retains a
constant energy density while the universe expands, becoming dominant
at late times when \rhom has decreased significantly. Each of these
energy densities are often scaled by the critical density required to
stop cosmic expansion: $\rhocrit = \frac{8}{3}\pi{}G\Ho{}^2c^{-2}$,
with \Ho{} the Hubble constant, G Newton's constant, and c the speed
of light. The density of each component relative to the critical
density is expressed as $\Omega$, for example:
\begin{equation}
\frac{\rhob}{\rhocrit} = \omegab.  \nonumber
\end{equation}

Baryons and dark matter, the two components of the total matter
density, have the same dependence on the expansionary scale factor
and redshift. Since the total matter density is just their sum, the
total matter density also scales identically:
\begin{align}
\rhoc \propto \rhob \propto \rhom & \propto \textrm{a}^{-3} \nonumber \\
& \propto (1+\textrm{z})^{3}. \nonumber
\end{align}
Therefore, we see that the ratio of baryons to dark matter will remain
constant from its primoridal level throughout the history of the
universe. A more useful and commonly studied constant is the
\textit{cosmic baryon fraction}, \fb, the fraction of all matter in
baryonic form:
\begin{equation}
\fb = \frac{\rhob}{\rhob + \rhoc} = \frac{\rhob}{\rhom} =
\frac{\omegab}{\omegam}.
\end{equation}

The above argument, that the baryon fraction remains constant
throughout cosmic history, applies only in the homogenous regime,
where there are no spatial variations in the overall mass
density. When inhomogeneities exist, self-interactions lead to the
complicated evolution of structure. Both baryons and dark matter are
subject to gravitational forces, causing intitial overdensities to
increase in magnitude, eventually collapsing into massive halos. Yet
while dark matter primarily interacts only through gravity, baryons
are subject to electromagnetic forces, thermal emission, and numerous
other interactions that lead to a divergence between the dark matter
distribution and the baryon distribution.

While both dark matter and baryons cause the gravitational collapse of
inhomogeneities and drive the growth of structure, it is obvious that
the baryon fraction plays an incredibly important role in determining
the makeup of our universe. Baryons are responsible for all other
phenomena studied in physics and astrophysics: the formation of
galaxies and stars, supernovae, photon radiation, and, eventually,
life itself. Although the local baryon fraction may vary hugely from
place to place, it is possible to estimate the cosmic baryon fraction
by averaging over a substantially large volume. This cosmic ratio is
a defining characteristic, and studying it is essential to properly
understanding the creation and evolution of our universe. \todo{This
  section can use some major work. Narrow the focus, but still give it
  some depth.}

The baryon fraction was a major factor in several important physical
processes in the early universe. Through observable consequences of
these processes, cosmologists have been able to place powerful
constraints on the cosmic baryon fraction at these early
times. Consistent with the literature, we will consider the baryon
fraction inferred from these methods to be the ``true'' cosmic
fraction against which we will compare measurements from the local,
highly inhomogenous universe.

One such mechanism is big bang nucleosynthesis (BBN), which generated
the first light elements beyond hydrogen\footnote{The discussion which
  follows is guided primarily by chapter 3.2 of \citet{Weinberg2008},
  a useful but relatively technical reference on the topic.}. In the
first seconds after the big bang, the only ordinary matter particles
which existed (and were stable) were protons ($p$), electrons
($e$), and neutrons ($n$). The high temperatures and densities of
nucleons allowed the conversion of protons and neutrons into more
complex and heavier nucleii, through processes such as:
\begin{align}
p + n &\rightarrow{} d + \gamma \nonumber \\
d + d &\rightarrow{} ^3\textrm{He} + n \nonumber \\
d +{} ^3\textrm{He} &\rightarrow{} ^4\textrm{He} + p .\nonumber
\end{align}
The BBN reactions began around 100 -- 200 seconds after the big bang
\citep{Weinberg2008}. The exact time when these reactions reached
thermal equilibrium depends weakly on the abundance of baryons,
$\omegab h^2$, where $\Ho = 100h~\kms Mpc^{-1}$ defines $h$, an
important scaling factor which we will discuss later. After the
universe expanded and cooled sufficiently, these reactions fell out of
equilibrium, leaving the universe enriched with helium ($^4$He and
$^3$He) and trace amounts of elements such as deuterium ($d$) and
lithium ($^7$Li). The transformation of $n$ and $d$ into helium is
more complete the higher $\omegab h^2$. Therefore, the baryon
abundance strongly affects the resulting abundance of $d$ and residual
elements like lithium and $^3$He. Figure \ref{fig:Abundances} shows
the dependance of these primordial abundances on the cosmic baryon
abundance.

\begin{figure*}[hbt]
\plotonebig{Abundances}
\caption{The BBN-predicted primordial abundances of deuterium (D),
  $^3$He, $^7$Li, and $^4$He (Y$_\textrm{P}$), as a function of the
  baryon abundance parameter $\eta_{10} \sim 274 \times \omegab
  h^2$. The width of the curves represents the uncertainties in
  various nuclear reaction rates. Figure taken from
  \citet{Steigman2006}, another helpful overview of BBN physics.}
\label{fig:Abundances}
\end{figure*}    



The baryon abundance can be determined through observational
constraints of, for example, the deuterium abundance, which among the
common byproducts of BBN depends most strongly on $\omegab h^2$. The
deuterium abundance has been inferred from variety of sources,
including from the Milky Way's ISM \citep{Linsky1993, Linsky1995},
absorption towards QSOs \citep{Tytler1996, Kirkman2003}, and even from
measurements of the composition of the Jovian atmosphere
\citep{Niemann1996}. All such methods have limitations, as deuterium
can be destroyed in stellar (and brown dwarf) cores, altering the
deuterium abundance slightly with time. \citet{Iocco2009} provides a
modern compilation of deuterium abundance measurements, placing the
constraint on the baryon abundance at $\omegab h^2 = 0.021 \pm
0.001$. Observed $^7$Li abundances are a factor of a few lower than
predicted from BBN, suggesting that there could be additional physics
responsible for destroying lithium \citep{Suzuki2000,
  Melendez2004}. This is known as the ``Lithium Problem,'' and is
still an unsolved problem in interpreting BBN.

A complementary method of inferring the baryon abundance at early
times is from measurements of the acoustic peaks in the Cosmic
Microwave Background (CMB) power spectrum. Remember that the CMB power
spectrum contains large peaks, representing correlations in the CMB on
particular scales. In the first few hundred thousand
years\footnote{Virtually a cosmic blink of the eye.} after the big
bang, the temperature of the universe was high enough to keep atoms
fully ionized into seperate nucleii and electrons. These charged
particles were strongly coupled to the photons through electromagnetic
interactions, so that the two combined to form what is called a
photon-baryon fluid. This fluid (which, at the time, had an energy
density around $\frac{1}{3}$ that of dark matter) fell towards the
centers of gravitational wells created by dark matter. However, unlike
the non-interacting dark matter, the photon-baryon fluid's pressure
rose when its density rose, forcing the fluid out of the well
until its pressure dropped enough to allow gravity to draw it back
once again. These ongoing fluctuations in the pressure and density
of the fluid were frozen into the CMB when the average temperature of
the universe dropped sufficiently to allow neutral atoms to form, and
the CMB photons began streaming freely through the
universe\footnote{For a good, low-level introduction to the concept of
  acoustic peaks in the CMB, see Chapter 9 of \citet{Ryden2003}.}.

The baryon abundance at this early time had several effects on the
acoustic peaks in the CMB, as did the overall mass abundance ($\omegam
h^2$). Figure \ref{fig:CMB_Power} shows how changes in these
abundances are reflected in the acoustic peaks. The location of the
first peak (and all subsequent peaks) is determined by the sound speed
at the epoch of last scattering. This sound speed increases when
$\omegab h^2$ increases, but is more sensitive to changes in $\omegam
h^2$, an increase of which results in a decrease in the sound speed
\citep[][ch.~9.8]{Mukhanov2005}. The relative heights of the acoustic
peaks is a further diagnostic of the baryon abundance. As seen in
Figure \ref{fig:CMB_Power}, odd-numbered peaks are higher than
even-numbered peaks in a universe with high $\omegab h^2$. This is
because an increase in the amount of massive baryons reduces the
frequency of acoustic oscillations
\citep[][ch.~8.7.3]{Dodelson2003}. Finally, the power spectrum
declines towards higher multipoles ($l$) due to a process known as
``Silk Damping''. This damping term is due to imperfections in the
photon-baryon coupling, and the characteristic damping scale is
influenced by $\omegab h^2$ \citep[][ch.~4.7]{Durrer2008}.

Through a combination of all the processes listed above, modern CMB
observations have been able to place strong constraints on both
$\omegab h^2$ and $\omegam h^2$. Two of the most noteworthy such
measurements come from the \textit{Wilkinson Microwave Anisotropy
  Probe} \citep[WMAP][]{Bennett2003} and the \textit{Planck} Satellite
\citep{PlanckResultsI}. Different constraints are placed on these
parameters depending on what additional data\footnote{\Ho, BAO,
  Polarization, etc.} is included in the analysis. We will take the
median value (from each paper) of parameters derived through these
various means, and we list the uncertainty as the systematic difference between
the inferred parameters (which is of the same order, but typically
larger than the statistical uncertainty of any one value).

From the 9-year data release of WMAP \citep{Hinshaw2013}, we take
values $\omegab h^2 = 0.02229 \pm 0.00035$ and $\omegac h^2 = 0.1138
\pm 0.0032$. From the results paper of Planck
\citep{PlanckResultsXVI}, we take the values $\omegab h^2 = 0.022115
\pm 0.0001$ and $\omegac h^2 = 0.11957 \pm 0.0011$. These values allow
us to constrain the cosmic baryon fraction as:
\begin{align}
\fb &= 0.164 \pm 0.004 \,\,(WMAP) \nonumber\\
&= 0.156 \pm 0.002 \,\,(Planck) \nonumber
\end{align}
It is a commonly
discussed fact \citep[e.g.~][]{Spergel2013} that the Planck results
yield \omegam significantly higher and \Ho significantly lower than
other previous measurements, including WMAP. Throughout this paper, we
will 

\begin{figure*}[hbtp]
\plotonebig{CMB_Power}
\caption{The variation in the CMB power spectrum's acoustic peaks due
  to variations in cosmic abundance parameters. The thick black line
  represents a fiducial universe with $\omegam h^2 = 0.16$, $\omegab
  h^2 = 0.021$, and $\omegal = 0.7$. Other lines represent the results
  from changes to these parameters, notably an increase in peak height
  and rightward shift of peaks ($\Delta l > 0$) with an increase in
  $\omegab h^2$. A decrease in $\omegam h^2$ has similar (but
  distinguishable) effects. Figure from \citet[][Fig.~8.19]{Dodelson2003}.}
\label{fig:CMB_Power}
\end{figure*}    

\afterpage{\clearpage}


\section{The ``Missing Baryon'' Problem...}
\label{s.Missing}
\subsection{In Clusters}
\label{s.Missing.Clusters}
\subsection{In Galaxies}
\label{s.Missing.Galaxies}
And some more stuff.\\


%=====================================================================
%Senior Thesis Chapter 2
%Ben Cook '14 (bacook@)
%Adviser: Neta Bahcall
\chapter{Observations and Data Analysis}
\label{chap:Data}

\section{Total Mass in Groups and Clusters}
\label{sec:Mass}

The total mass in groups and clusters can be measured in a number
of ways. The most direct method of calculating the mass profile of a
large halo is through gravitational lensing of the light of background
sources behind the halo. Strong gravitational lensing occurs when the
background object (ex: a high-redshift galaxy) is magnified and
severely warped by the gravitational potential of the lens (foreground
cluster). This allows a very accurate measurement of the mass of the
lens, but occurs only rarely when the background/foreground are in a
particular alignment. More commonly utilized is the technique of weak
gravitational lensing
\citeeg{Umetsu2009,Sheldon2009,VonderLinden2014}, where small
distortions of an immense number of background objects are used to
statistically determine the mass profile of a foreground halo.

Another common method of estimating the mass of groups and clusters is
through the assumption of \textit{hydrostatic equilibrium} (HSE). If
the gas in clusters is in HSE, then the pressure gradient offsets the
gravitational force:
\begin{align}
\frac{d\textrm{P(\textrm{r})}}{d\textrm{r}} &=
-\frac{\textrm{GM}(<\textrm{r})\rho(\textrm{r})}{\textrm{r}^2}
\nonumber
\end{align}
Therefore, assuming the ICM is in hydrostatic
  equilibrium, the total mass profile can be reconstructed from the
  observed gas density and pressure profiles:
\begin{align}
 \textrm{M}(<\textrm{r}) &=
 -\frac{\textrm{r}^2}{\textrm{G}~\rho_{gas}(\textrm{r})}\frac{d\textrm{P}(<\textrm{r})}{d\textrm{r}}\nonumber
 \intertext{Alternatively, the total mass can be calculated using the
   density and temperature profiles, assuming the ICM behaves as an
   ideal gas, P(r)=n(r)kT(r):} \textrm{M}(<\textrm{r}) &=
 -\frac{k~\textrm{T}(\textrm{r})~\textrm{r}}{\textrm{G}}\left(\frac{d\log{\rho_{gas}(\textrm{r})}}{d\textrm{r}}
 + \frac{d\log{\textrm{T}(\textrm{r})}}{d\textrm{r}}\right)\nonumber
\end{align} 
Gas density is typically measured using X-ray observations, as the gas
density is easily calculated from the X-ray surface
brightness. Temperature can be determined spectroscopically from X-ray
observations, and pressure can be measured directly through the
thermal SZ effect. 

The accuracy of the total mass derived through the assumption of
hydrostatic equilibrium (or the ``hydrostatic mass'') is highly
debated, as sources of non-thermal pressure (including cosmic rays,
merger-induced shocks, and AGN feedback) can invalidate the assumption
of hydrostatic equilibrium. Additionally, deviations from spherical
symmetry can bias hydrostatic mass estimates, as many gas measurements
are sensitive to the integrated profile along the
line-of-sight. Comparing hydrostatic masses to masses derived through
weak-lensing analysis can help constrain the bias inherent in the HSE
assumption, although weak-lensing masses are also sensitive to the
ellipticity of clusters.

The total matter density distribution is often modeled by the NFW
profile, an analytical equation first proposed by \citet{Navarro1996}
to describe the ``universal density profile'' of simulated dark matter
halos, regardless of size. The NFW profile has the form:
\begin{equation}
\rho_m(\radius) = \frac{\delta_c\rhocrit}{(\textrm{r/r}_s)(1 + \textrm{r/r}_s)^2}~.\nonumber
\end{equation}
r$_s$ is a characteristic radius representing the central core of the
dark matter halo, $\rhocrit$ is the critical matter density, and
$\delta_c$ is a normalization constant which sets the characteristic
overdensity of the cluster. Halos which fit the NFW profile are
self-similar, in that only $\radius_s$ is dependent on mass. When
$\radius \approx \radius_s$, the density profile decreases slowly
($\rho_m \propto \radius^{-1}$), while at $\radius \gg \radius_s$ the
profile steepens to $\rho_m \propto \radius^{-3}$. Observations of
group and cluster halos consistently find that the total mass profile
follows the NFW profile well
\citeeg{Vikhlinin2006,Mandelbaum2008,Sheldon2009,Umetsu2009}.

Several ``mass proxies'' have been used to estimate the total mass of
clusters. Mass proxies are typically easily-observed quantities that
are found to correlate strongly with the total cluster mass. Examples
include the X-ray temperature ($\textrm{kT}_\textrm{X}\approx1$ -- $10$ \keV{}
for groups and clusters), and $\textrm{Y}_{\textrm{X}} =
\Mass_{gas}\textrm{T}_\textrm{X}$. Using mass proxies allows observers
to place general constraints on the mass of a cluster without
requiring deep observations to recover the true gas or mass density
profiles. The assumption of hydrostatic equilibrium can also affect
the determination of mass through this method, as many Mass-Proxy
relations are calibrated against hydrostatic masses of clusters
\citeeg{Arnaud2007,Arnaud2010}.

\section{Cluster Gas Mass Fraction}
\label{sec:Gas}
The baryonic content of galaxy groups and clusters is dominated by hot
gas in the intracluster medium (ICM). Until very recently, the most
sensitive X-ray and SZ observations were only able to constrain the
gas mass in the ICM in the inner regions of groups and clusters,
typically to around \rfive{} \citeeg{Vikhlinin2006, Arnaud2007,
  Sun2009}. \todo{Any more?} Because \rvir{} is about twice \rfive{},
these observations only probe the inner $\sim \frac{1}{8}$ of the
virial volume of group and cluster halos. In order to measure the
baryon fraction within groups and clusters, it is essential to
consider the gas within a volume substantially larger than that within
\rfive{}. Here, we describe the relevant observations of groups and
clusters which measure both the ICM and total mass to the outskirts of
the dark matter halo. Because very few telescopes retain the
sensitivity required to measure the gas density in the outskirts of
clusters, we also discuss a method of using observed gas density
profiles to extrapolate observed gas fractions to higher radii.

\subsection{Observations}
\label{sec:Gas.Observations}
\textbf{\citet{Vikhlinin2006}} derived the gas and total mass profiles
of 10 low-redshift (median redshift z$ = 0.06$) relaxed clusters using
long-exposure \textit{Chandra} observations. The clusters have a
median mass $\Mvir = 7.3 \power{14}$ \Msun, and range from $\Mvir =
1.1\power{14}$ -- $1.5 \power{15}$ \Msun. Temperatures range from $\textrm{kT}
= 2$ -- $9$ \keV. The authors measured X-ray temperature and surface
brightness profiles to approximately \rfive{}. They modeled the
surface brightness profile (which is proportional to n$_e$n$_p$) to
recover the gas particle density, $\rho_{gas}($r$)$. The total mass
(\Mfive) was derived by solving the equation of hydrostatic
equilibrium, using the observed density and temperature profiles, and
is well-fit by an NFW profile in most cases. The integrated gas
density and total mass profiles were used to derive the gas fraction
interior to \rfive{}, $\fg(<\rfive)$. This gas fraction ranges widely
from cluster to cluster, from $6\%$ to $14\%$, with median
$11\%$. These observations were also used to derive a useful scaling
relation between \Mfive{} and the X-ray temperature T:
\begin{equation}
\label{eq:M-T}
\Mfive{} = (2.97 \pm 0.15)\times10^{14}~\Msun~h_{70}^{-1}
\left(\frac{\textrm{T}}{5~\textrm{keV}}\right)^{1.58 \pm 0.11}.
\end{equation}

\textbf{\citet{Arnaud2007}} used very similar methods to derive the
gas and total mass profiles of 10 low-redshift (median redshift z$ =
0.09$) relaxed clusters from \textit{XMM-Newton} observations. The
clusters range in mass from $\Mvir = 1.2\power{14}$ -- $1.16 \power{15}$ \Msun,
with a median of $4.2 \power{14}$ \Msun, and temperatures vary from
$\textrm{kT} = 2$ -- $8$ \keV. The total mass also relies on
the assumption of hydrostatic equilibrium, and was extrapolated from
$\sim$r$_{700}$ to \rfive{} using an NFW profile.  \fg{} was derived
out to \rfive{} for these clusters, varying from $5.5\%$ to $16\%$,
with median $11\%$, similar to the \citet{Vikhlinin2006} measurements.

\textbf{\citet{Sun2009}} analyzed the gas fraction in 43 groups from
archival \textit{Chandra} observations. All the groups are at low
redshifts ($z \lsim 0.1$). Of these 43 observations, 11 were sensitive
enough to measure the X-ray surface brightness to \rfive{}, while an
additional 12 measured surface brightness to r$_{1000}$ and were
extrapolated to \rfive{}. The total mass of the 23 best-measured
groups ranges from $\Mvir = 2.0\power{13}$ -- $2.1 \power{14}$ \Msun, with a
median of $1.1 \power{14}$ \Msun, and ICM temperatures range from
$\textrm{kT} = 0.7$ -- $2.7 \keV$. The total mass (assuming hydrostatic
equilibrium) and gas mass were calculated using similar principles to
\citet{Vikhlinin2006}, with errors estimated by using 1000 artificial
profiles generated from Monte-Carlo simulations. $\fg(<\rfive)$ for
these 23 groups ranges from $5\%$ -- $11\%$, with a median of $8\%$,
lower than for the more massive clusters of \citet{Vikhlinin2006} and
\citet{Arnaud2007}.

The above three samples were combined in the analysis of
\textbf{\citet{Giodini2009}} (G09), which used all 10 clusters from
\citet{Vikhlinin2006}, all 10 clusters from \citet{Arnaud2007}, and 21
of the 23 best-measured groups from \citet{Sun2009} to study the
group/cluster gas mass fraction over a wide range of halo masses. The
authors bin the 42 groups and clusters logarithmically by mass,
highlighting that lower mass halos have significantly lower gas
fractions. The best-fit trend is:
\begin{equation}
\fg(<\rfive) = (9.3 \pm 0.2)\power{-2}~h_{70}^{-3/2}~
\left(\frac{\Mfive}{2\power{14}\Msun}\right)^{0.21 \pm 0.03}.
\end{equation}
Figure \ref{fig:Giodini_Fgas} shows the distribution of the observed
gas fractions, as a function of halo mass, measured by the three works
above. We will use the G09 bins as 5 independent samples of \fg{} for
different masses. 
\begin{figure*}[hbtp]
\plotonebig{Giodini_Fgas}
\caption{The dependence of $\fg(<\rfive)$ on \Mfive{} ($\sim
0.73\Mvir$), as presented in \citet{Giodini2009}. The light-grey
  points represent individual group/cluster observations from
  \citet{Vikhlinin2006}, \citet{Arnaud2007}, and \citet{Sun2009},
  while the dark points are the average gas fractions, binned
  logarithmically with mass. Lower-mass halos show significantly lower
gas fractions, with $\fg(<\rfive)$ scaling roughly as
$\Mfive^{0.21}$. }
\label{fig:Giodini_Fgas}
\end{figure*}    


\afterpage{\clearpage}

Recent results from the \Planck{} satellite detect the ICM using the
Thermal SZ effect, which measures the integrated line-of-sight gas
pressure. \textbf{\citet{PlanckIntV}} (PC13) derives a stacked pressure
profile for 62 massive clusters which have archival observations by
\XMM. The cluster sample \citep[detailed in][]{PlanckEarlyXI} includes
clusters of mass $\Mvir = 3.3\power{14}$ -- $2.7 \power{15}$ \Msun,
with median mass approximately $\Mvir = 8.70 \power{14}$ \Msun. X-ray
temperatures range from $\textrm{kT} = 3.4$ -- $13 \keV$. Total mass
(\Mfive) was derived from a scaling relation with the quantity
$\textrm{Y}_X = \textrm{M}_{gas}\textrm{T}_X$, an easily-observable
quantity that has been found to be a good mass proxy. The scaling
relation in question \citep{Arnaud2010} was calibrated against X-ray
derived hydrostatic masses, and so the total mass profile of the
stacked \Planck{} clusters assumes hydrostatic equilibrium. The total
mass beyond \rfive{} was calculated assuming an NFW profile. The
stacked pressure profile is measured to unprecedented scales ($3\rfive
\approx 1.6\rvir$), although the X-ray temperature profile measured by
\XMM{} only extends to \rfive{}, so the authors extrapolated the
observed temperature profile to $3\rfive{}$ to match the pressure
observations.

Assuming the ICM acts as an ideal gas ($\textrm{P} \propto
\textrm{n}_\textrm{e}\textrm{kT}$), the authors inverted the pressure
and temperature profiles to retrieve the gas density profile and
derive $\fg{}(\textrm{r})$ out to 3\rfive{}\footnote{The authors also
  derive the gas-fraction assuming a conservative case in which the
  ICM is isothermal beyond \rfive{}, resulting in lower \fg{}.}. The
reconstruction of the temperature profile was initially flawed, and
the correct gas fraction profile was given in a corrigendum,
\citet{PlanckIntVb}. \fg{} increases from \rfive{} to \rvir{}
\citep[as predicted by][see \ref{sec:Gas.Extrapolation}]{Rasheed2011},
reaching a peak of $\approx 15\pm2\%$ at $1.6\rvir{}$.

\textbf{\citet{Eckert2013b}} (E13) combined the stacked pressure profile
from \citet{PlanckIntV} with a stacked X-ray surface-brightness
profile that directly constrains the gas density to \rtwo{}. The X-ray
observations were performed with \Rosat{}, on a sample of 31 clusters
($z\lsim0.2$) of temperatures $\textrm{kT} = 2.5$ -- $9$ \keV, with
median $\textrm{kT} = 6.5$ \keV. The cluster masses range from $\Mvir
\approx 1.4 \power{14}$ to $1.0 \power{15} \Msun$, with median $\Mvir =
6.0\power{14} \Msun$\footnote{The authors do not give the masses of
  the clusters, so these values are taken from the M-T relation of
  \citet{Vikhlinin2006} (equation \ref{eq:M-T}).}. The \Planck{}
pressure profile, combined with the gas density profile, is used to
directly calculate the total mass, assuming hydrostatic
equilibrium. This is different from the method used by
\citet{PlanckIntV}, which assumed an NFW profile. However, both
estimates rely on the assumption of hydrostatic equilibrium either
explicitly or implicitly through calibration of the
$\textrm{Y}_\textrm{X}-\Mfive$ relation. 18
clusters are in common between the \Rosat{} and \Planck{} samples, and
the authors separate them into cool-core (CC, 6 clusters) and
non-cool core (NCC, 12 clusters) categories. The gas fraction profile
is calculated separately for the two categories, and the authors find
that NCC clusters have significantly higher gas fractions within
\rtwo{} ($0.169 \pm 0.010$) than relaxed, CC clusters do ($0.134\pm
0.011$), suggesting that the irregular, non-spherical morphologies of
the disturbed clusters may bias the gas fractions high. They also find
that \fg{} increases from \rfive{} to \rtwo{} ($\fg(<\rfive) \approx
0.12$ for CC clusters). 

\textbf{\citet{Umetsu2009}} (U09) observed the ICM of four very massive
($\Mvir \gsim 1\power{15}$ \Msun, $\textrm{kT} \approx 9$ -- $10$ \keV)
clusters using Thermal SZ measurements with the \textit{AMiBA} CMB
telescope. After deriving pressure profiles from the SZ effect, the
authors calculated the gas density profile using archival X-ray
temperature measurements and a theoretical temperature-profile
\citep{Komatsu2001}. The authors use \textit{Subaru} observations to
derive the cluster masses with weak-lensing analysis. The average gas
fraction is calculated to the limit of the SZ observations, \rtwo, and
is found to be $\fg(<\rfive) = 0.126 \pm 0.019 \pm 0.016$ within
\rfive{} and $\fg(<\rtwo) = 0.133 \pm 0.020 \pm 0.018$ within
\rtwo{}. The two uncertainties on each fraction are the statistical
error and cluster-to-cluster standard deviation, respectively. These
observations also find that \fg{} increases with radius beyond
\rfive{}, in agreement with \citet{PlanckIntV} and
\citet{Eckert2013b}. We emphasize that the total mass for these
clusters are \textit{not} dependent on the assumption of hydrostatic
equilibrium (because they were derived using weak-lensing
measurements), so comparisons of \fg{} from this sample to the same
value for samples which assume HSE can put constraints on the validity
of the HSE assumption.

Table \ref{tab:F_gas} lists the samples described above with the most
important characteristics of each sample, including median mass
($\Mvir$), whether that mass is derived assuming HSE, and the gas
fraction at all directly-observed radii.

\begin{table}[hbt]
\caption{Observed Gas Fraction Measurements in Groups/Clusters}
\scriptsize
\begin{tabular}{llccccc}
\hline \hline\\
\footnotesize \textbf{Reference} & \footnotesize \textbf{\#{} Clusters} & \footnotesize \textbf{$\left<\Mvir\right>$ (\Msun)} & \footnotesize \textbf{HSE?} & \footnotesize \textbf{$\fg{}_{,500}$} & \footnotesize \textbf{$\fg{}_{,200}$} & \footnotesize \textbf{$\fg{}_{,vir}$} \\
\footnotesize (1) & \footnotesize (2)& \footnotesize (3)& \footnotesize (4)& \footnotesize (5)& \footnotesize (6)& \footnotesize (7) \\\\
\hline
G09 Bin 1 & 2 & $2.9\power{13}$ & $\surd$ & $0.074 \pm 0.028$ & & \\
\phantom{G09} Bin 2 & 7  & $7.0\power{13}$ & $\surd$ & $0.068 \pm 0.005$ & & \\
\phantom{G09} Bin 3 & 17 & $1.7\power{14}$ & $\surd$ & $0.080 \pm 0.003$ & & \\
\phantom{G09} Bin 4 & 5 & $4.1\power{14}$ & $\surd$ & $0.103 \pm 0.008$ & & \\
\phantom{G09} Bin 5 & 10 & $9.8\power{14}$ & $\surd$ & $0.123 \pm 0.007$ & & \\
PC13 & 62 & $8.7\power{14}$ & $\surd$ & $0.125\pm0.005$ & $0.137\pm0.003$ & $0.145\pm0.01$\\
E13 - CC & 6 & $5.9\power{14}$ & $\surd$ & $0.115\pm0.010$ & $0.134\pm0.011$ & \\
E13 - NCC & 12 & $5.9\power{14}$ & $\surd$ & $0.128\pm0.010$ &
$0.169\pm0.010$ &\\
U09 & 4 & $7.6\power{14}$ & & $0.126\pm0.025$ & $0.133\pm0.027$ & \\
\hline
\end{tabular}
\caption*{\small{(1) G09, PC13, E13, and U09 stand for
    \citet{Giodini2009}, \citet{PlanckIntV}, \citet{Eckert2013b}, and
    \citet{Umetsu2009}, respectively. CC (NCC) represents the sample
    of cool-core (non-cool core) clusters.\\ (2) The number of
    clusters in each sample.  (3) The median virial mass of the
    clusters. \\ (4) $\surd$ marks that the total mass assumes
    hydrostatic equilibrium.\\ (5) $\fg(<\rfive)$ (6) $\fg(<\rtwo)$
    (7) $\fg(<\rvir)$\\ }}
\label{tab:F_gas}
\end{table}


\subsection{Extrapolation of Gas Density Profiles}
\label{sec:Gas.Extrapolation}
As seen above, very few observations retain the necessary sensitivity to
measure the gas density all the way to \rvir{}. Therefore, to
constrain the gas fraction within the entire halo, we can extrapolate
the observed gas mass profile (at \rtwo{} or \rfive{}) to higher
radius by assuming a power-law profile:
\begin{equation}
\rho_{gas}(\textrm{r}) \propto \textrm{r}^{-\alpha_g}, \nonumber
\end{equation}
where $\alpha{}_g$ is the slope of the gas density profile, which can in
general change as radius increases. The total
matter density can be similarly modeled,
\begin{equation}
\rho_{m}(\textrm{r}) \propto \textrm(r)^{-\alpha_{m}}, \nonumber
\end{equation}
with $\alpha{}_m$ the slope of the total mass density profile. At
large radii, the full equation for the gas fraction simplifies to approximately:
\begin{align}
\fg(<\textrm{r}) = \frac{\textrm{M}_{gas}}{\textrm{M}_{tot}} &=
\frac{\int_0^r 4\pi{}r'^2dr'\rho_{gas}(\textrm{r}')}{\int_0^r
  4\pi{}r'^2dr'\rho_{m}(\textrm{r}')} \nonumber \\ & \approx
\frac{\rho_{gas}(\textrm{r})}{\rho_{tot}(\textrm{r})} \nonumber\\ &\propto \textrm{r}^{\alpha_{m} - \alpha_{tot}}. \nonumber
\end{align}
Therefore, the gas fraction can be extrapolated to larger radii using
the difference in slopes between the gas density and total mass
density profiles.

\textbf{\citet{Rasheed2011}} (R11) used this approach to extrapolate
the gas fraction of the G09 cluster samples to \rvir{}. X-ray and SZ
observations show that the gas density decreases more slowly with
radius than the total mass density ($\alpha_{m} > \alpha_{gas}$),
suggesting that the gas fraction should increase when the cluster
outskirts are considered. The authors hoped to place constraints on
the amount of ``missing baryons'' within the virial volume of
clusters.

R11 used a large survey of the literature to recover X-ray
measurements which constrain the gas density slope out to
\rfive{}. These measurements include observations with \Rosat{},
\Chandra, \XMM, and \textit{Suzaku}, and cover a wide range of cluster
masses and temperatures. Averaging over the many observations, the
authors find that the gas density slope at \rfive{} steepens with more
massive clusters, with $\alpha_{gas}$ ranging from $\approx 1.8 \pm
0.2$ for poor clusters (G09 bin 2, $\Mvir \approx 7\power{13}\Msun$)
to $\approx 2.3 \pm 0.02$ for the most massive G09 bin ($\Mvir
\approx9.8\power{14}\Msun$). 

Compared to the gas density profile, the total density (NFW) profile
is significantly steeper in the outer regions of the halo. In the mass
range of the G09 groups and clusters, the NFW profile has a slope of
$\alpha_m = 2.6$ between \rfive{} and \rtwo{}, and steepens to
$\alpha_m = 2.7$ in the region \rtwo{} to \rvir{}. Therefore, R11
predicted that the gas fraction rises significantly above
\rfive{}. Because $\alpha_{gas}$ increases with cluster mass, the gas
fraction is predicted to rise more quickly with radius for groups and
poor clusters ($\fg\propto \textrm{r}^{0.8}$ for G09 bin 2) than for
rich clusters ($\fg \propto \textrm{r}^{0.3}$ for G09 bin 5). For
these two bins, this model predicts increases in \fg{} by a factor of
roughly $1.6$ and $1.2$, respectively, from \rfive{} to \rvir{}. This
offers an explanation for why the missing baryon problem is more
severe in lower-mass clusters: the shallower gas profile implies the
ICM is spread farther out in lower-mass halos than in very massive
ones.

We adopt R11's extrapolation model in order to approximate the gas
fraction at high radius in the samples which do not measure \fg{} to
\rvir{} (all except PC13). $\alpha_{gas}$ for each sample in Table
\ref{tab:F_gas} is taken from the temperature-slope relation in R11,
we do not extrapolate the gas profiles for any individual cluster
sample with coverage beyond \rfive{}. We assume $\alpha_m$ as
above for the NFW profile. We extrapolate \fg{} from the maximum
observed radius, r$_a$, to a larger radius r$_b$ using:
\begin{equation}
\fg(<\textrm{r}_b) =
\fg(<\textrm{r}_a)\left(\frac{\textrm{r}_b}{\textrm{r}_a}\right)^{\alpha_m - \alpha_{gas}}.
\end{equation}
 For example, extrapolating the gas fraction of G09's bin 5 from \rfive{} to \rtwo{}:
\begin{align}
\fg{}(<\rtwo) &=
\fg{}(<\rfive)\left(\frac{\rtwo}{\rfive}\right)^{\alpha_m -
  \alpha_{gas}} \nonumber\\
&\approx .103~(1.45)^{2.6 - 2.3} \nonumber \\
&\approx .115 \nonumber
\end{align}

To calculate the uncertainty on the extrapolated gas fraction, we
propagate the errors in $\fg(<\textrm{r}_a)$ (or $\fg{}_{,a}$) and in
$\alpha_{gas}$, assuming no significant uncertainty exists in
$\alpha_m$ or r$_b$/r$_a$. The fractional errors add in quadrature.
\begin{align}
\frac{\Delta\fg{}_{,b}}{\fg{}_{,b}} &=
\sqrt{\left(\frac{\Delta\fg{}_{,a}}{\fg{}_{,a}}\right)^2 +
  \left(\frac{\Delta\left(\textrm{r}_b/\textrm{r}_a\right)^{\alpha_m-\alpha_{gas}}}{\left(\textrm{r}_b/\textrm{r}_a\right)^{\alpha_m-\alpha_{gas}}}\right)^2}\nonumber\\
\intertext{The uncertainty in the right term is}
\Delta\left(\textrm{r}_b/\textrm{r}_a\right)^{\alpha_m-\alpha_{gas}}
& =
\left(\textrm{r}_b/\textrm{r}_a\right)^{\alpha_m-\alpha_{gas}}\ln{\left(\textrm{r}_b/\textrm{r}_a\right)}\Delta\alpha_{gas}\,,\nonumber\\
\intertext{yielding the final result:}
\frac{\Delta\fg{}_{,b}}{\fg{}_{,b}} &=
\sqrt{\left(\frac{\Delta\fg{}_{,a}}{\fg{}_{,a}}\right)^2 +
  \left(\ln{\left(\textrm{r}_b/\textrm{r}_a\right)}\Delta\alpha_{gas}\right)^2}
\end{align}

The gas density profile is expected to steepen at very large radii,
such that it eventually matches the NFW profile \citeeg{Umetsu2009},
which translates to the gas fraction asymptotically approaching a constant
value. At large enough radius, extrapolation of the gas fraction as
described above will, therefore, become invalid, as $\alpha_{gas}$
will not remain fixed. The range at which the gas density steepens
significantly is not known, however, as observational data does not
currently constrain $\alpha_{gas}$ far beyond \rfive. R11's assumption
that this slope remains constant to \rvir{} is therefore a
questionable one, but no simple alternatives exist. We also assume
$\alpha_{gas}$ remains constant to \rvir{} (and slightly beyond), and
emphasize that the our calculation of the gas fraction will be biased
high if the gas density slope steepens significantly beyond
\rfive{}. In Chapter \ref{chap:Results}, we discuss how our results
may be able to constrain the evolution of this density slope. 

\section{Cluster Stellar Mass Fraction}
\label{sec:Stellar}

The integrated stellar mass of groups and clusters is also an
important (although subdominant) reservoir of baryons in these large
halos. The stellar mass of clusters comes almost entirely from the
stellar content of the individual cluster galaxies.

G09 approximated the stellar content of a large number ($>90$) groups
and clusters from the COSMOS survey. Using optical and infrared
observations from SUBARU, the authors fit a broad-band spectrum to the
clusters and used these spectral energy distributions to derive photometric redshifts for
the sample. Converting the IR luminosity of detected galaxies to
stellar mass, and accounting for the entire predicted galactic mass
function, G09 approximated the stellar fraction in clusters of sizes
$\power{13}$ -- $\power{15}$ \Msun. At \rfive{}, the stellar fraction
was found to be significantly 

\section{Galaxy Mass Fractions}
\label{sec:Galaxy}

\subsection{The Circumgalactic Medium}
\label{sec:Galaxy.CGM}

\subsection{Constraints on Other Galactic Components}
\label{sec:Galaxy.Components}

\subsection{Estimate of Galactic-Halo Baryon Fraction}
\label{sec:Galaxy.Fraction}


%Senior Thesis Chapter 3
%Ben Cook '14 (bacook@)
%Adviser: Neta Bahcall
\chapter{Results}
\label{chap:Results}

\section{The Distribution of Gas, Stars, and Baryons in Groups and Clusters}
\label{sec:Spatial}

From the observations presented in Chapter \ref{chap:Data}, we have
measurements of the gaseous component of the ICM in halos spanning the
entire mass range from poor groups to the most massive clusters. We
also have measurements of the stellar fraction well beyond the virial
radius, gathered from stacked optical and weak lensing
observations. Here, we show our results from this data and its
analysis.

In Figure \ref{fig:FgvR}, we present the gas fraction in these groups
and clusters as a function of radius, out to $1.2\rvir{}$. We use
the extrapolation methods described in Section
\ref{sec:Gas.Extrapolation} when necessary for observations which only
constrain the gas fraction within \rfive{} or \rtwo{}. The gas
fractions of each sample (both observed and extrapolated) are listed
in Table \ref{tab:F_gas_all}. We discuss the potential biases in using
the hydrostatic mass and extrapolating the gas density profile to
$1.2\rvir$ in Section \ref{sec:Limitations}.

\begin{figure*}[bt]
\plotonebig{Figs_Thesis/FgvR}
\caption{The cumulative hot gas fraction for each cluster sample in
  our study, plotted against the cluster-centric radius. Blue and cyan
  points (G09) represent data from \citet{Giodini2009}, red points
  (P1) are temperature hypothesis 1 from \citet{PlanckIntVb}, green
  points (E13) from \citet{Eckert2013a}, and yellow points (U09) from
  \citet{Umetsu2009}. See Section \ref{sec:Gas.Observations} for
  details of these sources. Many observations are extrapolated to
  $1.2\rvir{}$, as in \citet{Rasheed2011}. The f$_{gas}$ line
  represents the difference between the WMAP9 cosmic baryon fraction
  and the cosmic stellar fraction from \citet{Bahcall2014}. We discuss
  possible biases in using the HSE mass and extrapolating the gas
  density profile in Section \ref{sec:Limitations}. }
\label{fig:FgvR}
\end{figure*}    

%\afterpage{\clearpage}

\afterpage{\clearpage}

The gas fraction increases with radius in all groups and
clusters. Comparing to the ``expected'' gas fraction of $\approx
15.4\%$, which is the difference between the WMAP9 cosmic baryon
fraction ($16.4\%$) and the cosmic stellar fraction of BK14 ($1\%$),
nearly all clusters appear to contain the expected fraction of gas
within the virial radius or $1.2\rvir$. A portion of the ICM is,
therefore, ``hidden'' in the outskirts of groups and clusters, where
earlier observations interior to \rfive{} were not able to probe.

\begin{table}[hbt]
\caption{Gas Fraction in Groups/Clusters: Observed and Extrapolated}
\scriptsize
\begin{tabular}{lcccccc}
\hline \hline\\
\footnotesize \textbf{Reference} &\footnotesize
\textbf{$\left<\textrm{kT}\right>$} &\footnotesize \textbf{$\alpha{}_{gas}$} &\footnotesize
\textbf{$\fg{}_{,500}$} & \footnotesize \textbf{$\fg{}_{,200}$} &
\footnotesize \textbf{$\fg{}_{,vir}$} & \textbf{$\fg{}_{,1.2vir}$} \\
\footnotesize (1) & \footnotesize (2)& \footnotesize (3)&
\footnotesize (4)& \footnotesize (5)& \footnotesize (6) &
\footnotesize (7) \\\\
\hline
G09 Bin 1 & 0.93 \keV&$1.7\pm0.2$& $0.074 \pm 0.028$ & $0.100\pm0.039^*$ &
$0.131\pm0.052^*$ & $0.156\pm0.062^*$ \\
\phantom{G09} Bin 2 &1.6 \keV&$1.8\pm0.2$ & $0.068 \pm 0.005$ &
$0.091\pm0.009^*$ & $0.117\pm0.014^*$ & $0.137\pm0.019^*$ \\
\phantom{G09} Bin 3 & 2.8 \keV&$1.9\pm0.07$ & $0.080 \pm 0.003$ &
$0.103\pm0.005^*$ & $0.129\pm0.006^*$ & $0.149\pm0.008^*$ \\
\phantom{G09} Bin 4 &  5.0 \keV&$2.1\pm0.02$&  $0.103 \pm 0.008$ & $0.124\pm0.010^*$&$0.146\pm0.012^*$&$0.162\pm0.013^*$ \\
\phantom{G09} Bin 5 & 8.6 \keV&$2.3\pm0.02$&  $0.123 \pm 0.007$ &
$0.137\pm0.008^*$& $0.153\pm0.009^*$ & $0.165\pm0.010^*$\\
PC13 &$\dagger$ & $\dagger$& $0.125\pm0.005$ & $0.137\pm0.003$ &
$0.145\pm0.01$&$0.151\pm0.009$\\
E13 - CC &6.25 \keV& $2.2\pm0.05$& $0.115\pm0.010$ & $0.134\pm0.011$ & $0.153\pm0.013^*$&$0.167\pm0.014^*$ \\
U09 & 9.7 \keV&$2.4\pm0.1$& $0.126\pm0.025$ & $0.133\pm0.027$ & $0.143\pm0.029^*$&$0.151\pm0.031^*$\\
\hline
\end{tabular}
\caption*{\small{(1) Reference abbreviations as in Table
    \ref{tab:F_gas_obs}. \\ (2) The median temperatures of the
    groups/clusters in each sample. \\ (3) The gas density slope
    derived from R11.\\ (4) $\fg(<\rfive)$ (5) $\fg(<\rtwo)$ (6)
    $\fg(<\rvir)$ (7) $\fg(<1.2\rvir)$\\ *: Value
    represents extrapolation using the method of Section
    \ref{sec:Gas.Extrapolation}.\\ $\dagger$: No extrapolation required;
    T and $\alpha_{gas}$ not calculated. }}
\label{tab:F_gas_all}
\end{table}
 

Figure \ref{fig:FgvM} shows the group and cluster gas fraction as a
function of the virial mass of the system. The shortage of gas in
low-mass clusters is apparent at \rfive{}, with low-mass groups and
clusters falling further short of the expected fraction than larger
clusters. However, the shallower slope of the gas density profile in
low-mass systems leads to a dramatic increase in the gas fraction when
extrapolated to larger radius: from \rfive{} to \rvir{}, their gas
fraction has almost doubled. The gas distribution in low mass clusters
is more extended, likely due to the shallow gravitational potential
well. All systems from groups to massive clusters are consistent with
containing the expected gas fraction interior to
$\approx1.2\rvir$. Gas is less centrally concentrated than dark
matter, with a large fraction residing in the outskirts of groups and
clusters, but within $1.2\rvir$ the gas mass traces the total mass
well.

\begin{figure*}[hbt]
\plotonebig{Figs_Thesis/FgvM}
\caption{The cumulative hot gas fraction for each cluster sample in
  our study, plotted against the mean halo mass. Blue circles, green
  diamonds, and red squares show the gas fraction measured at
  r$_{500}$, r$_{200}$, and r$_{vir}$, respectively. The f$_{gas}$
  line represents roughly the expected hot-gas fraction, and is the
  the difference between the WMAP9 cosmic baryon fraction and the
  cosmic stellar fraction \citep[$\sim{}1\%$,][]{Bahcall2014}.}
\label{fig:FgvM}
\end{figure*}    

%\afterpage{\clearpage}

%\afterpage{\clearpage}

The mass of dark matter halos clearly has an important effect on the
distribution of baryons, as the virial mass appears to determine how
extended the baryonic distribution is. We therefore combine our
samples of groups and clusters into two bins: groups/poor clusters
($\Mvir{}<3\power{14}\Msun$), and rich clusters
($\Mvir{}>3\power{14}\Msun$). In Figure \ref{fig:FxvR}, we show the
averaged gas fraction in each bin, as a function of radius. Each point
represents the weighted mean of gas fraction at that radius, for all
clusters in that mass range. These mass bins translate roughly to
the medium-richness and high-richness bins of BK14,
respectively. Therefore, we also include the average stellar fraction,
which is measured far beyond $\rvir$.

\begin{figure*}[hbt]
\plotonebig{Figs_Thesis/FxvR}
\caption{The cumulative stellar \citep{Bahcall2014} and hot gas (Fig.
  \ref{fig:FgvR}) fractions for groups and clusters, as a function of
  cluster-centric radius. \citet{Bahcall2014} presented the stellar
  fraction for various cluster richness bins. The gas fractions of
  Figure \ref{fig:FgvR} have been sorted into corresponding bins, using the
  mass-richness relation of \citet{Sheldon2009b}.}
\label{fig:FxvR}
\end{figure*}    

\afterpage{\clearpage}

%\afterpage{\clearpage}

At \rfive{}, the average gas fraction is $\approx7.5\%$ in groups and
poor clusters and $\approx12\%$ in rich clusters. The gas fraction
increases steeply in groups and poor clusters, reaching about $13\%$
at \rvir{} and $15\%$ at 1.2\rvir{}. In rich clusters, the gas
fraction increases more slowly, reaching $15\%$ at \rvir{}, and $16\%$
at 1.2\rvir{}, representing the increased concentration of gas in high
mas clusters. The stellar fraction decreases from $2\%$ at \rfive{} to
$1.5\%$ at 1.2\rvir{} in the low-mass bin, and remains steady at $1\%$
in the high-mass bin. We note that the apparent steepening of the
\fg{} profile at \rvir{} is simply a relic of the logarithmic scale of
the X axis. As a whole, low mass clusters have lower gas fractions and
higher stellar fractions at any given radius, with both distributions
being more extended than in massive clusters.

We are now able to combine the gas fraction and stellar fraction for
clusters in these two mass ranges, yielding the total baryon fraction
distribution. This is presented in Figure \ref{fig:FbvR}. The overall
baryon fraction increases with radius, reaching the cosmic baryon
fraction at $\approx\rvir{}$ in massive clusters and at
$\approx1.2\rvir{}$ in groups and poor clusters. The entire baryonic
mass associated with the virial mass of groups and clusters is detected
or inferred within the dark matter halo, indicating that the baryonic
matter content of the universe is a strong proxy for the dark matter
distribution, when averaged on appreciably large scales.

\begin{figure*}[hbt]
\plotonebig{Figs_Thesis/FbvR}
\caption{The cumulative baryon fraction for groups and clusters, as a
  function of cluster-centric radius. The baryon fraction (f$_b$) is
  the sum of the cumulative stellar fraction and the cumulative hot
  gas fraction of Figure \ref{fig:FxvR}. Green squares represent the
  averaged fractions of groups and smaller clusters, while blue
  circles represent larger clutsers. The baryon fraction in large
  clusters appears to approach the cosmic fraction at large scales,
  and we extrapolate the value at larger scales as the value at
  $1.2\rvir{}$, an assumption we discuss in Section \ref{sec:Limitations.Slope}.}
\label{fig:FbvR}
\end{figure*}    

%\afterpage{\clearpage}

%\afterpage{\clearpage}

The stellar fraction is observed to approach a constant value at high
radii \citep{Bahcall2014}. The gas fraction is predicted to do the
same, as the gas density profile likely approaches the total mass
(NFW) profile at high radius \citep[e.g.,][and
  refs.~therein]{Umetsu2009}.  We approximate this by assuming that,
beyond $\approx 1.2\rvir{}$, the gas fraction reaches a constant
value, and represent this as an extrapolation of the baryon fraction
out to high radius in Figure \ref{fig:FbvR}. Therefore, current
observations are consistent with the picture that the baryon
distribution matches the distribution of matter well at all scales
larger than the virial radius of clusters, and implies that the cosmic
baryon fraction also populates the large-scale structure outside the
virial radius of groups and clusters. The extrapolation to $1.2\rvir$
and not beyond is arbitrary, as the gas density slope likely steepens
before this point. We discuss this simplification and the validity of
our approximation in Section \ref{sec:Limitations.Slope}.

\section{The Baryonic Content of Systems from Galaxies to Groups and Clusters}
\label{sec:Baryonic}

We have just shown that, within roughly the virial radius, dark matter
halos of both group and cluster sizes are observed to hold the entire
cosmic baryon fraction. Combined with the measurements of the baryonic
components in galactic systems (Section \ref{sec:Galaxy.Fraction}), we
are able to place global constraints on the baryon distribution within
halos ranging over three orders of magnitude in mass.

Figure \ref{fig:FbvM} presents the collection of observations on the
baryon fraction in halos of a wide range of masses. Shown, as a
function of mass, are the current limits on the baryon fraction for
the previously mentioned samples of galaxies, groups, and
clusters. Observations of the outskirts of groups and clusters show
that the baryon fraction reaches the cosmic value between \rvir{} and
$1.2\rvir{}$. We plot the group and cluster baryon fraction, which is
the sum of the gas fraction (Figure \ref{fig:FgvM}) and stellar
fraction (Figure \ref{fig:BK14}), at these two radii and
sorted by mass. Observations of galactic systems show that -- between the
stellar disk and ISM, cold CGM, warm CGM, and X-ray CGM -- the entire
baryon fraction could be contained within the virial volume of galactic
halos. We plot the baryon fraction range constrained by
\citet{Werk2014}, which includes lower and upper limits of
$\fb(<\rvir) = 9\% - 19\%$. The best-estimate value, taking the mean
of the ranges given for each component, is $14.5\% \pm 3\%$.

\begin{figure*}[hbt]
\plotonebig{Figs_Thesis/FbvM}
\caption{The cumulative baryon fraction (f$_b$) for galaxies, groups,
  and clusters, as a function of the average mass of the sample. The
  baryon fraction of groups and clusters (filled red points: within
  r$_{vir}$; open green points: within 1.2r$_{vir}$) is the sum of the
  cumulative stellar and hot gas fractions of Figures \ref{fig:BK14}
  and \ref{fig:FgvM}. The baryon fraction of L$^*$ galaxies (open blue
  point) is gathered from \citet{Werk2014}. The arrows indicate upper
  and lower-limits on the galaxy baryon fraction, while the error-bars
  represent the propogated error on the fraction, assuming independent
  uncertainties on the different galactic components (Section
  \ref{sec:Galaxy.Fraction}).}
\label{fig:FbvM}
\end{figure*}                
                                  
%\afterpage{\clearpage}

%\afterpage{\clearpage}

Groups and poor clusters are still slightly short of baryons within
the virial radius (shown by red points in Figure
\ref{fig:FbvM}). However, there is no reason to assume that the gas
density must cut-off suddenly at this relatively-arbitrary
point. Direct observations \citep{PlanckIntV} indicate that the gas
fraction in massive clusters continues to increase beyond
\rvir{}. Hence groups and poor clusters (where the baryon distribution
is observed to be more extended than in massive clusters) should
also have baryon reservoirs beyond this radius, which justifies
our choice to extrapolate \fg{} to $1.2\rvir$.

From galaxies to the most massive clusters, current observations are
consistent with the entire cosmic baryon fraction being contained
within the dark matter halo. Integrating out to \rvir{} and slightly
beyond, there is no significant shortage of baryons, even in low-mass
clusters and galactic systems, where previous estimates claimed a
dramatic lack of baryonic mass. Considering the entire range of mass,
there does not appear to be any strong correlation between enclosed
baryon fraction with mass. The baryonic mass traces the dark matter
mass in halos once their outer regions are included. Therefore, in
response to the question ``Where are the baryons in the universe'', we
show that baryons have fallen into and stayed within dark matter
halos, with some gas being expelled due to shock heating and other
processes, and some colder gas infalling into galaxies. But the cosmic
baryon fraction approximately traces the total mass distribution in
the universe within and beyond the relevant \rvir{}.



%Senior Thesis Chapter 4
%Ben Cook '14 (bacook@)
%Adviser: Neta Bahcall
\chapter{Discussion}
\label{chap:Discussion}

\section{Limitations and Observational Biases}
\label{sec:Limitations}

\subsection{Assumption of Hydrostatic Equilibrium}
\label{sec:Limitations.HSE}
As was highlighted throughout Chapter \ref{chap:Data}, there are
several different ways to calculate the total mass of group and
cluster halos. One key characteristic of a mass estimation method is
whether it relies on the assumption of hydrostatic equilibrium. The
so-called ``hydrostatic mass'' ($\Mass{}_{HSE}$) can be calculated by combining the gas
density profile with either the temperature or pressure profile, a
method ubiquitous in X-ray observations. Using mass-scaling relations
(such as the Y$_\textrm{SZ}$-\Mfive{} relation) may also be sensitive
to the assumption of HSE if the relation is calibrated against
hydrostatic masses. Masses which do not rely on the assumption of HSE
are primarily derived from gravitational lensing.

Hydrostatic masses could be systematically biased, relative to the
``true'' total mass (usually assumed to be the lensing mass, $\Mass{}_{WL}$) if there
are significant sources of non-thermal pressure in the ICM. These
could include kinetic bulk motions or magnetic fields. Hydrostatic
equilibrium assumes that the gravitational force (the total mass) is
offset by the pressure gradient, so assuming that only the gas
pressure contributes can lead to an incorrect calculation of the
mass. 

The magnitude of the hydrostatic mass bias is of paramount importance
to precision cosmology. Cosmological parameters (such as \omegam{})
derived from \Planck{} cluster counts and hydrostatic mass estimates
disagree significantly from values derived directly from the CMB power
spectrum, but a large hydrostatic mass bias ($ b= 1-
\Mass{}_{HSE}/\Mass{}_{WL} \approx 0.3$) could relieve the
observational tension \citep{Gruen2013,VonderLinden2014}. Such a large
bias on the total halo mass would also dramatically affect the gas
fraction derived using hydrostatic masses, limiting its use as a
cosmological probe \citeeg{Grego2001,Ettori2009b}. As the majority of
our \fg{} measurements in clusters were measured relative to
hydrostatic masses (with the exception of \citet{Umetsu2009}), our
results are likewise sensitive to the hydrostatic bias.

Cosmological simulations are a major tool used to constrain this bias.
True mass calculations can be compared to mock X-ray observations
which assume hydrostatic equilibrium, determining the bias factor as a
function of mass and overdensity. Simulations nearly unanimously
indicate that the hydrostatic mass is biased low compared to the true
halo mass ($b>0$) and is more significant towards the outskirts of
clusters or in unrelaxed clusters, where merger disruptions and bulk flows
become more significant. However, different simulations and physical
prescriptions place the bias anywhere from $5\%$
\citeeg{Lau2009,Meneghetti2010,Burns2010,Nelson2012} to $20\%$
\citeeg{Arnaud2007,Nagai2007,Battaglia2013}.

Observational constraints on the hydrostatic mass bias vary
widely. Some weak-lensing measurements of clusters suggest that
hydrostatic X-ray or SZ masses are biased low by $10\%$
\citep{Andersson2011,High2012}, while others indicate this bias is as
large as $20-30\%$
\citep{Arnaud2007,Ichikawa2013,VonderLinden2014}. Yet others, however,
find no significant difference between weak-lensing masses and
hydrostatic masses, with some hints that hydrostatic equilibrium
assumptions may even \textit{overestimate} the true mass in lower-mass
clusters \citep{Gruen2013, Israel2014}. Figure \ref{fig:Gruen}, from
\citet{Gruen2013}, shows the agreement in measured weak lensing and
hydrostatic masses. 

\begin{figure*}[hbtp]
\plotonebig{Copied_Figs/Gruen2013}
\caption{A comparison of weak lensing masses ($\Mass{}^{WL}_{500c}$)
  and X-ray hydrostatic masses ($\Mass{}^{X}_{500c}$) for SZ clusters,
  as presented in \citet{Gruen2013}. Error bars indicate the best-fit
  ``single-halo'' fit, while the filled symbols include corrections to
  $\Mass^{WL}$ for other structures in the field of view. In tension
  with simulations, the authors find that hydrostatic masses are not
  systematically biased low relative to the weak lensing mass.}
\label{fig:Gruen}
\end{figure*}    


\afterpage{\clearpage}

The issue of hydrostatic mass bias is far from solved. Due to the
inconsistent observational and simulated constraints, it is unclear
how large of a hydrostatic correction factor should be included in our
measurements of the gas fraction, or if one is even necessary. A bias
low in the hydrostatic mass would bias the gas fraction high, meaning
that groups and clusters are slightly more deficient of baryons at a
given radius. Our reservations to use a hydrostatic correction factor
are slightly justified by the fact that one of our samples (U09)
measures gas fraction against the weak-lensing mass, and this fraction
agrees well with gas fraction derived from the HSE
assumption.

\subsection{Gas Clumping in Cluster Outskirts}
\label{sec:Limitations.Clumping}

The primary means of deriving the gas density profile of clusters is
from measurements of X-ray surface-brightness, which scales with the
square of electron density. Due to this $n^2$ dependence, clumpy
structures in the ICM will emit more than their share of X-ray,
biasing gas density measurements high. The magnitude of this bias
depends on the smoothness of the ICM gas distribution, which can vary
widely from cluster to cluster. 

Simulations typically predict a clumping bias (overestimate of
$\Mass{}_{gas}$) of $\approx10-15\%$ \citep{Nagai2011,Battaglia2013},
which increases in unrelaxed clusters and towards cluster outskirts,
where recent interactions have a more significant dynamical
effect. This could explain the large differences in measured gas
fraction between CC and NCC clusters, although observations suggest
that the level of clumpiness is overestimated in simulations, and that
the average bias is below $10\%$ \citep{Eckert2013c}. The majority of our
cluster samples represent relaxed (CC) clusters, so we do not include
a clumping bias factor at this time. We also note that the P13
\citep{PlanckIntV} and U09 \citep{Umetsu2009} gas fractions are
derived from SZ measurements, which do not suffer from this clumping
bias, and agree well with our X-ray data sets. 

\subsection{Extrapolation of the Density Profile Slope}
\label{sec:Limitations.Slope}
The choice to extrapolate the gas fraction to $1.2\rvir{}$ and then
stop is arbitrary. The radius at which the gas profile steepens
substantially ($\alpha{}_{gas}$ increases) is not well constrained.
 Between \rfive{} (where $\alpha{}_{gas}$ is well
measured) and \rtwo{}, observations disagree whether the slope remains
roughly constant \citep{Dai2010} or steepens by roughly $10\%$
\citep{Ettori2009a}. Assuming the true evolution is somewhere in
between, the gas density slope should not change appreciably relative
to the total mass (NFW) profile, which steepens by about $4\%$ within
this range. Therefore, our assumption that $\alpha{}_{m}$ and
$\alpha_{gas}$ remain constant beyond \rtwo{} should roughly
approximate the increase of \fg{} with radius. 

It is expected that the slope will eventually asymptote to match the
total mass (NFW) profile \citep{Umetsu2009, Battaglia2013}, suggesting
that, at large radius, $\alpha_{gas}$ will steepen more quickly that
$\alpha_{m}$. The point at which this occurs is unknown, but it will
taper the growth of \fg{}, which should eventually reach a constant
value, similar to the stellar fraction.  \citet{PlanckIntV} finds that
the gas fraction in stacked \Planck{} clusters flattens out between
$1$ and $1.5\rvir{}$. We approximate this by extrapolating \fb{} as
constant above $1.2\rvir{}$. However, until temperature measurements
beyond \rfive{} improve, the true radius where the baryon fraction
reaches a maximum will remain unconstrained. 

\section{Comparison to Simulations}
\label{sec:Simulations}

\begin{figure*}[hbt]
\plotonebig{Copied_Figs/Battaglia_FbvM}
\caption{The gas, stellar, and baryon fraction (scaled by the cosmic
  baryon fraction) as a function of cluster mass from a series of
  simulations by \citet{Battaglia2013}. The baryon fraction within
  \rtwo{} is close to the cosmic value for the entire mass range of
  $10^{14}-10^{15}\Msun{}$, except for in the AGN Feedback model,
  where low-mass clusters have fewer baryons within \rtwo{}. These
  simulations are in great agreement with our results, although the
  simulations predict a large hydrostatic mass bias (Section
  \ref{sec:Limitations.HSE}) that has yet to be confirmed observationally.}
\label{fig:Battaglia_FbvM}
\end{figure*}    


\begin{figure*}[hbtp]
\plotonebig{Copied_Figs/Battaglia_FbvR}
\caption{The gas, stellar, and baryon fraction distributions (scaled
  by the cosmic baryon fraction) as a function of radius in the
  simulations of \citet{Battaglia2013}, binned in four parts by
  cluster mass. The stellar fraction and gas profiles match
  observations well, and the total baryon fraction approaches the
  cosmic value at high radius.}
\label{fig:Battaglia_FbvR}
\end{figure*}    


\begin{figure*}[hbt]
\plotonebig{Copied_Figs/Battaglia_Fgvz}
\caption{The gas fraction (scaled by the cosmic baryon fraction)
  within various radii, as a function of redshift, as measured by the
  simulations of \citet{Battaglia2013}, and binned by mass. There is
  no significant redshift evolution of the gas fraction within any
  given radius, except possibly within $2\rtwo{}$, suggesting that
  measurements of the cluster baryon fraction should be fairly
  redshift independent back to $z\approx1$, where massive clusters
  first begin to form.}
\label{fig:Battaglia_Fgvz}
\end{figure*}    





\section{Implications}
\label{sec:Implications}

\subsection{Deviations from Scale Invariance}
\label{sec:Implications.Invariance}

\subsection{Hierarchical Structure Formation}
\label{sec:Implications.Hierarchical}

\subsection{Where are the Baryons?}
\label{sec:Implications.Where}

%Senior Thesis Chapter 5
%Ben Cook '14 (bacook@)
%Adviser: Neta Bahcall
\chapter{Summary and Conclusions}
\label{chap:Conclusions}

In this thesis, we present a synthesis of observational constraints on
the distribution and abundance of baryons in dark matter halos over a
wide range of sizes. Baryons, while only $16\%$ of the total mass in
the universe, are the most easily detectable form of matter and are
incredibly important tools in understanding and observing the
formation of structure and dark matter halos in the universe. Previous
observations suggest halos are deficient in baryons relative to the
cosmic fraction, a discrepancy known as the ``Missing Baryon
Problem''. We trace the baryon distribution in halos from galactic
scales ($\Mvir{} = 10^{12}\Msun$) to groups ($\Mvir{}=10^{13}\Msun$),
and clusters ($\Mvir{} = 10^{14}-10^{15}\Msun$) to address the missing
baryon problem and to obtain a global picture of the distribution of
baryons relative to dark matter in the universe. 

Our group and cluster gas mass data is comprised of a collection of X-ray and
SZ measurements of the gas density, temperature, and pressure profiles
in the ICM. The total mass is derived either through the assumption of
HSE or by using weak lensing. When the gas fraction is not measured to
the virial radius, we extrapolated the observed gas fraction using gas
density profile slopes appropriate to the given halo mass or ICM
temperature. Our galaxy cluster data comes from a compilation by
\citet{Werk2014}, and includes absorption measurements of the
multiphase CGM, stellar masses, and ISM masses from HI surveys.

Our main results are as follows:
\begin{enumerate}
\item Although the gas fraction within \rfive{} in clusters is
  significantly lower than the cosmic baryon fraction, the gas density
  profile has a shallower slope than the total mass (NFW) profile,
  resulting in an increasing gas fraction at higher radius. 
\item The gaseous ICM is more extended in low-mass halos, explaining
  why the gas fraction is observed to be low within \rfive{} in these
  halos. The gas density slope in groups and poor clusters is
  shallower than in massive clusters, and our extrapolations predict that
  the gas fraction of all groups and clusters should converge to the
  cosmic value near \rvir{}.
\item The cluster stellar fraction at any given radius is higher in
  groups and poor clusters, although in clusters of all sizes the
  fraction asymptotically approaches the ``cosmic'' stellar fraction
  $0.01\pm0.004$ at high enough radius. 
\item Observational constraints on the gas content of galactic halos
  find the cool CGM can account for $25-50\%$ of the total baryonic
  mass. Current constraints of the stellar disk, ISM, and multiphase
  CGM are consistent with the galactic baryon fraction matching the
  cosmic value within the viral radius. 
\item Combining the observed baryonic components of galaxy, group, and
  cluster halos, we show that dark matter halos contain the cosmic
  fraction of baryons within approximately the virial radius, across
  three orders of magnitude in halo mass. 
\item The baryon distribution of the universe traces the dark matter
  distribution well, with no need for additional unseen or unbound
  reservoirs of baryons from halos. Averaged over scales larger than
  the virial radius, baryons map the total structure of the universe,
  and the baryonic mass of clusters is an effective proxy for the
  total mass.
\item The baryonic content of halos is not self-similar. Baryons in
  less-massive halos are pushed farther into the outskirts by feedback
  mechanisms and shallow gravitational potentials, not reaching the
  cosmic fraction until higher radii. The abundance of different
  baryonic components (particularly stellar mass and gas) also changes
  with cluster mass.
\item The consistent baryon fraction in galactic and cluster halos
  suggests that cluster dark matter and baryonic masses could be
  composed entirely of matter originally in galactic halos which fell
  into cluster halos. Using the ratio of stellar fractions, we
  show that galactic halos contribute \textbf{no less than $40\%$} of
  the total matter of clusters. 
\end{enumerate}

Improvements on the precision of our results will come from better
constraints of the ICM temperature profile at high radius (in
clusters) and of the mass in the warm CGM phase (in
galaxies). Additionally, further weak lensing calibrations are
required to constrain the magnitude of the hydrostatic equilibrium
bias on the total cluster mass. 

The baryonic content of the universe, while energetically small, is
what makes up all physical objects in the universe, what drives
astrophysical phenomena such as radiation and planetary formation, and
what eventually gave rise to life. Yet, until recently, baryons have
been overshadowed by dark matter as tools for studying cosmology of
the low-redshift universe and the growth of large-scale structure. Our
results show that the baryon distribution is an excellent proxy for
the dark matter distribution, and represents a new way of approaching
the growth and evolution of halos and structure in the universe. 


\footnotesize
\bibliography{Thesis}
\addcontentsline{toc}{chapter}{Bibliography}

\appendix

\end{document}
