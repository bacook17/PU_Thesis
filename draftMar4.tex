%Senior Thesis draft
%Advisor Michael Strauss
%March 4 2012

\documentclass[preprint]{aastex}

\usepackage{natbib}
\bibliographystyle{apj}
\usepackage{graphicx}
\usepackage{multirow}
\usepackage{color}

\begin{document}

\title{Discovery of High Redshift Type II Quasars in the SDSS BOSS Survey: Near Infrared Spectroscopy Follow-Up}
\author{Rachael M. Alexandroff }
\affil{Princeton University Department of Astrophysical Sciences}
\affil{ Peyton Hall, Princeton, NJ 08544, USA}
\email{rmalexan@princeton.edu}
\date{\today}

\newpage

\begin{abstract}

%Luminous AGN with their central engine obscured by gas and dust are classified as type II quasars.  Type II quasars at high redshift are suggested by AGN unification models and would help to account for the cosmic X-ray background as well as further our understanding of supermassive blackhole formation and evolution.  To date, most identified type II quasars at high redshift are radio-loud.  In this paper we present a sample of 207 type II quasar candidates at redshifts 1.55 $< z <$ 4.22 from the spectroscopic data of the Sloan Digital Sky Survey -III.  All objects presented have C IV Gaussian width less than 900km/s (FWHM $<$ 2100km/s) and exhibit a low continuum level.  We describe the selection procedure and then describe the sample's bulk propertites as well as presenting several sample spectra.  We also show the composite spectra of several subsets of our sample to better examine broad components noticed in some of our spectra.  Further study at different wavelengths would allow for confirmation of true obscuration in our sample.  

\end{abstract}

\newpage

\section{Introduction}

The extremely luminous regions at the center of some galaxies are known as Active Galactic Nuclei (AGN).  The first of these objects was identified in 1943 by Carl Seyfert.  The spectral energy density  of these objects are consistent over a wide range of frequencies suggesting these objects are not radiating blackbodies and implying a non-stellar explanation for their emission.  Current theories suggest that their extremely large luminosity ($>$10$^{48}$ erg/s) is attributable to the accretion of gas onto a supermassive black hole (SMBH)  of M $>$ 10$^6$M$_{\bigodot}$ or "central engine" at the center of the galaxy.  Rather than falling directly into the black hole, matter forms an accretion disk around the center as it spirals past the event horizon due to conservation of angular momentum.  Viscosity during rotation then converts kinetic energy into the heat and radiation observed.  The bolometric luminosity output of the accretion disk is related to its mass accretion rate by the equation below where $L_{disk}$ is the bolometric luminosity of accretion, $\epsilon_r$ is the radiative efficiency of the process and $\dot{M}_{acc}$ is the mass accretion rate onto the SMBH \citep{astrobook}.

\begin{equation}
L_{disk} = \epsilon_r \dot{M}_{acc} c^2
\end{equation}

The separation between Seyfert and quasar AGN is due to a difference in luminosity across a variety of wavelengths.  Quasars are the more luminous AGN over all wavelengths, while Seyferts are defined at some lower luminosity.  For example, low X-ray luminosity objects ($L_X < 10^{37} W$) are characterized as Seyfert galaxies while high X-ray luminosity objects ($L_X > 10^{37} W$) are characterized as quasars \citep{2006MNRAS.370.1479M}.  Optical astronomers use a cut in optical luminosity to distinguish between Seyferts and quasars but, as in the X-ray, this cut is somewhat arbitrary. 

It is believed that a unification model can be used to classify the properties of AGN where variety in the observed properties are the result of differences in orientation, luminosity and obscuration \citep{2002ApJ...571..218N}. It is postulated that the accretion disk of a SMBH is surrounded by a torus of gas and dust whose existence can make it difficult to view the central region around the SMBH at optical wavelengths.  According to the unified model an object is classified as obscured or type II if its orientation is such that the dusty torus lies along the line of sight whereas an object is classified as unobscured or type I if the obscuring torus does not lie along the line of sight  \citep{2006MNRAS.370.1479M}.  There is some evidence to support our "obscuration" hypothesis thanks to infrared (IR) observations.  When objects are viewed in the infrared what is observed is the radiation from heated dust.  Type II or obscured objects emit brightly in the IR though they do not emit as brightly in the optical, lending credence to the idea that much of their luminosity is obscured by a torus of hot dust. Thus, we can differentiate type II from type I objects because type II or obscured objects have a much higher ratio of $L_{IR}/L_{Optical}$ than do type I objects.  In addition, obscured object are expected to have hard X-ray spectra because obscuring gas along the line of sight from the central engine also absorbs soft X-rays but not hard X-rays \citep{2003AJ....126.2125Z}.  As a consequence of obscuration, type 1 objects have narrow forbidden but broad permitted (FWHM up to $10^4$ km/s) emission lines and a blue continuum while type 2 objectss have narrow forbidden and narrow permitted (FWHM $\preceq$ 1500 km/s) emission lines (\cite{hassen}).  

Forbidden emission lines are the result of energy level transitions that are not permitted by electric dipole selection.  Yet there is a discrete, and very small probability of these transitions occurring spontaneously if the material is of very low density because collisional de-excitations occur more rarely.  The most extreme laboratory vacuum cannot replicate these conditions on earth(resulting in the name "forbidden") but in the far less dense regions in the Interstellar Medium (ISM) of galaxies these conditions do occur.   

How wide an emission line is depends on its distance from the central region.  Orbital velocities of gas decrease at greater distance from the central potential well (including the central blackhole and the galactic bulge).  Gas closer to the central potential well experiences larger velocities translating spectrosopically into the doppler broadening of emission lines.  Farther from the central potential well, at lower velocities, there is less doppler broadening and the emission lines remain narrow.  This is why forbidden lines which can only exist in low-density regions far from the central region are always narrow and why permitted lines which can exist in high-density regions near the central region are usually broad.  As a result type 2 objects have both narrow permitted and forbidden lines because the emission lines we see lie only in the less-dense outer region of the galaxy that is not obscured. Along a similar vein, type 1 objects have broad permitted lines and blue continuum (this power law spectrum is believed to be the result of synchrotron radiation in the accretion disk) because we are able to view the central region without obscuration\citep{hassen}.     

While we have identified many samples of type II Seyfert galaxies and more recently type II quasars at lower redshifts, a large sample of type II quasars at high redshifts, has yet to be identified.  Searches of previous Sloan Digital Sky Survey (SDSS) data releases have revealed a large sample of potential type II quasars at redshifts 0.3 $< z <$ 0.83 \citep[see, for example][]{2003AJ....126.2125Z} and Narrow Line Seyfert 1s (NLS1s) at 1.5 $< z <$ 3.8 \citep[see, for example][]{hassen} suggesting that the SDSS presents the large sample size and defined selection criteria necessary to identify and categorize potentially large numbers of AGN for the study of their bulk properties.  There has been a sample of high redshift radio-loud type II quasars for decades but their radio-quiet counterparts in the optical,  IR and X-ray have been much harder to find.  In general, there is a discrepancy between the kinds of candidates found using optical and X-ray observations.  X-ray observations search up to high redshift in a narrow volume of space and so discover many faint quasars that are difficult to observe for follow-up confirmation at optical wavelengths.  Optical surveys tend to cover a wide volume of space in less detail and so catch only the most luminous objects \citep{2003AJ....126.2125Z}.  Often, objects identified in the X-ray are found to possess broad components when optical spectra are taken and so only about 20$\%$ of X-ray candidates possess the truly narrow emission lines necessary to be designated type II quasars \citep{2003AJ....126.2125Z}.   

We are interested in being able to develop a picture of the evolution and lifetime of AGN in order to better understand the formation and lifetime of SMBHs and their interaction with intra-galaxy star-formation.  One of the important criteria we want to constrain is the fraction of obscured quasars by redshift.  At low redshift it appears that the ratio of type I to type II quasars is close to 1:1 \citep[see][]{2008AJ....136.2373R} but it will take the identification of a much larger sample of type II quasars at high redshift to do the same for high $z$.  In addition, a quantitative understanding of high redshift quasars would allow us to better understand the accretion history of the universe and study the affects of luminosity on AGN structure \citep{2008AJ....136.2373R}.  We know that type I quasars were most common at redshifts 2-3 when SMBHs did most of their growing but in order to fully understand the implications of this result for AGN evolution, we also need comparable data for type II quasars.  Also, it is thought that high redshift type II quasars contribute a substantial fraction of the cosmic hard x-ray background \citep{2003AJ....126.2125Z}.\color{red}this section could use some expanding\color{black}

\subsection{Quasar Targeting with SDSS}

\subsection{AGN emission lines}

Type II quasars are optically-selected based on the presence of narrow permitted emission lines without underlying broad components and high-ionization line ratios that indicate nonstellar ionization.  They are the high-luminosity counterparts of type II Seyfert galaxies.

It is difficult to set a standard definition for the split between narrow and broad emission lines and the chosen value tends to vary with the emission line chosen.  In this paper we adopted a cut of C IV gaussian $\sigma$ width $<$ 1500km/s which translates into C IV FWHM $<$ 3500.  This allows for much broader emission lines than the work of  \cite{2003AJ....126.2125Z} which uses a limit of FWHM(H$\beta$) $<$ 2000km/s or $\sigma$ width $<$ 850 km/s and certainly much broader than the criterion of \cite{2005AJ....129.1783H} which uses a limit of  FWHM(H$\beta$) $<$ 1200 km/s or $\sigma$ width $<$ 510 km/s.  It is the same as the criteria used in \cite{hassen}.

Often, the selection of narrow emission lines alone is not enough to exclude Narrow Line Seyfert 1 (NLS1) galaxies from selection.  NLS1s 

Traditionally, in addition ionization line ratios also help to separate and identify different AGN.  High-ionization line ratios are the result of photoionization by the AGN and were originally used to separate type II quasars from star-forming galaxies for it is observed that AGN have higher flux ratios of [OIII]/H$\beta$.  Yet \cite{2003AJ....126.2125Z} also argued that similar criteria could be used to separate NLS1s from type II quasars. \cite{2003AJ....126.2125Z} used the following relative flux ratios to identify type II quasars:

\begin{equation}
\log({{[OIII] \lambda5008}\over{H\beta}}) > {{0.61}\over{\log({{[NII]}\over{H\alpha}})-0.47}} + 1.19 
\label{eq:z1}
\end{equation}

\begin{equation}
\log({{[OIII] \lambda5008}\over{H\beta}}) > {{0.72}\over{\log({{[SII]}\over{H\alpha}})-0.32}} + 1.30 
\label{eq:z2}
\end{equation}

and at $z > 0.4$

\begin{equation}
\log({{[OIII] \lambda5008}\over{H\beta}}) >  0.3 
\label{eq:z2}
\end{equation}

In addition, most works \citep[for example see][]{2003AJ....126.2125Z} require a sign of AGN activity: either the detection of [Ne v] a high ionization line indicative of AGN activity or a FWHM([OIII]) $>$ 400 km/s which is not present in low-metallicity emission line systems.  Unfortunately the relevant emission lines for these diagnostics do not appear in our high-redshift search so we were constrained to rely solely on the width of the CIV and Ly$\alpha$ emission lines for  the initial phases of this study.  

\subsection{Overview}

In this paper we present 471high redshift (1.55 $< z <$ 4.22) type II quasar candidates optically selected from the new SDSS-III BOSS survey (DR9). \S 2 describes the optical selection and bulk properties of our sample while \S 3 investigates the properties of the sample in a number of public databases.  Then in \S4 and \S5 we describe the investigation of our sample in the Near-Infrared using the Apache Point Telescope (\S4) and the Magellan Telescope (\S5).  Finally, in \S6 we offer some overall conclusions and set out a course for future work.  We assume a flat $\Lambda$CDM Cosmology with $\Omega_m$ = 0.26, $\Omega_{\Lambda}$ = 0.74 and $h$ = 0.71 (\cite{2007ApJS..170..377S}) throughout this paper.         

\pagebreak

\section{Obscured Quasar Identification in the Sloan Digital Sky Survey Baryon Oscillation Spectroscopic Survey}

\subsection{Sloan Digital Sky Survey Baryon Oscillation Spectroscopic Survey}
The Sloan Digital Sky Survey (SDSS) has been in routine operation since 2000 and the Baryon Oscillation Spectroscopic Survey (BOSS) forms a part of SDSS-III (2008-2014) \citep{2011arXiv1101.1529E}.  SDSS uses a dedicated wide field of view, 2.5-meter telescope at Apache Point Observatory in New Mexico with a large mosaic CCD camera and two double spectrographs each fed by 500 optical fibers  with 2$''$ optical diameter for its operation since the BOSS upgrade.  The spectral resolution of the new BOSS spectrographs varies from $\lambda/\Delta\lambda \sim 1300$ at 3600$\AA$ to $\lambda/\Delta\lambda \sim 3000$ at 10,000$\AA$  \citep{2011arXiv1101.1529E}. SDSS-III includes four surveys and has three scientific themes: dark energy and cosmological parameters, the history and structure of the Milky Way, and the population of giant planets around other stars  \citep{2011arXiv1101.1529E}.  BOSS is devoted to the first scientific theme and will attempt to better define the expansion history of the universe using the baryon acoustic oscillation (BAO) feature in large-scale structure as a standard ruler for measuring cosmological distances.  To that end, over its lifetime BOSS will measure the redshifts of 1.5 million galaxies and the Ly$\alpha$ forest spectra of 150 000 quasars in order to make extremely accurate measurements (percent-level precision) of the distance-redshift relation $d_A(z)$ and the Hubble parameter $H(z)$.  BOSS will cover 10,000 deg$^2$, and as of the beginning of this research BOSS had already observed 240,000 galaxy spectra and 29,000 high-redshift (z $\geq$ 2.2) quasar spectra \citep{2011arXiv1101.1529E}.  
  
More precise measurements of the distance-redshift relation and Hubble parameter at large distances will aid in our understanding of cosmic acceleration and dark energy by allowing further constraints on the dark energy equation-of-state parameter $w$ and its derivative with respect to redshift $w_a$ as well as better constraints on the Hubble constant $H_0$ and the curvature parameter $\Omega_k \equiv 1 - \Omega_m - \Omega_{\Lambda} - \Omega_{rad}$ (where  $\Omega_k$ is the total density parameter if k represents the curvature of the universe,  $\Omega_m$ is the density of matter, both baryonic and dark,  $\Omega_{\Lambda}$ is the density of dark energy and $\Omega_{rad}$ is the density of radiation).  BOSS will make these measurements using BAO because the transverse BAO scale constrains the angular diameter distance and the line-of-sight BAO constrains the Hubble parameter.  BAO measurements from SDSS-I/II were made for a redshift of $z \approx 0.275$.  BOSS will be the first survey to attempt to detect the BAO feature in the Ly$\alpha$ forest and should be able to measure $d_A(z)$ to within 4.5$\%$ and $H(z)$ to within 2.6$\%$ assuming 15 quasars per degree$^2$ over 10, 000 degree$^2$ at a redshift of $z \approx$ 2.5 \citep{2011arXiv1101.1529E}.    

Of interest to us is the 1.5 $\times$ 10$^5$ high-redshift quasar spectra in the process of being gathered for BOSS out of 4 $\times$ 10$^5$ targets.  So many targets are necessary because the target selection algorithm is far from perfect and so the sample tends to suffer from stellar contamination.  Targets were selected over a redshift range of 2.2 $\leq z \leq 4$ with 40 targets per degree$^2$ in order to result in the 15 quasar detections per degree$^2$ needed in order to use Ly$\alpha$ forest absorption within the range of BOSS spectrographs to observe the desired high-redshift structure.  While BAO science does not require a homogeneous sample, other quasar science is improved by the existence of such a sample so 20 of the 40 targets per degree$^2$ were chosen using a "core" selection method \citep{2011arXiv1101.1529E}.  While initially useful for BAO measurements, we will examine the quasar spectra in an attempt to identify type II quasars.

\subsection{Obscured Quasar Selection Criteria}

Objects from the entire BOSS survey to date were included in our initial search.  BOSS data allows us to examine candidates at redshift $z > 1.5$.  At this redshift the [OIII] emission line (commonly used as a diagnostic for identifying obscured quasars) no longer falls within the range of wavelengths covered by BOSS so we chose to use the the Ly$\alpha$ (1216 $\AA$) line and the C IV (1549 $\AA$) line as a potential diagnostics to distinguish objects with narrow lines.  

All collected spectra are run through the SDSS BOSS pipeline (v45) which produces best fits for the object's redshift and classifies the object as a galaxy, star or quasi-stellar object accordingly.  This data is then stored along with the spectra in several FITS files produced by the Princeton-1D code written by Dr. David Schlegel \citep[see][]{2003AAS...20314503S}.  Of these files, SPZline and SPZbest FITS files were accessed throughout the course of this research.  SPZbest FITS files contain $\chi^2$ best fits for a variety of data including redshift.  These fits are accomplished by the SDSS spectral classification algorithm which fits a template to each spectrum \citep{2003AAS...20314503S}.  SPZline FITS files contain information on a selection of emission lines (such as Ly$\alpha$ and C IV) including line width and line flux and their respective measurement uncertainty.  It should be noted that while Ly$\alpha$ is modelled individually (owing to the difficulties of fitting the width of this emission line caused by the Ly$\alpha$ forest), all other diagnostic lines are fit to a single gaussian by the pipeline. 

First, the IDL package $readspec$ developed by Dr. Schlegal was used to extract  Ly$\alpha$ and CIV $\sigma$ line width information from SPZline FITS files.  The resulting distribution of CIV $\sigma$ line width can be found in figure \ref{fig:linewidth_hist_all}.  If objects are at a redshift high enough for the Ly$\alpha$ and C IV emission lines to be visible then they are AGN and so our code identified all possible AGN (and any objects incorrectly labelled as AGN by an incorrect redshift determination).  Redshifts below $z$ = 1.5 do not include the CIV emission line in the observable range of the SDSS  and so are not part of our sample.  A redshift of $z$ = 6 represents the farthest observable quasar and so represents the upper cut-off of redshift (see figure \ref{fig:redshift_hist_all} for the redshift distribution of all objects identified as quasars in the BOSS survey DR9). 

\begin{figure}
\begin{center}
\includegraphics[width=\columnwidth]{linewidth_hist_all_test}
\end{center}
\caption{The distribution of CIV $\lambda$1549 $\sigma$ width for all AGN in the SDSS BOSS survey excluding objects with a zwarning value that calls into question the accuracy of their redshift determination.  Many objects with small CIV $\sigma$ width were eliminated in this cut making it likely the pipeline attempted to label noise as a very narrow CIV line but the mistake was caught.  Some of the eliminated objects may be type II quasars with the wrong redshift.  Follow-up visual inspection of these objects might reveal extremely narrow type II quasar candidates.}
\label{fig:linewidth_hist_all}
\end{figure}

\begin{figure}
\begin{center}
\includegraphics[width=\columnwidth]{redshift_hist_all_test}
\end{center}
\caption{The distribution of redshift for all AGN in SDSS BOSS survey excluding objects with a $zwarning$ value that calls into question the accuracy of their redshift determination.  Many of the objects thought to be at a high redshift were removed at this cut making it likely the pipeline did not correctly determine their redshift.  Some of the eliminated objects may still be type II quasars with the wrong redshift.}
\label{fig:redshift_hist_all}
\end{figure}

Our first criteria was to eliminate objects with any $zwarning$ flag set.  The $zwarning$ flag is set by the BOSS pipeline and marks spectra with bad redshift fits.  Errors include ''${\chi}^2$ fit is too close to ${\chi}^2$ for the next best fit"  meaning the pipeline couldn't differentiate between its chosen model fit and the next best fit so there is a high probability the fit is incorrect.  Another common $zwarning$ is "Negative emission in a QSO line" which means an object classified as an AGN has an absorption line which is unphysical and so the pipeline fit must be incorrect.  A $zwarning$ flag calls into doubt the calculated redshift but the object in question could still be a type II quasar.  As a future task these objects might be explored further.  

The next criteria was to select only objects with narrow emission lines.  We required both CIV and Ly$\alpha$ to have a $\sigma$ width $<$1500km/s and a flux detection above 5$\sigma$ to select only strongly detected emission lines.  Finally, to observe the most likely candidates first we combined information on the candidate objects line width and flux.  In search for objects with narrow line width but strong emissions, we first looked at objects with the greatest value of $flux - line width$.  We did this ordering based on the flux and $\sigma$ width of CIV.  This was deemed the best approach to avoid looking at objects where noise was mistaken for a very narrow Ly$\alpha$ or CIV line because with a large flux the emission would be very narrow but still strong.

The final inspection was accomplished visually.  A strong type two quasar candidate possessed narrow but strong emission lines and an extremely weak, flat continuum.  Objects observed visually were classified as either obscured quasars, narrow line seyfert 1s (NLS1s), broad absorption line (BAL) objects or occasionally junk spectra.  NLS1s resemble obscured quasars in that optical Balmer lines expected to be broad in Seyfert objects also appear narrow in NLS1s.  Interestingly however, they are not obscured objects as they still exhibit the strong blue continuum of unobscured objects.  An example of an identified NLS1 is shown in figure \color{red}figure of NLS1\color{black}.  Unlike a type II quasar the continuum level is high and blue (sloping down to the right) and the lines appear relatively broad.  BAL objects have broad absorption troughs blueward of emission lines caused when the continuum emission is absorbed by gas in the region surrounding the central engine.  If this gas is close to the central engine this absorption trough can truncate the emission line blueward of the observed wavelength causing the emission line to appear narrow even when it isn't and so causing these objects to be accidentally caught in our pipeline.  An example of an identified BAL can be seen in figure \color{red}figure of BAL\color{black}. 

In total, \color{red} number of objects inspected \color{black} were visually inspected.  The sample that was observed can be shown in Figure \ref{fig:civfluxvwidth}. The full number were not inspected because the ordering schema allowed us to identify and inspect the most promising candidates first.  The list thus reached a point after which no new candidate type II quasars were being identified and so visual inspection could then cease.  Visual inspection identified two categories of type II quasar candidates: objects that possessed all of the qualities of an obscured quasar (narrow emission lines, no absorption, low continuum) and objects that possessed one or more of the qualities of an obscured quasar and one or more qualities that might identify the object as an NLS1 or a BAL (a component of a broad base, an absorption feature or a blueward sloping, high continuum). \color{red}Effort to quantify this?\color{black}  The full sample of strong type II quasar candidates as well as potential type II quasar candidates can be found in Appendix A.

\begin{figure}
\begin{center}
\includegraphics[width=\columnwidth]{civfluxvwidth_final_test}
\end{center}
\caption{The distribution of CIV $\lambda$1549 FWHM compared to CIV $\lambda$1549 flux for all objects in SDSS BOSS survey with CIV $\sigma$ width $<$ 1500 km/s and Ly$\alpha$ $\sigma$ width $<$ 1500km/s.  All objects in this sample represent potential type II quasar candidates.  This distribution was used to determine the order in which objects were visual inspected by looking at objects with the highest value of $line area - line width$ first.  Objects represented by red diamonds were visually inspected for classification and were determined to be likely type II quasars.  Objects represented by blue triangles were visually inspected for classification and were determined to be NLS1s.  Objects represented by green squares were visually inspected for classification and were determined to be BALs. It is interesting to note that, of the objects inspected, almost no type II quasar candidates were identified above a FWHM of 2000km/s suggesting we could have made our line width cut-off smaller and missed very few candidates.}
\label{fig:civfluxvwidth}
\end{figure}

Strong type II candidates were selected based on a combination of extremely narrow emission lines and low continuum levels.  Figure \color{red}add figure \color{black} represents an extremely strong candidate type II quasar.  The emission lines are extremely narrow while the continuum is extremely low and does not appear blue.  Other candidates, while still strong contenders, presented some unexpected features.  For an example refer to figure \color{red}add figure\color{black}.  While the emission lines in this image are extremely narrow, the candidate exhibits a much broader base which was not identified by the pipeline due to the absorption feature to the left of the emission line.  The absorption feature is probably the result of gas in the surrounding galaxy.  A broad base such as this does not necessarily mean the object is not a type II quasar.  Another possible explanation is that the broad component is light from the central engine that has scattered off dust clouds near the quasar and been reflected along the line of sight.  One way to confirm this hypothesis would be to examine polarimetry of sources with a broad component for scattering off of a nearby cloud should polarize the light (see \cite{2005AJ....129.1212Z}).  %Other candidates presented more of a mystery.  For examples, see figures \color{red}include some examples\color{black}.  Both of these spectra have the extremely narrow emission lines we would expect of a type II quasar with a broadline region obscured by gas.  Yet, both exhibit strong, blue continua that makes it extremely unlikely they are obscured objects. 

\subsection{Results}

In the end, 471 candidates were identified that we believe to be strong or possible type II quasars.  A list of these objects can be found in \color{red}table 1 \color{black} for the 150 strong candidates identified and in \color{red}table 2 \color{black} for the 321 less certain candidates that display only certain features of type II quasars.  The median FWHM of our candidates was 572.4 km/s (see figure \ref{fig:linewidthq_hist}).  The mean redshift of our sample is 2.70, with a redshift range from 2.03 to 4.24 (see figure \ref{fig:redshiftq_hist}).  

\begin{figure}
\begin{center}
\includegraphics[width=\columnwidth]{linewidth_hist_q}
\end{center}
\caption{Distribution of FWHM for the type 2 quasar candidates identified.  {2003AJ....126.2125Z} use a FWHM of H$\beta$ $<$2000 km/s to identify type II quasars.  The majority of our objects do meet this criteria.}
\label{fig:linewidthq_hist}
\end{figure}

\begin{figure}
\begin{center}
\includegraphics[width=\columnwidth]{redshiftq_hist}
\end{center}
\caption{Distribution of redshift for the type 2 quasar candidates identified.  The mean redshift of our sample is 2.70 with a range of redshifts from $2.03 < z < 4.24$}
\label{fig:redshiftq_hist}
\end{figure}

Several efforts were made to try and distinguish our selected sample of type II quasars from NLS1s in order to identify quantifiable diagnostics for use at this redshift range.  Figure \ref{fig:continuum_luminosity} shows the luminosity of our candidates compared to the luminosity of those objects we identified as NLS1s.  Continuum flux at 1450 $\AA$ngstroms were gathered by converting the rest wavelength to emitted wavelength using the equation  

\begin{equation}
\lambda_{emitted} = \lambda_{rest} \times (1+z)
\label{eq:lambda}
\end{equation}

and then by averaging the flux at the 5 pixels around the calculated wavelength.  To convert from flux to luminosity the formula below was used (see \cite{1999astro.ph..5116H}).

\begin{equation}
L = 4\pi D_L^2 S
\end{equation}

Where $D_L$ is the luminosity distance.  The luminosity distance was calculated by numerical integration using sm code written by Dr. Michael Strauss and then translated into IDL.  As we would expect for obscured objects, our type II quasars had a lower optical luminosity at a continuum wavelength than did the NLS1s.  One way to distinguish NLS1s from type II quasars is to observe objects in the mid-Infrared (mid-IR) where dust surrounding an obscured object would radiate.  Follow-up observations of candidates in the mid-IR would confirm identification as obscured objects.  Our type II quasars will exhibit a much higher ratio of $L_{IR}/L_{optical}$ than unobscured objects such as NLS1s (see \cite{2004AJ....128.1002Z}).

\begin{figure}
\begin{center}
\includegraphics[width=\columnwidth]{continuum_luminosity_test}
\end{center}
\caption{Graph of continuum (rest $\lambda$ = 1450) luminosity versus CIV ($\lambda$1549) $\sigma$ line width.   Because type II quasars have their central engines obscured by a dusty torus we would expect them to have smaller continuum luminosities than NLS1s.  Sources represented by red diamonds were visually inspected for classification and were determined to be likely type II quasars.  Sources represented by blue triangles were visually inspected for classification and were determined not to be type II quasars.}
\label{fig:continuum_luminosity}
\end{figure}

The equivalent width of an emission line is defined to be the ratio of the emission line area to the continuum level.  We would expect type II quasars to have a higher equivalent width than NLS1s because obscured objects have a lower continuum flux.  Figure \ref{fig:ew_distribution} is the CIV equivalent width distribution of our sample of NLS1s and our sample of potential type II quasars and figure \ref{fig:civewvw} shows the distribution of CIV equivalent width versus CIV gaussian $\sigma$-width for our sample of potential type II quasars and NLS1s.  However, while the CIV equivalent widths of our sample of potential type II quasars do appear to be higher, no clear bimodality between the two samples presents itself.  Indeed, the bimodality of CIV gaussian $\sigma$-width appears much more apparent suggesting again that we could have used a stronger cut in CIV emission line $\sigma$-width. 

\begin{figure}
\begin{center}
\includegraphics[width=\columnwidth]{ew_distribution_test}
\end{center}
\caption{Graph of CIV equivalent width distribution for all objects visually inspected. The equivalent width of an emission line is the ratio of its area to the continuum level.  We would expect type II quasars to have greater values of equivalent width than NLS1s as obscured objects have lower continuum levels.  The solid line is the distribution of objects classified as NLS1s and the dashed line is the distribution of objects classified as potential type II quasars.}
\label{fig:ew_distribution}
\end{figure}

\begin{figure}
\begin{center}
\includegraphics[width=\columnwidth]{civewvw_test}
\end{center}
\caption{Graph of CIV equivalent width ($\AA$ngstroms) to CIV ($\lambda$1549) $\sigma$ line width for all objects visually inspected. The equivalent width of an emission line is the ratio of its area to the continuum level.  We would expect type II quasars to have greater values of equivalent width than NLS1s as obscured objects have lower continuum levels.   Red diamonds represent sources that were visually inspected for classification and were determined to be likely type II quasars.  Blue triangles represent sources that were visually inspected for classification and were determined not to be type II quasars}
\label{fig:civewvw}
\end{figure}

\subsection{Composite Spectra}

This section presents composite spectra of our candidate type II quasars.  Creating composite spectra is a useful technique because it generates spectra with higher signal to noise than a single spectra which enables us to examine weaker emission lines, the continuum (see \cite{hassen}) and weak broadening features.  For this paper spectra were created using the $coadd$ macro for supermongo (sm) written by Dr. Michael Strauss.  The $coadd$ macro first uses spline interpolation to convert all spectra to rest wavelengths and then performs the coaddition by simply averaging the spectra.  The results for a variety of subsections of our data are displayed in figures \ref{fig:coadd_best20} and \ref{fig:coadd_quasars}.  Figure \ref{fig:coadd_best20} was made using the spectra of the 21 candidates that were visually the strongest, showing no signs of a broad base or absorption feature in either CIV or Ly$\alpha$.  Then, figure \ref{fig:coadd_quasars} was made by coadding all the spectra of strong candidate type II quasars contained in table \ref{tab:dataquasars}.  Interestingly, while the CIV line appears very narrow in this coadded spectra, the Ly$\alpha$ emission line does appear to have a bit of a broad base.  The coadded spectra tend to be very noisy at the extreme ends because there are fewer spectra covering this range.      

\begin{figure}
\begin{center}
\plotone{coadd_best20_test}
\end{center}
\caption{Average composite of the 21strongest optically-selected type II candidates determined by visual inspection.}
\label{fig:coadd_best20}
\end{figure}

\begin{figure}
\begin{center}
\plotone{coadd_quasars}
\end{center}
\caption{Average composite of strong type II quasar candidates.  Interestingly, while the CIV line appears very narrow in this coadded spectra, the Ly$\alpha$ emission line does appear to have a bit of a broad base.}
\label{fig:coadd_quasars}
\end{figure}

\subsection{Conclusions}

\pagebreak

\section{Matching Sample to Existing Databases}

\subsection{Faint Images of the Radio Sky at 20cm Survey}

The Faint Images of the Radio Sky at Twenty-cm(FIRST) survey operated using the National Radio Astronomy Observatory (NRAO) Very Large Array (VLA)  in its B-configuration in Socorro, New Mexico.  FIRST data was collected from 1993-2004 with some additional data taken in 2009.  The North and South Galactic Cap were both covered and so FIRST covers a similar area to SDSS.  Using an automated mapping pipeline, images are produced covering 1.8''/pixel with an rms of 0.15 mJy and a resolution of 5''.  Measurements of peak and integrated flux are also produced by fitting a two-dimensional Gaussian to each source \citep{hassen}.  Approximately 3$\%$ of sources so far have counterparts in SDSS.

While a population of high redshift radio-quiet type II quasars has only recently been discovered, a population of high redshift radio-loud type II quasars has existed in the literature for decades \citep[see, for example][]{2004AJ....128.1002Z}.  We have found 13 matches within 3'' of our type II quasar candidates within the FIRST catalog meaning only 2.8\% of our sources have a radio component.  Of the strong potential candidates only 3 of 150 or 2\% have a radio match with FIRST though all 150 objects were covered in the sample. %It is well document that between 8$\%$ and 20$\%$ of all type I AGN are radio-loud \citep{2004AJ....128.1002Z} so it is relatively unsurprising to see approximately the same distribution for type II AGN.  Several of the identified radio-loud quasars are distinctly doubled meaning we are seeing lobes from relativistic jets.  Interestingly, several of the identified radio sources were not deemed to be strong type II candidates but were instead identified as potential candidates exhibiting only some of the features of a type II quasar.

\subsection{Wide-Field Infrared Survey Explorer}

\subsection{Conclusions}

\pagebreak

\section{Near Infrared Follow-up with the Apache Point Observatory}

To understand better the properties of these objects it is necessary to probe additional parts of their spectrum.  In particular, it is extremely beneficial to examine the region containing H$\alpha$, H$\beta$ and [OIII] because these lines are typically used to determine if quasars can truly be considered obscured objects and [OIII] acts as a proxy that allows us to measure an object's bolometric luminosity.  In particular, it has been observed that AGN have a bimodal H$\alpha$ line width distribution with a cut occurring around FWHM(H$\alpha$) = 1200 km $s^{-1}$ \citep{2003AJ....126.2125Z} such that we can characterize objects with a larger FWHM as unobscured AGN and objects with a smaller FWHM as obscured AGN.  In addition, the width of [OIII] can be used to distinguish AGN from star-forming galaxies.  \citet{2005AJ....129.1783H} found that the majority of AGN have FWHM([OIII]) $>$ 400 km s$^{-1}$ while the majority of star-forming galaxies have FWHM([OIII]) $<$ 400 km s$^{-1}$.  At a redshift of $z > 1.5$ these three lines fall in the observed near infrared (NIR).  We observed \color{red}number of objects\color{black} of the candidate objects in the NIR using the Triplespec instrument on the Apache Point 3.5m telescope to obtain spectra in the j,h, and k band corresponding roughly to 0.95-2.46 $\mu$m.  Successful observations took place on the nights of August 23, November 15, December 26 and March 14.

\subsection{TripleSpec Instrument}
Triplespec is a NIR echelle spectrograph.  \color{red} details on triplespec \color{black}.  

\subsection{Observation Selection Criteria} 

Obscured quasar candidates were selected for observation based on a number of increasingly specific criteria.  The initial list of potential objects was taken to be the list of strong quasar candidates, those possessing all of the expected features of an obscured quasar.  Then objects were selected that would be observable on the particular evening: lunar midnight was taken from the Apache Point website and objects two hours in right ascension before twilight until two hours in right ascension after sunrise were selected.  Then any objects that had been previously observed were eliminated in addition to any objects with a match in the FIRST Radio catalog.  Objects in FIRST are undoubtably quasars but our interest was in radio-quiet objects excluding anything with a match in FIRST.  Next, redshift cuts were made to ensure that the emission lines of interest fell in to the observable range of the detector.  Objects were thus selected with a redshift of $2.1 < z < 2.6$ or a redshift of $3.0 < z < 3.4$.  Finally, the z-band magnitude of all objects was recorded.  The z-band is the closest optical band to the NIR and while the slit-detector for Triplespec operates in the k-band, the z-band magnitude acted as a rough proxy for the NIR magnitude of the source expected in the k-band.  It was found that obscured quasar candidates with a z-band magnitude above 21 were extremely difficult to identify in the slit-detector.  Finally, if choice remained priority was given to objects with the best observed optical properties or to objects with matches in various X-ray surveys (COSMOS, XMM, Chandra).

\subsection{Observations}

Our program to observe potentially obscured quasars in the near-infrared was awarded a total of 6 nights on the Apache Point Observatory 3.5m telescope. Of these nights data was obtained on August 23, November 15, December 26 and March 14.  The remaining evenings were unsuccessful due to cloudy, windy or rainy conditions that prevented us from observing.  All observations were taken with Triplespec, an instrument built for NIR spectroscopy with the Apache Point 3.5m telescope. All observing was conducted remotely from Princeton University using the software package TUI (Telescope User Interface).  

Observations were completed using the 1.5'' slit (except for December 26 when the 1.1'' slit was used) and a Fowler Sampling of N = 8. \color{red}  describe Fowler Sampling \color{black}. At some point during the evening flat field images with a bright quartz lamp were taken for background subtraction.  In addition, Neon Argon arcs and open shutter darks were taken as an extra precaution though none of these images were used during reduction.  For each object a standard A0V star was observed for telluric calibration before and after the target was observed.  The spectrum of standard A0V stars are well known so fitting a model to a standard star allows us to subtract $H_2O$ absorption in the atmosphere from our object of interest. Targets of interest were sampled in ABBA mode where A and B represent offset positions on the detector slit for sky subtraction.  Individual science exposures lasted four minutes and approximately two hours was spent on-source for each target.

\subsection{Data Reduction}

Data reduction was accomplished using Xspextool from the Apache Point Observatory. 

\subsection{Results}

\subsection{Conclusions}

\pagebreak

\section{Near Infrared Spectroscopic Follow-up with the Magellan Telescope}

\subsection{FIRE Instrument}

\subsection{Observation Selection Criteria}

\subsection{Observations}

\subsection{Data Reduction}

\subsection{Results}

\subsection{Conclusions}

\pagebreak

\section{Conclusion}

\subsection{Future Work}

{\it Acknowledgements:}\acknowledgements The author would like to thank absolutely everyone involved in making this work possible.

{\it Facilities:} \facility{Sloan Digital Sky Survey}

\pagebreak

\appendix

\section{Tables of Candidates Type Two Quasars}

\section{Sample Optical Spectra from BOSS}

\section{Near-Infrared Spectra from Apache Point}

\section{Near-Infrared Spectra from Magellan}

\section{FIRST Data}

\section{WISE Data}

\pagebreak

\bibliography{quasarbib}

\end{document}