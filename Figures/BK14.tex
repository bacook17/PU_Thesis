\begin{figure*}[hbt]
\plottwobig{Copied_Figs/BK14_FsvM}{Copied_Figs/BK14_FsvR}
\caption{The stellar fraction from stacked optical and weak-lensing
  observations, as presented in \citet{Bahcall2014}. \textit{Left}:
  The stellar fraction within $\radius{}_{200b} \approx \rvir{}$ in
  groups and clusters, as a function of total mass. f$_*$ decreases in
  more massive clusters, in agreement with observations collected by
  \citet{Giodini2009}. \textit{Right}: The stellar fraction as a
  function of halo-centric radius. The profiles are binned by
  richness, a proxy for mass. More massive clusters have lower stellar
  fraction at any given radius (lower-mass halos are more dominated by
  their BCGs), but the stellar fraction tends towards a constant value
  at high radius irrespective of mass, the ``cosmic stellar fraction''
  $\approx 1\%$.}
\label{fig:BK14}
\end{figure*}                
                                  
%\afterpage{\clearpage}
