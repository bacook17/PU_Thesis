\begin{figure*}[hbtp]
\plottwobig{Copied_Figs/BK14_FsvM}{Copied_Figs/BK14_FsvR}
\plotoneman{Copied_Figs/BK14_FsvR_local}{.49}
\caption{The stellar fraction from stacked optical and weak lensing
  observations, as presented in \citet{Bahcall2014}. \textit{Upper
    left}: The cumulative stellar fraction within $\radius{}_{200b}
  \approx \rvir{}$ in groups and clusters, as a function of total
  mass. f$_*$ decreases in more massive clusters, in agreement with
  observations collected by \citet{Giodini2009}. \textit{Upper right}:
  The cumulative stellar fraction as a function of halo-centric
  radius. The profiles are binned by richness, a proxy for mass. More
  massive clusters have lower stellar fraction at any given radius
  (lower mass halos are more dominated by their BCGs), but the stellar
  fraction tends towards a constant value at high radius irrespective
  of mass: the ``cosmic stellar fraction'' $\approx
  1\%$. \textit{Lower}: The local stellar fraction of all groups and
  clusters, regardless of richness, is approximately $1\%$ at all
  scales above the BCG.}
\label{fig:BK14}
\end{figure*}                
                                  
%\afterpage{\clearpage}
