\begin{figure*}[bt]
\plotonebig{Figs_Thesis/FgvR}
\caption{The cumulative hot gas fraction for each cluster sample in
  our study, plotted against the cluster-centric radius. Blue and cyan
  points (G09) represent data from \citet{Giodini2009}, red points
  (P1) are temperature hypothesis 1 from \citet{PlanckIntVb}, green
  points (E13) from \citet{Eckert2013a}, and yellow points (U09) from
  \citet{Umetsu2009}. See Section \ref{sec:Gas.Observations} for
  details of these sources. Many observations are extrapolated to
  $1.2\rvir{}$, as in \citet{Rasheed2011}. The f$_{gas}$ line
  represents the difference between the WMAP9 cosmic baryon fraction
  and the cosmic stellar fraction from \citet{Bahcall2014}. We discuss
  possible biases in using the HSE mass and extrapolating the gas
  density profile in Section \ref{sec:Limitations}. }
\label{fig:FgvR}
\end{figure*}    

%\afterpage{\clearpage}
