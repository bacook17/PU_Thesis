%Senior Thesis Chapter 4
%Ben Cook '14 (bacook@)
%Adviser: Neta Bahcall
\chapter{Discussion}
\label{chap:Discussion}

\section{Limitations and Observational Biases}
\label{sec:Limitations}

\subsection{Assumption of Hydrostatic Equilibrium}
\label{sec:Limitations.HSE}
As was highlighted throughout Chapter \ref{chap:Data}, there are
several different ways to calculate the total mass of group and
cluster halos. One key characteristic of a mass estimation method is
whether it relies on the assumption of hydrostatic equilibrium. The
so-called ``hydrostatic mass'' ($\Mass{}_{HSE}$) can be calculated by combining the gas
density profile with either the temperature or pressure profile, a
method ubiquitous in X-ray observations. Using mass-scaling relations
(such as the Y$_\textrm{SZ}$-\Mfive{} relation) may also be sensitive
to the assumption of HSE if the relation is calibrated against
hydrostatic masses. Masses which do not rely on the assumption of HSE
are primarily derived from gravitational lensing.

Hydrostatic masses could be systematically biased, relative to the
``true'' total mass (usually assumed to be the lensing mass, $\Mass{}_{WL}$) if there
are significant sources of non-thermal pressure in the ICM. These
could include kinetic bulk motions or magnetic fields. Hydrostatic
equilibrium assumes that the gravitational force (the total mass) is
offset by the pressure gradient, so assuming that only the gas
pressure contributes can lead to an incorrect calculation of the
mass. 

The magnitude of the hydrostatic mass bias is of paramount importance
to precision cosmology. Cosmological parameters (such as \omegam{})
derived from \Planck{} cluster counts and hydrostatic mass estimates
disagree significantly from values derived directly from the CMB power
spectrum, but a large hydrostatic mass bias ($ b= 1-
\Mass{}_{HSE}/\Mass{}_{WL} \approx 0.3$) could relieve the
observational tension \citep{Gruen2013,VonderLinden2014}. Such a large
bias on the total halo mass would also dramatically affect the gas
fraction derived using hydrostatic masses, limiting its use as a
cosmological probe \citeeg{Grego2001,Ettori2009b}. As the majority of
our \fg{} measurements in clusters were measured relative to
hydrostatic masses (with the exception of \citet{Umetsu2009}), our
results are likewise sensitive to the hydrostatic bias.

Cosmological simulations are a major tool used to constrain this bias.
True mass calculations can be compared to mock X-ray observations
which assume hydrostatic equilibrium, determining the bias factor as a
function of mass and overdensity. Simulations nearly unanimously
indicate that the hydrostatic mass is biased low compared to the true
halo mass ($b>0$) and is more significant towards the outskirts of
clusters or in unrelaxed clusters, where merger disruptions and bulk flows
become more significant. However, different simulations and physical
prescriptions place the bias anywhere from $5\%$
\citeeg{Lau2009,Meneghetti2010,Burns2010,Nelson2012} to $20\%$
\citeeg{Arnaud2007,Nagai2007,Battaglia2013}.

Observational constraints on the hydrostatic mass bias vary
widely. Some weak-lensing measurements of clusters suggest that
hydrostatic X-ray or SZ masses are biased low by $10\%$
\citep{Andersson2011,High2012}, while others indicate this bias is as
large as $20-30\%$
\citep{Arnaud2007,Ichikawa2013,VonderLinden2014}. Yet others, however,
find no significant difference between weak-lensing masses and
hydrostatic masses, with some hints that hydrostatic equilibrium
assumptions may even \textit{overestimate} the true mass in lower-mass
clusters \citep{Gruen2013, Israel2014}. Figure \ref{fig:Gruen}, from
\citet{Gruen2013}, shows the agreement in measured weak lensing and
hydrostatic masses. 

\begin{figure*}[hbtp]
\plotonebig{Copied_Figs/Gruen2013}
\caption{A comparison of weak lensing masses ($\Mass{}^{WL}_{500c}$)
  and X-ray hydrostatic masses ($\Mass{}^{X}_{500c}$) for SZ clusters,
  as presented in \citet{Gruen2013}. Error bars indicate the best-fit
  ``single-halo'' fit, while the filled symbols include corrections to
  $\Mass^{WL}$ for other structures in the field of view. In tension
  with simulations, the authors find that hydrostatic masses are not
  systematically biased low relative to the weak lensing mass.}
\label{fig:Gruen}
\end{figure*}    


\afterpage{\clearpage}

The issue of hydrostatic mass bias is far from solved. Due to the
inconsistent observational and simulated constraints, it is unclear
how large of a hydrostatic correction factor should be included in our
measurements of the gas fraction, or if one is even necessary. A bias
low in the hydrostatic mass would bias the gas fraction high, meaning
that groups and clusters are slightly more deficient of baryons at a
given radius. Our reservations to use a hydrostatic correction factor
are slightly justified by the fact that one of our samples (U09)
measures gas fraction against the weak-lensing mass, and this fraction
agrees well with gas fraction derived from the HSE
assumption.

\subsection{Gas Clumping in Cluster Outskirts}
\label{sec:Limitations.Clumping}

The primary means of deriving the gas density profile of clusters is
from measurements of X-ray surface-brightness, which scales with the
square of electron density. Due to this $n^2$ dependence, clumpy
structures in the ICM will emit more than their share of X-ray,
biasing gas density measurements high. The magnitude of this bias
depends on the smoothness of the ICM gas distribution, which can vary
widely from cluster to cluster. 

Simulations typically predict a clumping bias (overestimate of
$\Mass{}_{gas}$) of $\approx10-15\%$ \citep{Nagai2011,Battaglia2013},
which increases in unrelaxed clusters and towards cluster outskirts,
where recent interactions have a more significant dynamical
effect. This could explain the large differences in measured gas
fraction between CC and NCC clusters, although observations suggest
that the level of clumpiness is overestimated in simulations, and that
the average bias is below $10\%$ \citep{Eckert2013c}. The majority of our
cluster samples represent relaxed (CC) clusters, so we do not include
a clumping bias factor at this time. We also note that the P13
\citep{PlanckIntV} and U09 \citep{Umetsu2009} gas fractions are
derived from SZ measurements, which do not suffer from this clumping
bias, and agree well with our X-ray data sets. 

\subsection{Extrapolation of the Density Profile Slope}
\label{sec:Limitations.Slope}
The choice to extrapolate the gas fraction to $1.2\rvir{}$ and then
stop is arbitrary. The radius at which the gas profile steepens
substantially ($\alpha{}_{gas}$ increases) is not well constrained.
 Between \rfive{} (where $\alpha{}_{gas}$ is well
measured) and \rtwo{}, observations disagree whether the slope remains
roughly constant \citep{Dai2010} or steepens by roughly $10\%$
\citep{Ettori2009a}. Assuming the true evolution is somewhere in
between, the gas density slope should not change appreciably relative
to the total mass (NFW) profile, which steepens by about $4\%$ within
this range. Therefore, our assumption that $\alpha{}_{m}$ and
$\alpha_{gas}$ remain constant beyond \rtwo{} should roughly
approximate the increase of \fg{} with radius. 

It is expected that the slope will eventually asymptote to match the
total mass (NFW) profile \citep{Umetsu2009, Battaglia2013}, suggesting
that, at large radius, $\alpha_{gas}$ will steepen more quickly that
$\alpha_{m}$. The point at which this occurs is unknown, but it will
taper the growth of \fg{}, which should eventually reach a constant
value, similar to the stellar fraction.  \citet{PlanckIntV} finds that
the gas fraction in stacked \Planck{} clusters flattens out between
$1$ and $1.5\rvir{}$. We approximate this by extrapolating \fb{} as
constant above $1.2\rvir{}$. However, until temperature measurements
beyond \rfive{} improve, the true radius where the baryon fraction
reaches a maximum will remain unconstrained. 

\section{Comparison to Simulations}
\label{sec:Simulations}

As discussed above, cosmological simulations have been commonly used
to examine the magnitude of observational biases on measurements of
cluster properties. Here we discuss the measured cluster properties
themselves, from the simulations of \textbf{\citet{Battaglia2013}}.

The authors conducted a series of smoothed particle hydrodynamic (SPH)
simulations with three different physical feedback prescriptions: 1) a
``shock-heating'' only method, 2) a method that also included
radiative cooling and star formation/supernovae feedback, and 3) a
prescription including AGN thermal feedback. The simulations produced
(at $z=0$) a sample of over 1000 clusters with $\Mtwo > 7\power{13}$
\Msun{}, and 800 above $10^{14}$ \Msun{}.

\begin{figure*}[hbt]
\plotonebig{Copied_Figs/Battaglia_FbvM}
\caption{The gas, stellar, and baryon fraction (scaled by the cosmic
  baryon fraction) as a function of cluster mass from a series of
  simulations by \citet{Battaglia2013}. The baryon fraction within
  \rtwo{} is close to the cosmic value for the entire mass range of
  $10^{14}\dash10^{15}~\Msun{}$, except for in the AGN feedback model,
  where low-mass clusters have fewer baryons within \rtwo{}. These
  simulations are in agreement with our results, although the
  simulations predict a large hydrostatic mass bias (Section
  \ref{sec:Limitations.HSE}) that has yet to be confirmed observationally.}
\label{fig:Battaglia_FbvM}
\end{figure*}    



Figure \ref{fig:Battaglia_FbvM} shows the cumulative stellar, gas, and
baryon fractions in the simulated clusters, as a function of mass
(compare to our Figure \ref{fig:FgvM}). In the shock heating and
radiative cooling models, the total baryon fraction within
\rtwo{} is nearly the cosmic value in all clusters ranging from
$10^{14}-10^{15}$ \Msun{}. In the AGN feedback model, \fb{}
decreases in smaller halos, due primarily to suppressed stellar mass
production, although partly due to increased non-thermal
pressure pushing the gas towards higher radius. These simulations are
in good agreement with our results that conclude the entire baryon
fraction can be recovered in clusters of all mass ranges. 

\begin{figure*}[hbtp]
\plotonebig{Copied_Figs/Battaglia_FbvR}
\caption{The gas, stellar, and baryon fraction distributions (scaled
  by the cosmic baryon fraction) as a function of radius in the
  simulations of \citet{Battaglia2013}, binned in four parts by
  cluster mass. The stellar fraction and gas profiles match
  observations well, and the total baryon fraction approaches the
  cosmic value at high radius.}
\label{fig:Battaglia_FbvR}
\end{figure*}    



Figure \ref{fig:Battaglia_FbvR} presents the radial distribution of
gas, stellar, and baryon fractions for clusters binned by mass. Only
the results for the AGN feedback model are presented. The stellar
fraction decreases significantly from the center of the clusters, with
a higher stellar fraction in low-mass clusters for any given
radius. In all mass bins, however, the stellar fraction asymptotically
approaches a ``cosmic'' value of $\approx10\%$ of the baryon
fraction. This behavior matches very well the stellar fraction profile
measured by \citet{Bahcall2014}, although the asymptotic cosmic
fraction is higher than observed ($\approx6\%$). 

The gas fraction profile also matches our observed trends well. The
gas fraction in low-mass clusters is significantly lower in the
central cluster regions, but increases more rapidly with radius than
high-mass clusters. The gas fraction in clusters of all mass converges
towards the same profile at high radius. Of particular interest,
however, is that even at $\radius{}>3\rtwo{}$ the gas fraction is
still increasing steadily. If this represents the gas profile in real
clusters (not certain, as the different feedback models have different
behavior in the outskirts), then the our assumption that the gas
fraction remains constant above $1.2\rvir{}$ may need adjustment. 

The overall baryon fraction increases from a minimum of $70-80\%$
cosmic around $0.5\rtwo{}$ to $>90\%$ outside the virial radius. The
fact that the baryon fraction does not reach the cosmic fraction until
very high radius ($>4\rtwo{}$) is in slight tension with our results,
and improved observations of the gas profile slope beyond \rtwo{} will
help constrain whether these simulations accurately predict how far
beyond the virial radius cluster baryons are pushed out. Additionally,
the simulations assume a cosmic baryon fraction of $\fb{} = 0.172$,
higher than current observational estimates. The increased baryon
abundance could lead to enhanced feedback effects, pushing baryons
further from the cluster centers than actually observed. Simulations
which match the current cosmological constraints would help clarify
this issue. 

\begin{figure*}[hbt]
\plotonebig{Copied_Figs/Battaglia_Fgvz}
\caption{The gas fraction (scaled by the cosmic baryon fraction)
  within various radii, as a function of redshift, as measured by the
  simulations of \citet{Battaglia2013}, and binned by mass. There is
  no significant redshift evolution of the gas fraction within any
  given radius, except possibly within $2\rtwo{}$, suggesting that
  measurements of the cluster baryon fraction should be fairly
  redshift independent back to $z\approx1$, where massive clusters
  first begin to form.}
\label{fig:Battaglia_Fgvz}
\end{figure*}    



Finally, in Figure \ref{fig:Battaglia_Fgvz}, we show the redshift
evolution of the gas fraction of the \citet{Battaglia2013} simulated
clusters. Measured from the cluster centers to far beyond the virial
radius, the gas fraction appears constant with redshift at all radii,
with the exception of a slight decrease within $2\rtwo$. Measured from
$z=1.5-0$, this shows remarkable stability within the ICM over many
gigayears, including the phase when the most massive clusters begin to
form. \todo{What does this mean cosmologically? What else should we
  say about the redshift evolution?}

\section{Implications}
\label{sec:Implications}

\subsection{Where are the Baryons?}
\label{sec:Implications.Where}

Our results show that the entire expected baryon content of dark
matter halos can be detected or inferred using current
observations. While the baryonic components (such as stellar mass or
ICM gas) are more extended than the dark matter, they are still bound
to their host halos, and there is no significant problem of baryons
missing from halos. Therefore, the answer to the question, ``Where are
the baryons in dark matter halos?'' is answered simply as: they are
all found in the halos, distributed in various combinations between
the central regions and the diffuse outskirts. 

The size and frequency of virialized halos define the overall
structure of matter in the universe. Because baryons populate in equal
proportions galactic, group, and cluster halos, on large scales they
represent excellent tracers of the overall dark matter distribution
and structure. A relevant physical scale seems to be the halo virial
radius: this scale separates the regimes where baryons do/do not trace
the dark matter. On scales smaller than the virial radius, baryons are
under-abundant, relative to their cosmic fraction, and this abundance
is dependent on the overall mass of the halo. However, averaging on
scales larger than \rvir{}, baryons match the dark matter distribution
of the cosmos at a constant, unvarying fraction. 

This suggests new ways to place constraints on the cosmology of the
universe. Dark matter is notoriously difficult to detect or measure
accurately, with the only direct method being weak lensing. Baryons,
however, emit or absorb radiation in a variety of measurable
ways. The knowledge that baryons trace the underlying dark matter
profile allows the use of easily observed baryon mass distributions to
constrain the abundance and masses of halos \citeeg{Ettori2009b}. 

Furthermore, our results can help place constraints on the expected
contents of baryon reservoirs exterior to halos. Examples of these
interhalo components include the IGM, WHIM, and Ly$\alpha$ forest
outside of galaxies and clusters, as well as the sparsely populated
cosmic voids. If halos were deficient in baryons, then baryons in
these interhalo components would necessarily be overabundant, holding
the additional baryons removed from halos. Because halos retain their
relative baryon fractions, these diffuse components must also, and the
baryonic mass in these components (as a whole) must match the total
dark matter mass outside of halos in the cosmic fraction. \todo{How
  else should we answer where are the baryons?}

\subsection{Deviations from Self-Similarity}
\label{sec:Implications.Invariance}

Dark matter halos (both observationally and in simulations) are found
to fit the NFW density profile, and hence are, roughly,
``self-similar''. This means that all dark matter halos have the same
proportions and shapes, varying only by overall size. Assuming simple
gravitational collapse, physical properties of baryons in clusters
(such as pressure and temperature) are likewise to obey ``universal
profiles'', once scaled to the appropriate halo size
\citeeg{Arnaud2010,PlanckIntV}.

Although we have shown that the entire baryonic mass can be accounted
for in halos ranging over three orders of magnitude in mass, the gas
and stellar fractions reach their expected cosmic levels at
overdensities which appear to be mass-dependent. Pure self-similarity
would suggest that, for instance, the stellar fraction should reach
the cosmic value of 0.01 at the same overdensity (such as $\rtwo$ or
$\radius{}_{100}$) regardless of halo mass. However, observations of
the gas fraction (Section \ref{sec:Spatial}) and the stellar fraction
\citep{Bahcall2014} in clusters indicate that the scale at which these
fractions asymptote to the cosmic values decreases with cluster
mass. Baryons are spread to larger scales, relative to the halo size,
in smaller halos, an observation which is supported by cosmological
simulations \citep{Battaglia2013}.

Therefore, baryonic physics plays an important role in shaping the
distribution of baryons in dark matter halos. Feedback effects (such
as merger shocks, AGN feedback, and star formation) are able to push
baryons further into the outskirts of small halos than in large halos,
due to the decreased gravitational potential. However, as our results
indicate, these feedback effects are not so powerful as to remove a
substantial fraction of baryons from the halos altogether, and nearly
the entire baryon fraction remains bound to dark matter halos.

\subsection{The Contribution of Individual Galaxies to Clusters}
\label{sec:Implications.Individual}

\citet{Bahcall2014} found that the overall mass-to-light ratio of
clusters matches closely the average mass-to-light ratio of individual
galaxies, when normalized by the changing ratio of elliptical and
spiral galaxies with cluster radius (the ``density-morphology
relation''). This is consistent with the picture that
the entire dark matter content of clusters is comprised of the dark
matter originally bound to galaxy halos, which fell into the clusters
and was stripped to form the cluster halo. There is no reason, they
argue, to assume that groups and clusters contain more dark matter (relative to
light) than galaxy halos do \citeeg{Ostriker1974, Guo2010}. 

Our results suggest a similar interpretation involving the entire
baryonic content of clusters. We show that clusters of all sizes
contain the cosmic fraction of baryons within about the virial radius,
distributed in about 15 parts hot ICM gas to 1 part stellar mass. Yet,
combining the stellar mass of galaxies with observations of the
multiphase CGM gas, \citet{Werk2014} demonstrates that large, L$^*$
galaxies also contain approximately the cosmic fraction of baryons
within their halos. If clusters form entirely from the disruption of
infalling galaxies, then the galaxies would bring a baryon abundance
equal to the cosmic fraction with them, filling clusters with dark
matter, stars, and gas from their halos. Clusters do not necessarily
need a separate source of gas to explain the high abundance of baryons
in their halos: the cluster baryons could be comprised entirely of the
baryonic mass once held in the disk and CGM of constituent
galaxies. The ICM, therefore, could be the stripped remains of
galactic ISMs and CGMs, with the stellar disks either remaining as
cluster galaxies or being dispersed as the diffuse intracluster light
(ICL).

One inconsistency that arises in this model is the stellar makeup of
the baryons in structures: around $15\%$ of the baryons in L$^*$
galaxies are held in stars, while the stellar mass of clusters is only
$\approx6\%$ of their total mass. One possible explanation for this
discrepancy is a suppression of star formation in the galaxies that
fall into clusters. Due to the hot, relatively dense ICM, frictional
heating of the ISM of infalling galaxies could prevent these galaxies
from forming stars. The stellar fraction in clusters could be
explained if galaxies which merged into clusters had $\approx 2.5$
times lower stellar mass than presently observed field galaxies,
leaving the remainder of the baryons in a gaseous phase to populate
the ICM. This level of star formation suppression is high, but not
unreasonable. Most large clusters collapsed around redshift $z=1-1.5$
\citep{Eke1996, Battaglia2013} and while the galactic star formation
rate (per year) peaked around $z=2-3$, a large majority of stellar
mass formed during the long period between $z=2-0$
\citep{Hopkins2006}. Even if star formation suppression of this
magnitude did not occur, we can place a lower limit on the total
contribution by galaxies to the baryons (and dark matter) of clusters
using the observed ratio of stellar fraction.

In the conservative case, let us assume that the entire stellar mass
of galaxies formed during the peak of star formation, such that
galaxies are not forming any stars during the period of cluster
growth, and that there is no inherent difference between galaxies
which will fall into clusters and those that will not. Therefore, the
current observed stellar fractions of field galaxies ($15\%$)
represents accurately the stellar fraction of cluster galaxies just
before they merge, and no star formation suppression effects need to
be considered. Assuming no star formation occurs except in galaxies,
then the total stellar fraction in clusters ($6\%$) came entirely from
the previously formed stars of merging galaxies. Galaxies, therefore,
must have contributed $6/15 = 40\%$ of the total mass of clusters,
with the remaining $60\%$ coming from diffuse IGM gas which collapsed
into the cluster along with the galaxies. 

This estimate represents a lower limit, due to the assumption of no
star formation past $z\approx2$. In reality, stellar mass buildup has
continued strongly to $z=0$, such that galaxies which began merging
into clusters at $z\approx1$ likely had a stellar fraction lower than
$15\%$. In this case, star formation in cluster galaxies will surely
be suppressed by falling into clusters, raising our estimate of the
galactic contribution to clusters. 

According to the models of \citet{Hopkins2006}, the accumulated
stellar mass in galaxies at $z=1$ was only $\approx60\%$ its current
value, and only $45\%$ at $z=1.5$. If galaxies had a $50\%$ overall
suppression in star formation after falling into clusters at $z=1.5$,
their current stellar fraction would be $11\%$, suggesting that
individual galaxies contributed $>50\%$ of the matter of clusters,
while suppression of $80\%$ of star formation would raise the galactic
contribution to clusters to $70\%$. \todo{Cite Bahcall,Bahcall,Pop on the stripping
  of gas in galaxies falling into clusters??}

