%Senior Thesis Chapter 4
%Ben Cook '14 (bacook@)
%Adviser: Neta Bahcall
\chapter{Discussion}
\label{chap:Discussion}

\section{Limitations and Observational Biases}
\label{sec:Limitations}

\subsection{Assumption of Hydrostatic Equilibrium}
\label{sec:Limitations.HSE}
As was highlighted throughout Chapter \ref{chap:Data}, there are
several different ways to calculate the total mass of group and
cluster halos. One key characteristic of a mass estimation method is
whether it relies on the assumption of hydrostatic equilibrium. The
so-called ``hydrostatic mass'' ($\Mass{}_{HSE}$) can be calculated by combining the gas
density profile with either the temperature or pressure profile, a
method ubiquitous in X-ray observations. Using mass-scaling relations
(such as the Y$_\textrm{SZ}$-\Mfive{} relation) may also be sensitive
to the assumption of HSE if the relation is calibrated against
hydrostatic masses. Masses which do not rely on the assumption of HSE
are primarily derived from gravitational lensing.

Hydrostatic masses could be systematically biased, relative to the
``true'' total mass (usually assumed to be the lensing mass, $\Mass{}_{WL}$) if there
are significant sources of non-thermal pressure in the ICM. These
could include kinetic bulk motions or magnetic fields. Hydrostatic
equilibrium assumes that the gravitational force (the total mass) is
offset by the pressure gradient, so assuming that only the gas
pressure contributes can lead to an incorrect calculation of the
mass. 

The magnitude of the hydrostatic mass bias is of paramount importance
to precision cosmology. Cosmological parameters (such as \omegam{})
derived from \Planck{} cluster counts and hydrostatic mass estimates
disagree significantly from values derived directly from the CMB power
spectrum, but a large hydrostatic mass bias ($ b= 1-
\Mass{}_{HSE}/\Mass{}_{WL} \approx 0.3$) could relieve the
observational tension \citep{Gruen2013,VonderLinden2014}. Such a large
bias on the total halo mass would also dramatically affect the gas
fraction derived using hydrostatic masses, limiting its use as a
cosmological probe \citeeg{Grego2001,Ettori2009b}. As the majority of
our \fg{} measurements in clusters were measured relative to
hydrostatic masses (with the exception of \citet{Umetsu2009}), our
results are likewise sensitive to the hydrostatic bias.

Cosmological simulations are a major tool used to constrain this bias.
True mass calculations can be compared to mock X-ray observations
which assume hydrostatic equilibrium, determining the bias factor as a
function of mass and overdensity. Simulations nearly unanimously
indicate that the hydrostatic mass is biased low compared to the true
halo mass ($b>0$) and is more significant towards the outskirts of
clusters or in unrelaxed clusters, where merger disruptions and bulk flows
become more significant. However, different simulations and physical
prescriptions place the bias anywhere from $5\%$
\citeeg{Lau2009,Meneghetti2010,Burns2010,Nelson2012} to $20\%$
\citeeg{Arnaud2007,Nagai2007,Battaglia2013}.

Observational constraints on the hydrostatic mass bias vary
widely. Some weak-lensing measurements of clusters suggest that
hydrostatic X-ray or SZ masses are biased low by $10\%$
\citep{Andersson2011,High2012}, while others indicate this bias is as
large as $20-30\%$
\citep{Arnaud2007,Ichikawa2013,VonderLinden2014}. Yet others, however,
find no significant difference between weak-lensing masses and
hydrostatic masses, with some hints that hydrostatic equilibrium
assumptions may even \textit{overestimate} the true mass in lower-mass
clusters \citep{Gruen2013, Israel2014}. Figure \ref{fig:Gruen}, from
\citet{Gruen2013}, shows the agreement in measured weak lensing and
hydrostatic masses. 

\begin{figure*}[hbtp]
\plotonebig{Copied_Figs/Gruen2013}
\caption{A comparison of weak lensing masses ($\Mass{}^{WL}_{500c}$)
  and X-ray hydrostatic masses ($\Mass{}^{X}_{500c}$) for SZ clusters,
  as presented in \citet{Gruen2013}. Error bars indicate the best-fit
  ``single-halo'' fit, while the filled symbols include corrections to
  $\Mass^{WL}$ for other structures in the field of view. In tension
  with simulations, the authors find that hydrostatic masses are not
  systematically biased low relative to the weak lensing mass.}
\label{fig:Gruen}
\end{figure*}    


\afterpage{\clearpage}

The issue of hydrostatic mass bias is far from solved. Due to the
inconsistent observational and simulated constraints, it is unclear
how large of a hydrostatic correction factor should be included in our
measurements of the gas fraction, or if one is even necessary. A bias
low in the hydrostatic mass would bias the gas fraction high, meaning
that groups and clusters are slightly more deficient of baryons at a
given radius. Our reservations to use a hydrostatic correction factor
are slightly justified by the fact that one of our samples (U09)
measures gas fraction against the weak-lensing mass, and this fraction
agrees well with gas fraction derived from the HSE
assumption.

\subsection{Gas Clumping in Cluster Outskirts}
\label{sec:Limitations.Clumping}

The primary means of deriving the gas density profile of clusters is
from measurements of X-ray surface-brightness, which scales with the
square of electron density. Due to this $n^2$ dependence, clumpy
structures in the ICM will emit more than their share of X-ray,
biasing gas density measurements high. The magnitude of this bias
depends on the smoothness of the ICM gas distribution, which can vary
widely from cluster to cluster. 

Simulations typically predict a clumping bias (overestimate of
$\Mass{}_{gas}$) of $\approx10-15\%$ \citep{Nagai2011,Battaglia2013},
which increases in unrelaxed clusters and towards cluster outskirts,
where recent interactions have a more significant dynamical
effect. This could explain the large differences in measured gas
fraction between CC and NCC clusters, although observations suggest
that the level of clumpiness is overestimated in simulations, and that
the average bias is below $10\%$ \citep{Eckert2013c}. The majority of our
cluster samples represent relaxed (CC) clusters, so we do not include
a clumping bias factor at this time. We also note that the P13
\citep{PlanckIntV} and U09 \citep{Umetsu2009} gas fractions are
derived from SZ measurements, which do not suffer from this clumping
bias, and agree well with our X-ray data sets. 

\subsection{Extrapolation of the Density Profile Slope}
\label{sec:Limitations.Slope}
The choice to extrapolate the gas fraction to $1.2\rvir{}$ and then
stop is arbitrary. The radius at which the gas profile steepens
substantially ($\alpha{}_{gas}$ increases) is not well constrained.
 Between \rfive{} (where $\alpha{}_{gas}$ is well
measured) and \rtwo{}, observations disagree whether the slope remains
roughly constant \citep{Dai2010} or steepens by roughly $10\%$
\citep{Ettori2009a}. Assuming the true evolution is somewhere in
between, the gas density slope should not change appreciably relative
to the total mass (NFW) profile, which steepens by about $4\%$ within
this range. Therefore, our assumption that $\alpha{}_{m}$ and
$\alpha_{gas}$ remain constant beyond \rtwo{} should roughly
approximate the increase of \fg{} with radius. 

It is expected that the slope will eventually asymptote to match the
total mass (NFW) profile \citep{Umetsu2009, Battaglia2013}, suggesting
that, at large radius, $\alpha_{gas}$ will steepen more quickly that
$\alpha_{m}$. The point at which this occurs is unknown, but it will
taper the growth of \fg{}, which should eventually reach a constant
value, similar to the stellar fraction.  \citet{PlanckIntV} finds that
the gas fraction in stacked \Planck{} clusters flattens out between
$1$ and $1.5\rvir{}$. We approximate this by extrapolating \fb{} as
constant above $1.2\rvir{}$. However, until temperature measurements
beyond \rfive{} improve, the true radius where the baryon fraction
reaches a maximum will remain unconstrained. 

\section{Comparison to Simulations}
\label{sec:Simulations}

\begin{figure*}[hbt]
\plotonebig{Copied_Figs/Battaglia_FbvM}
\caption{The gas, stellar, and baryon fraction (scaled by the cosmic
  baryon fraction) as a function of cluster mass from a series of
  simulations by \citet{Battaglia2013}. The baryon fraction within
  \rtwo{} is close to the cosmic value for the entire mass range of
  $10^{14}-10^{15}\Msun{}$, except for in the AGN Feedback model,
  where low-mass clusters have fewer baryons within \rtwo{}. These
  simulations are in great agreement with our results, although the
  simulations predict a large hydrostatic mass bias (Section
  \ref{sec:Limitations.HSE}) that has yet to be confirmed observationally.}
\label{fig:Battaglia_FbvM}
\end{figure*}    


\begin{figure*}[hbtp]
\plotonebig{Copied_Figs/Battaglia_FbvR}
\caption{The gas, stellar, and baryon fraction distributions (scaled
  by the cosmic baryon fraction) as a function of radius in the
  simulations of \citet{Battaglia2013}, binned in four parts by
  cluster mass. The stellar fraction and gas profiles match
  observations well, and the total baryon fraction approaches the
  cosmic value at high radius.}
\label{fig:Battaglia_FbvR}
\end{figure*}    


\begin{figure*}[hbt]
\plotonebig{Copied_Figs/Battaglia_Fgvz}
\caption{The gas fraction (scaled by the cosmic baryon fraction)
  within various radii, as a function of redshift, as measured by the
  simulations of \citet{Battaglia2013}, and binned by mass. There is
  no significant redshift evolution of the gas fraction within any
  given radius, except possibly within $2\rtwo{}$, suggesting that
  measurements of the cluster baryon fraction should be fairly
  redshift independent back to $z\approx1$, where massive clusters
  first begin to form.}
\label{fig:Battaglia_Fgvz}
\end{figure*}    





\section{Implications}
\label{sec:Implications}

\subsection{Deviations from Scale Invariance}
\label{sec:Implications.Invariance}

\subsection{Hierarchical Structure Formation}
\label{sec:Implications.Hierarchical}

\subsection{Where are the Baryons?}
\label{sec:Implications.Where}
