%Senior Thesis Chapter 3
%Ben Cook '14 (bacook@)
%Adviser: Neta Bahcall
\chapter{Results - Where are the Baryons?}
\label{chap:Results}

\section{The Distribution of Gas and Baryons in Groups and Clusters}
\label{sec:Spatial}

Using the observations presented in Chapter \ref{chap:Data}, we have
measurements of the gaseous component of the ICM in halos spanning the
entire mass range from poor groups to the most massive clusters. In
Figure \ref{fig:FgvR}, we present the gas fraction in these groups and
clusters as a function of radius, out to $1.2\rvir{}$. We using the
extrapolation methods described in Section \ref{sec:Gas.Extrapolation}
when necessary for observations which only constrain the gas fraction
within \rfive{} and \rtwo{}. The gas fractions of
each sample (both observed and extrapolated) are listed in Table
\ref{tab:F_gas_all}.

The gas fraction increases with radius in all groups and
clusters. Comparing to the ``expected'' gas fraction of $\approx
15.4\%$, which is the difference between the WMAP9 cosmic baryon
fraction ($16.4\%$) and the cosmic stellar fraction of BK14 ($1\%$),
nearly all groups appear to contain the expected fraction of gas
within the virial radius, or slightly beyond. A large portion of the
ICM is, therefore, ``hidden'' in the outskirts of groups and clusters,
where earlier observations within \rfive{} were not able to probe.

\begin{figure*}[bt]
\plotonebig{Figs_Thesis/FgvR}
\caption{The cumulative hot gas fraction for each cluster sample in
  our study, plotted against the cluster-centric radius. Blue and cyan
  points (G09) represent data from \citet{Giodini2009}, red points
  (P1) are temperature hypothesis 1 from \citet{PlanckIntVb}, green
  points (E13) from \citet{Eckert2013a}, and yellow points (U09) from
  \citet{Umetsu2009}. See Section \ref{sec:Gas.Observations} for
  details of these sources. Many observations are extrapolated to
  $1.2\rvir{}$, as in \citet{Rasheed2011}. The f$_{gas}$ line
  represents the difference between the WMAP9 cosmic baryon fraction
  and the cosmic stellar fraction from \citet{Bahcall2014}. We discuss
  possible biases in using the HSE mass and extrapolating the gas
  density profile in Section \ref{sec:Limitations}. }
\label{fig:FgvR}
\end{figure*}    

%\afterpage{\clearpage}


\begin{table}[hbt]
\caption{Gas Fraction in Groups/Clusters: Observed and Extrapolated}
\scriptsize
\begin{tabular}{lcccccc}
\hline \hline\\
\footnotesize \textbf{Reference} &\footnotesize
\textbf{$\left<\textrm{kT}\right>$} &\footnotesize \textbf{$\alpha{}_{gas}$} &\footnotesize
\textbf{$\fg{}_{,500}$} & \footnotesize \textbf{$\fg{}_{,200}$} &
\footnotesize \textbf{$\fg{}_{,vir}$} & \textbf{$\fg{}_{,1.2vir}$} \\
\footnotesize (1) & \footnotesize (2)& \footnotesize (3)&
\footnotesize (4)& \footnotesize (5)& \footnotesize (6) &
\footnotesize (7) \\\\
\hline
G09 Bin 1 & 0.93 \keV&$1.7\pm0.2$& $0.074 \pm 0.028$ & $0.100\pm0.039^*$ &
$0.131\pm0.052^*$ & $0.156\pm0.062^*$ \\
\phantom{G09} Bin 2 &1.6 \keV&$1.8\pm0.2$ & $0.068 \pm 0.005$ &
$0.091\pm0.009^*$ & $0.117\pm0.014^*$ & $0.137\pm0.019^*$ \\
\phantom{G09} Bin 3 & 2.8 \keV&$1.9\pm0.07$ & $0.080 \pm 0.003$ &
$0.103\pm0.005^*$ & $0.129\pm0.006^*$ & $0.149\pm0.008^*$ \\
\phantom{G09} Bin 4 &  5.0 \keV&$2.1\pm0.02$&  $0.103 \pm 0.008$ & $0.124\pm0.010^*$&$0.146\pm0.012^*$&$0.162\pm0.013^*$ \\
\phantom{G09} Bin 5 & 8.6 \keV&$2.3\pm0.02$&  $0.123 \pm 0.007$ &
$0.137\pm0.008^*$& $0.153\pm0.009^*$ & $0.165\pm0.010^*$\\
PC13 &$\dagger$ & $\dagger$& $0.125\pm0.005$ & $0.137\pm0.003$ &
$0.145\pm0.01$&$0.151\pm0.009$\\
E13 - CC &6.25 \keV& $2.2\pm0.05$& $0.115\pm0.010$ & $0.134\pm0.011$ & $0.153\pm0.013^*$&$0.167\pm0.014^*$ \\
U09 & 9.7 \keV&$2.4\pm0.1$& $0.126\pm0.025$ & $0.133\pm0.027$ & $0.143\pm0.029^*$&$0.151\pm0.031^*$\\
\hline
\end{tabular}
\caption*{\small{(1) Reference abbreviations as in Table
    \ref{tab:F_gas_obs}. \\ (2) The median temperatures of the
    groups/clusters in each sample. \\ (3) The gas density slope
    derived from R11.\\ (4) $\fg(<\rfive)$ (5) $\fg(<\rtwo)$ (6)
    $\fg(<\rvir)$ (7) $\fg(<1.2\rvir)$\\ *: Value
    represents extrapolation using the method of Section
    \ref{sec:Gas.Extrapolation}.\\ $\dagger$: No extrapolation required;
    T and $\alpha_{gas}$ not calculated. }}
\label{tab:F_gas_all}
\end{table}
 
%\afterpage{\clearpage}

Figure \ref{fig:FgvM} shows the halo gas fraction as a function of the
virial mass of the halo. The shortage of gas in low-mass clusters is
apparent at \rfive{}, with low-mass groups and clusters falling
further short of the expected fraction than larger clusters. However,
the shallower slope of the gas density profile in low-mass halos leads
to a dramatic increase in the gas fraction when extrapolated to higher
radius: from \rfive{} to \rvir{}, their gas fraction has almost
doubled. All halos are consistent with containing the expected gas
fraction interior to $\approx1.2\rvir$. Gas may be spread further out
than dark matter, with a large fraction residing in the outskirts of
groups and clusters, but it traces the total dark matter mass content
closely.

\begin{figure*}[hbt]
\plotonebig{Figs_Thesis/FgvM}
\caption{The cumulative hot gas fraction for each cluster sample in
  our study, plotted against the mean halo mass. Blue circles, green
  diamonds, and red squares show the gas fraction measured at
  r$_{500}$, r$_{200}$, and r$_{vir}$, respectively. The f$_{gas}$
  line represents roughly the expected hot-gas fraction, and is the
  the difference between the WMAP9 cosmic baryon fraction and the
  cosmic stellar fraction \citep[$\sim{}1\%$,][]{Bahcall2014}.}
\label{fig:FgvM}
\end{figure*}    

%\afterpage{\clearpage}

\afterpage{\clearpage}

The mass of dark matter halos clearly has an important effect on the
distribution of baryons. We therefore combine our samples of groups
and clusters into two bins: groups/poor clusters
($\Mvir{}<3\power{14}\Msun$), and rich clusters
($\Mvir{}>3\power{14}\Msun$). In Figure \ref{fig:FxvR}, we show the
averaged gas fraction in each bin, as a function of radius. Each point
represents the weighted mean of gas fraction at that radius, for all
clusters in that mass range. These mass ranges translate roughly to
the medium-richness and high-richness bins of BK14,
respectively. Therefore, we also include the average stellar fraction,
which is measured far beyond $\rvir$.

\begin{figure*}[hbt]
\plotonebig{Figs_Thesis/FxvR}
\caption{The cumulative stellar \citep{Bahcall2014} and hot gas (Fig.
  \ref{fig:FgvR}) fractions for groups and clusters, as a function of
  cluster-centric radius. \citet{Bahcall2014} presented the stellar
  fraction for various cluster richness bins. The gas fractions of
  Figure \ref{fig:FgvR} have been sorted into corresponding bins, using the
  mass-richness relation of \citet{Sheldon2009b}.}
\label{fig:FxvR}
\end{figure*}    

\afterpage{\clearpage}

\afterpage{\clearpage}

At \rfive{}, the average gas fraction is $\approx7.5\%$ in groups and
poor clusters and $\approx12\%$ in rich clusters. The gas fraction
increases steeply in groups and poor clusters, reaching about $13\%$
at \rvir{} and $15\%$ at 1.2\rvir{}. In rich clusters, the gas
fraction increases more slowly, reaching $15\%$ at \rvir{}, and $16\%$
at 1.2\rvir{}. The stellar fraction decreases from $2\%$ at \rfive{} to
$1.5\%$ at 1.2\rvir{} in the low-mass bin, and remains steady at $1\%$
in the high-mass bin. We note that the apparent steepening of the
\fg{} profile at \rvir{} is simply a relic of the logarithmic scale of the X
axis. 

We are now able to combine the gas fraction and stellar fraction for
clusters in these two mass ranges, yielding the total baryon fraction
distribution. This is presented in Figure \ref{fig:FbvR}. We find that
the overall baryon fraction increases with radius, reaching the cosmic
baryon fraction at $\approx\rvir{}$ in massive clusters and at
$\approx1.2\rvir{}$ in groups and poor clusters. The entire baryonic
mass associated with the dark matter of groups and clusters is located
within the dark matter halo, indicating that the baryonic matter
content of the universe, when considered on large-enough scales,
clusters identically to and follows the dark matter mass. 

\begin{figure*}[hbt]
\plotonebig{Figs_Thesis/FbvR}
\caption{The cumulative baryon fraction for groups and clusters, as a
  function of cluster-centric radius. The baryon fraction (f$_b$) is
  the sum of the cumulative stellar fraction and the cumulative hot
  gas fraction of Figure \ref{fig:FxvR}. Green squares represent the
  averaged fractions of groups and smaller clusters, while blue
  circles represent larger clutsers. The baryon fraction in large
  clusters appears to approach the cosmic fraction at large scales,
  and we extrapolate the value at larger scales as the value at
  $1.2\rvir{}$, an assumption we discuss in Section \ref{sec:Limitations.Slope}.}
\label{fig:FbvR}
\end{figure*}    

%\afterpage{\clearpage}

\afterpage{\clearpage}

The stellar fraction is observed to approach a constant value at high
radii \citep{Bahcall2014}. The gas fraction is predicted to do the
same, as the gas density profile likely approaches the total mass
(NFW) profile at high radius \citep[e.g.,][and
  refs.~therein]{Umetsu2009}.  We approximate this by assuming that,
beyond $\approx 1.2\rvir{}$, the gas fraction reaches a constant
value, which we represent as an extrapolation of the baryon fraction
out to high radius in Figure \ref{fig:FbvR}. Therefore, current
observations are consistent with the picture that the baryon
distribution matches the distribution of matter well at all scales
larger than the virial radius of clusters. 

\section{The Baryonic Content of Halos}
\label{sec:Baryonic}

We have just shown that, within roughly the virial radius, dark matter
halos of both group and cluster sizes are observed to hold the entire
cosmic baryon fraction. Combined with the measurements of the baryonic
components in galactic halos (Section \ref{sec:Galaxy.Fraction}), we
are able to place global constraints on the baryon fraction within
halos ranging over three orders of magnitude in mass. 

Figure \ref{fig:FbvM} presents the combination of these observations
on the baryon fraction in halos. Shown, as a function of mass, are the
current limits on the baryon fraction for the previously mentioned
samples. Observations of the outskirts of groups and clusters show
that the baryon fraction reaches the cosmic value between \rvir{} and
$1.2\rvir{}$. We plot the baryon fraction for all group and cluster at
these two radii, which is the sum of the gas fraction (Figure
\ref{fig:FgvM}) and stellar fraction (Figure \ref{fig:BK14}) for
clusters of that mass. Observations of galaxy halos show that --
between the stellar disk and ISM, cold CGM, warm CGM, and X-ray CGM --
the entire baryon fraction could be contained within the virial volume
of galaxy halos. We plot the baryon fraction range constrained by
\citet{Werk2014}, which includes lower and upper limits of
$\fb(<\rvir) = 9\% - 19\%$. The best-estimate value, taking the mean
of the ranges given for each component, is $14.5\% \pm 3\%$.

\begin{figure*}[hbt]
\plotonebig{Figs_Thesis/FbvM}
\caption{The cumulative baryon fraction (f$_b$) for galaxies, groups,
  and clusters, as a function of the average mass of the sample. The
  baryon fraction of groups and clusters (filled red points: within
  r$_{vir}$; open green points: within 1.2r$_{vir}$) is the sum of the
  cumulative stellar and hot gas fractions of Figures \ref{fig:BK14}
  and \ref{fig:FgvM}. The baryon fraction of L$^*$ galaxies (open blue
  point) is gathered from \citet{Werk2014}. The arrows indicate upper
  and lower-limits on the galaxy baryon fraction, while the error-bars
  represent the propogated error on the fraction, assuming independent
  uncertainties on the different galactic components (Section
  \ref{sec:Galaxy.Fraction}).}
\label{fig:FbvM}
\end{figure*}                
                                  
%\afterpage{\clearpage}

\afterpage{\clearpage}

Groups and poor clusters are still slightly short of baryons within
the exact virial radius (shown by red points in Figure
\ref{fig:FbvM}. However, there is no reason to assume that the gas
density must cut-off suddenly at this relatively-arbitrary
point. Direct observations \citep{PlanckIntV} indicate that the gas
fraction in massive clusters continues to increase beyond
\rvir{}. Hence groups and poor clusters (where baryons are observed to
be more spread-out than in massive clusters) should also have large
baryon reservoirs beyond this radius, which justifies our choice to
extrapolate \fg{} to $1.2\rvir$. 

From galaxies to the most massive clusters, current observations are
consistent with the entire baryon fraction being contained within the
dark matter halo. Including the baryons out to \rvir{} and slightly
beyond, there is no significant shortage of baryons, even in low-mass
clusters and galactic halos, where previous estimates claimed a
dramatic lack of baryonic mass. While rich clusters tend to have a
higher baryon fraction at small-radii, this is not the case outside
the virial radius. Considering the entire range of mass, there does
not appear to be any strong correlation between enclosed baryon
fraction with mass. No matter the scale of the halo, the entire
expected baryon mass is bound and contained to this halo. This also
suggests that, in voids and regions with less dark matter, there
should be any excess of baryons, as they are not missing from the
dense halos. Therefore, in response to the question ``Where are the
baryons in the universe'', we show the answer is: baryons have fallen
into and stayed within dark matter halos, and trace closely the total
mass distribution in the universe. 


