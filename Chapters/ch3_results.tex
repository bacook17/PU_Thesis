%Senior Thesis Chapter 3
%Ben Cook '14 (bacook@)
%Adviser: Neta Bahcall
\chapter{Results\\Where are the Baryons?}
\label{chap:Results}

\section{The Distribution of Gas and Baryons in Clusters}
\label{sec:Spatial}

Using the observations presented in Chapter \ref{chap:Data}, we have
measurements of the gaseous component of the ICM in halos spanning the
entire mass range from poor groups to the most massive clusters. In
Figure \ref{fig:FgvR}, we present the gas fraction in these groups and
clusters as a function of radius, out to $1.2\rvir{}$. We using the
extrapolation methods described in Section \ref{sec:Gas.Extrapolation}
when necessary for observations which only constrain the gas fraction
within \rfive{} and \rtwo{}. 

\begin{figure*}[bt]
\plotonebig{Figs_Thesis/FgvR}
\caption{The cumulative hot gas fraction for each cluster sample in
  our study, plotted against the cluster-centric radius. Blue and cyan
  points (G09) represent data from \citet{Giodini2009}, red points
  (P1) are temperature hypothesis 1 from \citet{PlanckIntVb}, green
  points (E13) from \citet{Eckert2013a}, and yellow points (U09) from
  \citet{Umetsu2009}. See Section \ref{sec:Gas.Observations} for
  details of these sources. Many observations are extrapolated to
  $1.2\rvir{}$, as in \citet{Rasheed2011}. The f$_{gas}$ line
  represents the difference between the WMAP9 cosmic baryon fraction
  and the cosmic stellar fraction from \citet{Bahcall2014}. We discuss
  possible biases in using the HSE mass and extrapolating the gas
  density profile in Section \ref{sec:Limitations}. }
\label{fig:FgvR}
\end{figure*}    

%\afterpage{\clearpage}


The gas fraction increases with radius in all groups and
clusters. Comparing to the ``expected'' gas fraction of $\approx
15.4\%$, which is the difference between the WMAP9 cosmic baryon
fraction ($16.4\%$) and the cosmic stellar fraction of BK14 ($1\%$),
nearly all groups appear to contain the expected fraction of gas
within the virial radius, or slightly beyond. The gas fractions of
each sample (both observed and extrapolated) are listed in Table
\ref{tab:F_gas_all}.

\begin{table}[hbt]
\caption{Gas Fraction in Groups/Clusters: Observed and Extrapolated}
\scriptsize
\begin{tabular}{lcccccc}
\hline \hline\\
\footnotesize \textbf{Reference} &\footnotesize
\textbf{$\left<\textrm{kT}\right>$} &\footnotesize \textbf{$\alpha{}_{gas}$} &\footnotesize
\textbf{$\fg{}_{,500}$} & \footnotesize \textbf{$\fg{}_{,200}$} &
\footnotesize \textbf{$\fg{}_{,vir}$} & \textbf{$\fg{}_{,1.2vir}$} \\
\footnotesize (1) & \footnotesize (2)& \footnotesize (3)&
\footnotesize (4)& \footnotesize (5)& \footnotesize (6) &
\footnotesize (7) \\\\
\hline
G09 Bin 1 & 0.93 \keV&$1.7\pm0.2$& $0.074 \pm 0.028$ & $0.100\pm0.039^*$ &
$0.131\pm0.052^*$ & $0.156\pm0.062^*$ \\
\phantom{G09} Bin 2 &1.6 \keV&$1.8\pm0.2$ & $0.068 \pm 0.005$ &
$0.091\pm0.009^*$ & $0.117\pm0.014^*$ & $0.137\pm0.019^*$ \\
\phantom{G09} Bin 3 & 2.8 \keV&$1.9\pm0.07$ & $0.080 \pm 0.003$ &
$0.103\pm0.005^*$ & $0.129\pm0.006^*$ & $0.149\pm0.008^*$ \\
\phantom{G09} Bin 4 &  5.0 \keV&$2.1\pm0.02$&  $0.103 \pm 0.008$ & $0.124\pm0.010^*$&$0.146\pm0.012^*$&$0.162\pm0.013^*$ \\
\phantom{G09} Bin 5 & 8.6 \keV&$2.3\pm0.02$&  $0.123 \pm 0.007$ &
$0.137\pm0.008^*$& $0.153\pm0.009^*$ & $0.165\pm0.010^*$\\
PC13 &$\dagger$ & $\dagger$& $0.125\pm0.005$ & $0.137\pm0.003$ &
$0.145\pm0.01$&$0.151\pm0.009$\\
E13 - CC &6.25 \keV& $2.2\pm0.05$& $0.115\pm0.010$ & $0.134\pm0.011$ & $0.153\pm0.013^*$&$0.167\pm0.014^*$ \\
U09 & 9.7 \keV&$2.4\pm0.1$& $0.126\pm0.025$ & $0.133\pm0.027$ & $0.143\pm0.029^*$&$0.151\pm0.031^*$\\
\hline
\end{tabular}
\caption*{\small{(1) Reference abbreviations as in Table
    \ref{tab:F_gas_obs}. \\ (2) The median temperatures of the
    groups/clusters in each sample. \\ (3) The gas density slope
    derived from R11.\\ (4) $\fg(<\rfive)$ (5) $\fg(<\rtwo)$ (6)
    $\fg(<\rvir)$ (7) $\fg(<1.2\rvir)$\\ *: Value
    represents extrapolation using the method of Section
    \ref{sec:Gas.Extrapolation}.\\ $\dagger$: No extrapolation required;
    T and $\alpha_{gas}$ not calculated. }}
\label{tab:F_gas_all}
\end{table}
 

Figure \ref{fig:FgvM} shows the halo gas fraction as a function of the
virial mass of the halo. 

\begin{figure*}[hbt]
\plotonebig{Figs_Thesis/FgvM}
\caption{The cumulative hot gas fraction for each cluster sample in
  our study, plotted against the mean halo mass. Blue circles, green
  diamonds, and red squares show the gas fraction measured at
  r$_{500}$, r$_{200}$, and r$_{vir}$, respectively. The f$_{gas}$
  line represents roughly the expected hot-gas fraction, and is the
  the difference between the WMAP9 cosmic baryon fraction and the
  cosmic stellar fraction \citep[$\sim{}1\%$,][]{Bahcall2014}.}
\label{fig:FgvM}
\end{figure*}    

%\afterpage{\clearpage}



\section{The Baryonic Content of Halos}
\label{sec:Baryonic}
