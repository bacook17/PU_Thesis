%Senior Thesis - First Chapter
%Ben Cook '14 (bacook@)
%Adviser: Neta Bahcall

\chapter{Introduction}
\label{c.Intro}
\section{The Cosmic Baryon Fraction}
\label{s.BaryonFraction}
In the beginning, there was the big bang. All of the contributions to
the cosmic energy budget -- all of the forms that matter and energy
take today -- originated around 13.8 billion years ago, when the
universe was unimaginably hot and dense. Various
mechanisms\footnote{most notably the theory of inflation, which will
  surely see a dramatic increase in interest since the detections of
  B-mode polarization in the CMB \citep{BICEP22014}.} have been
offered to explain the genesis of the energy-filled, rapidly expanding
universe which appeared in the briefest fraction of a second after the
big bang. As the universe expanded and cooled, the available energy
distributed itself into various different forms, eventually settling
down into the primary energy components we observe today, including
radiation (photons and neutrinos), baryons (``ordinary'' matter,
comprised of protons, neutrons, and electrons), as well as the
mysterious dark matter and dark energy. 

Energy did not populate these forms in equal proportions; the energy
densities of each population differed by many orders of magnitude,
initially, and their ratios changed continually throughout the
expansion history of the universe. Radiation density -- primarily the
photon density ($\rho_\gamma$), dominant in the earliest periods after
the big bang -- diluted quickly from the combination of expansionary
volume increase and Doppler redshifting. The matter density --
$\rho_m$, comprised of both baryons ($\rho_b$) and cold dark matter
($\rho_c$) -- was initially only a miniscule portion of the cosmic
energy budget, but eventually matter dominated the cosmic scene after
expansion ``cooled'' the photon temperature significantly. Buried far
below the other components originally, dark energy -- $\rho_\nu$,
commonly thought to be a cosmological constant $\Lambda$ -- retains a
constant energy density while the universe expands, becoming dominant
at late times when \rhom has decreased significantly. Each of these
energy densities are often scaled by the critical density required to
stop cosmic expansion: $\rhocrit = \frac{8}{3}\pi{}G\Ho{}^2c^{-2}$,
with \Ho{} the Hubble constant, G Newton's constant, and c the speed
of light. The density of each component relative to the critical
density is expressed as $\Omega$, for example:
\begin{equation}
\frac{\rhob}{\rhocrit} = \omegab.  \nonumber
\end{equation}

Baryons and dark matter, the two components of the total matter
density, have the same dependence on the expansionary scale factor
and redshift. Since the total matter density is just their sum, the
total matter density also scales identically:
\begin{align}
\rhoc \propto \rhob \propto \rhom & \propto \textrm{a}^{-3} \nonumber \\
& \propto (1+\textrm{z})^{3}. \nonumber
\end{align}
Therefore, we see that the ratio of baryons to dark matter will remain
constant from its primoridal level throughout the history of the
universe. A more useful and commonly studied constant is the
\textit{cosmic baryon fraction}, \fb, the fraction of all matter in
baryonic form:
\begin{equation}
\fb = \frac{\rhob}{\rhob + \rhoc} = \frac{\rhob}{\rhom} =
\frac{\omegab}{\omegam}.
\end{equation}

The above argument, that the baryon fraction remains constant
throughout cosmic history, applies only in the homogenous regime,
where there are no spatial variations in the overall mass
density. When inhomogeneities exist, self-interactions lead to the
complicated evolution of structure. Both baryons and dark matter are
subject to gravitational forces, causing intitial overdensities to
increase in magnitude, eventually collapsing into massive halos. Yet
while dark matter primarily interacts only through gravity, baryons
are subject to electromagnetic forces, thermal emission, and numerous
other interactions that lead to a divergence between the dark matter
distribution and the baryon distribution.

While both dark matter and baryons cause the gravitational collapse of
inhomogeneities and drive the growth of structure, it is obvious that
the baryon fraction plays an incredibly important role in determining
the makeup of our universe. Baryons are responsible for all other
phenomena studied in physics and astrophysics: the formation of
galaxies and stars, supernovae, photon radiation, and, eventually,
life itself. Although the local baryon fraction may vary hugely from
place to place, it is possible to estimate the cosmic baryon fraction
by averaging over a substantially large volume. This cosmic ratio is
a defining characteristic, and studying it is essential to properly
understanding the creation and evolution of our universe. \todo{This
  section can use some major work. Narrow the focus, but still give it
  some depth.}

The baryon fraction was a major factor in several important physical
processes in the early universe. Through observable consequences of
these processes, cosmologists have been able to place powerful
constraints on the cosmic baryon fraction at these early
times. Consistent with the literature, we will consider the baryon
fraction inferred from these methods to be the ``true'' cosmic
fraction against which we will compare measurements from the local,
highly inhomogenous universe.

One such mechanism is big bang nucleosynthesis (BBN), which generated
the first light elements beyond hydrogen\footnote{The discussion which
  follows is guided primarily by chapter 3.2 of \citet{Weinberg2008},
  a useful but relatively technical reference on the topic.}. In the
first seconds after the big bang, the only ordinary matter particles
which existed (and were stable) were protons ($p$), electrons
($e$), and neutrons ($n$). The high temperatures and densities of
nucleons allowed the conversion of protons and neutrons into more
complex and heavier nucleii, through processes such as:
\begin{align}
p + n &\rightarrow{} d + \gamma \nonumber \\
d + d &\rightarrow{} ^3\textrm{He} + n \nonumber \\
d +{} ^3\textrm{He} &\rightarrow{} ^4\textrm{He} + p .\nonumber
\end{align}
The BBN reactions began around 100 -- 200 seconds after the big bang
\citep{Weinberg2008}. The exact time when these reactions reached
thermal equilibrium depends weakly on the abundance of baryons,
$\omegab h^2$, where $\Ho = 100h~\kms Mpc^{-1}$ defines $h$, an
important scaling factor which we will discuss later. After the
universe expanded and cooled sufficiently, these reactions fell out of
equilibrium, leaving the universe enriched with helium ($^4$He and
$^3$He) and trace amounts of elements such as deuterium ($d$) and
lithium ($^7$Li). The transformation of $n$ and $d$ into helium is
more complete the higher $\omegab h^2$. Therefore, the baryon
abundance strongly affects the resulting abundance of $d$ and residual
elements like lithium and $^3$He. Figure \ref{fig:Abundances} shows
the dependance of these primordial abundances on the cosmic baryon
abundance.

\begin{figure*}[hbt]
\plotonebig{Abundances}
\caption{The BBN-predicted primordial abundances of deuterium (D),
  $^3$He, $^7$Li, and $^4$He (Y$_\textrm{P}$), as a function of the
  baryon abundance parameter $\eta_{10} \sim 274 \times \omegab
  h^2$. The width of the curves represents the uncertainties in
  various nuclear reaction rates. Figure taken from
  \citet{Steigman2006}, another helpful overview of BBN physics.}
\label{fig:Abundances}
\end{figure*}    



The baryon abundance can be determined through observational
constraints of, for example, the deuterium abundance, which among the
common byproducts of BBN depends most strongly on $\omegab h^2$. The
deuterium abundance has been inferred from variety of sources,
including from the Milky Way's ISM \citep{Linsky1993, Linsky1995},
absorption towards QSOs \citep{Tytler1996, Kirkman2003}, and even from
measurements of the composition of the Jovian atmosphere
\citep{Niemann1996}. All such methods have limitations, as deuterium
can be destroyed in stellar (and brown dwarf) cores, altering the
deuterium abundance slightly with time. \citet{Iocco2009} provides a
modern compilation of deuterium abundance measurements, placing the
constraint on the baryon abundance at $\omegab h^2 = 0.021 \pm
0.001$. Observed $^7$Li abundances are a factor of a few lower than
predicted from BBN, suggesting that there could be additional physics
responsible for destroying lithium \citep{Suzuki2000,
  Melendez2004}. This is known as the ``Lithium Problem,'' and is
still an unsolved problem in interpreting BBN.

A complementary method of inferring the baryon abundance at early
times is from measurements of the acoustic peaks in the Cosmic
Microwave Background (CMB) power spectrum. Remember that the CMB power
spectrum contains large peaks, representing correlations in the CMB on
particular scales. In the first few hundred thousand
years\footnote{Virtually a cosmic blink of the eye.} after the big
bang, the temperature of the universe was high enough to keep atoms
fully ionized into seperate nucleii and electrons. These charged
particles were strongly coupled to the photons through electromagnetic
interactions, so that the two combined to form what is called a
photon-baryon fluid. This fluid (which, at the time, had an energy
density around $\frac{1}{3}$ that of dark matter) fell towards the
centers of gravitational wells created by dark matter. However, unlike
the non-interacting dark matter, the photon-baryon fluid's pressure
rose when its density rose, forcing the fluid out of the well
until its pressure dropped enough to allow gravity to draw it back
once again. These ongoing fluctuations in the pressure and density
of the fluid were frozen into the CMB when the average temperature of
the universe dropped sufficiently to allow neutral atoms to form, and
the CMB photons began streaming freely through the
universe\footnote{For a good, low-level introduction to the concept of
  acoustic peaks in the CMB, see Chapter 9 of \citet{Ryden2003}.}.

The baryon abundance at this early time had several effects on the
acoustic peaks in the CMB, as did the overall mass abundance ($\omegam
h^2$). Figure \ref{fig:CMB_Power} shows how changes in these
abundances are reflected in the acoustic peaks. The location of the
first peak (and all subsequent peaks) is determined by the sound speed
at the epoch of last scattering. This sound speed increases when
$\omegab h^2$ increases, but is more sensitive to changes in $\omegam
h^2$, an increase of which results in a decrease in the sound speed
\citep[][ch.~9.8]{Mukhanov2005}. The relative heights of the acoustic
peaks is a further diagnostic of the baryon abundance. As seen in
Figure \ref{fig:CMB_Power}, odd-numbered peaks are higher than
even-numbered peaks in a universe with high $\omegab h^2$. This is
because an increase in the amount of massive baryons reduces the
frequency of acoustic oscillations
\citep[][ch.~8.7.3]{Dodelson2003}. Finally, the power spectrum
declines towards higher multipoles ($l$) due to a process known as
``Silk Damping''. This damping term is due to imperfections in the
photon-baryon coupling, and the characteristic damping scale is
influenced by $\omegab h^2$ \citep[][ch.~4.7]{Durrer2008}.

Through a combination of all the processes listed above, modern CMB
observations have been able to place strong constraints on both
$\omegab h^2$ and $\omegam h^2$. Two of the most noteworthy such
measurements come from the \textit{Wilkinson Microwave Anisotropy
  Probe} \citep[WMAP][]{Bennett2003} and the \textit{Planck} Satellite
\citep{PlanckResultsI}. Different constraints are placed on these
parameters depending on what additional data\footnote{\Ho, BAO,
  Polarization, etc.} is included in the analysis. We will take the
median value (from each paper) of parameters derived through these
various means, and we list the uncertainty as the systematic difference between
the inferred parameters (which is of the same order, but typically
larger than the statistical uncertainty of any one value).

From the 9-year data release of WMAP \citep{Hinshaw2013}, we take
values $\omegab h^2 = 0.02229 \pm 0.00035$ and $\omegac h^2 = 0.1138
\pm 0.0032$. From the results paper of Planck
\citep{PlanckResultsXVI}, we take the values $\omegab h^2 = 0.022115
\pm 0.0001$ and $\omegac h^2 = 0.11957 \pm 0.0011$. These values allow
us to constrain the cosmic baryon fraction as:
\begin{align}
\fb &= 0.164 \pm 0.004 \,\,(WMAP) \nonumber\\
&= 0.156 \pm 0.002 \,\,(Planck) \nonumber
\end{align}
It is a commonly
discussed fact \citep[e.g.~][]{Spergel2013} that the Planck results
yield \omegam significantly higher and \Ho significantly lower than
other previous measurements, including WMAP. Throughout this paper, we
will 

\begin{figure*}[hbtp]
\plotonebig{CMB_Power}
\caption{The variation in the CMB power spectrum's acoustic peaks due
  to variations in cosmic abundance parameters. The thick black line
  represents a fiducial universe with $\omegam h^2 = 0.16$, $\omegab
  h^2 = 0.021$, and $\omegal = 0.7$. Other lines represent the results
  from changes to these parameters, notably an increase in peak height
  and rightward shift of peaks ($\Delta l > 0$) with an increase in
  $\omegab h^2$. A decrease in $\omegam h^2$ has similar (but
  distinguishable) effects. Figure from \citet[][Fig.~8.19]{Dodelson2003}.}
\label{fig:CMB_Power}
\end{figure*}    

\afterpage{\clearpage}


\section{The ``Missing Baryon'' Problem...}
\label{s.Missing}
\subsection{In Clusters}
\label{s.Missing.Clusters}
\subsection{In Galaxies}
\label{s.Missing.Galaxies}
And some more stuff.\\
