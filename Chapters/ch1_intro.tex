%Senior Thesis - First Chapter
%Ben Cook '14 (bacook@)
%Adviser: Neta Bahcall

\chapter{Introduction}
\label{chap:Intro}
\section{The Cosmic Matter-Energy Components}
\label{sec:Intro.Components}
After the big bang, the energy content of the universe was distributed
in a diverse number of components. As the universe expanded and
cooled, the available energy settled down into the primary energy
forms we observe today, including radiation (photons and neutrinos),
baryons (``ordinary'' matter, comprised of protons, neutrons, and
electrons), as well as the mysterious dark matter and dark energy.

Energy did not populate these forms in equal proportions; the energy
densities of each component differed by many orders of magnitude,
initially, and their ratios changed continually throughout the
expansion history of the universe. Radiation density -- primarily the
photon density (\rhog), dominant in the earliest periods after the big
bang -- diluted quickly from the combination of expansionary volume
increase and Doppler redshifting. The matter density -- \rhom,
comprised of both baryons (\rhob) and cold dark matter (\rhoc) -- was
initially only a minuscule portion of the cosmic energy budget, but
eventually matter dominated the cosmic scene after expansion
``cooled'' the photon temperature significantly. Initially
insignificant relative to the energy content of the other components,
dark energy -- $\rho_\Lambda$, commonly thought to be a cosmological
constant $\Lambda$ -- retains a constant energy density while the
universe expands and became dominant at late times when \rhom{} had
decreased significantly. Each of these energy densities are often
scaled by the critical density required to stop cosmic expansion:
\begin{equation}
\rhocrit = \frac{3H^2}{8\pi{}G} 
\end{equation}
with $H$ the Hubble parameter and $G$ Newton's constant. The density of each component relative to the critical
density is expressed as $\Omega$. For example:
\begin{equation}
\frac{\rhob}{\rhocrit} = \omegab\;.
\end{equation}


\section{The Cosmic Baryon Fraction}
\label{sec:Intro.Baryons}
Baryons and dark matter, the two components of the total matter
density, have the same dependence on the expansionary scale factor
and redshift. Since the total matter density is just their sum, the
total matter density also scales identically:
\begin{align}
\rhoc \propto \rhob \propto \rhom & \propto a^{-3} \nonumber \\
& \propto (1+z)^{3}\;. 
\end{align}
Therefore, we see that the ratio of baryons to dark matter will remain
constant at its primordial level throughout the history of the
universe. A useful and commonly studied constant is the \textit{cosmic
  baryon fraction}, \fb, the fraction of all matter in baryonic form:
\begin{equation}
\fb = \frac{\rhob}{\rhob + \rhoc} = \frac{\rhob}{\rhom} =
\frac{\omegab}{\omegam}\;.
\end{equation}

The above argument, that the baryon fraction remains constant
throughout cosmic history, applies only in the homogeneous regime,
when there are no spatial variations in the overall mass density. If
inhomogeneities exist, self-interactions lead to the complicated
evolution of structure. While dark matter (by energy density)
dominates the gravitational collapse of inhomogeneities and drives the
growth of structure, it is obvious that the baryon abundance plays
an incredibly important role in determining the makeup of our
universe. Dark matter interacts only through gravity, but baryons are
subject to electromagnetic forces, thermal emission, collisions, and
numerous other interactions collectively known as ``baryonic
physics''. Baryons are solely responsible for all non-gravitational
phenomena studied in physics and astrophysics: the formation of
galaxies, stars, and planets, supernovae, radiation, magnetism,
chemistry, and, eventually, life itself. 

Initially, baryons fall into the gravitational potentials made by dark
matter (known as ``dark matter halos'') in a proportion equal to the
cosmic baryon fraction. Over time, the baryonic gas cools, forming
stars and galaxies, and baryonic physics can drive dramatic changes in
the distribution of baryons relative to dark matter. Halos are
initially populated by the cosmic fraction of baryons, but mechanisms
such as shock heating, galactic winds, or AGN activity could drive
baryons out of the potential well altogether, leaving systems
deficient of baryons relative to the cosmic fraction. Although the
local baryon fraction may vary within different regions of virialized
systems, it is possible to estimate the baryon fraction in these
systems by averaging over a substantially large volume. By studying
the baryon fraction (and what forms the baryons take) we can learn
about the relative distributions of dark matter and baryons, and the
contributions of baryonic and gravitational physics to the formation
of structure.

The baryon and dark matter abundances were major factors in several
important physical processes in the early universe. Through observable
signatures of these processes, cosmologists have been able to place
powerful constraints on the cosmic baryon fraction at these early
times. Consistent with the literature, we will consider the baryon
fraction inferred from these methods to be the cosmic fraction against
which we will compare measurements from the local, highly
inhomogeneous universe.

One mechanism that constrains the baryonic abundance is big bang
nucleosynthesis (BBN), the process which generated the first light
elements beyond hydrogen\footnote{The discussion which follows is
  guided primarily by Chapter 3.2 of \citet{Weinberg2008}, a useful
  but relatively technical reference on the topic.}. In the first
seconds after the big bang, the only ordinary matter particles which
existed (and were stable) were protons ($p$), electrons ($e$),
neutrons ($n$), and deuterium ($d$). The high temperatures and densities of nucleons
allowed the conversion of protons and neutrons into more complex and
heavier nuclei, through processes such as:
\begin{align}
p + n &\rightarrow{} d + \gamma \nonumber \\
d + d &\rightarrow{} ^3\textrm{He} + n \nonumber \\
d +{} ^3\textrm{He} &\rightarrow{} ^4\textrm{He} + p\; .\nonumber
\end{align}
The BBN reactions began around 100 -- 200 seconds after the big bang
\citep{Weinberg2008}. The exact time when these reactions reached
thermal equilibrium depends weakly on the abundance of baryons,
$\omegab h^2$, where $\Ho = 100h~\kmsMpc$ defines $h$, an important
scaling factor\footnote{Uncertainties on the true expansion rate ($h$)
  translate into uncertainties on a number of observables such as
  matter abundances ($\Omega$) and dark matter halo properties
  (\Mass{} or \radius{}). Therefore, many derived properties in the
  literature are often given in terms of $h_{70} \equiv{} \Ho /
  70~\kmsMpc$.}. After the universe expanded and cooled sufficiently,
these reactions fell out of equilibrium, leaving the universe enriched
with helium ($^4$He and $^3$He) and trace amounts of elements such as
deuterium and lithium ($^7$Li). The higher $\omegab h^2$, the more
complete the transformation of $n$ and $d$ into helium. Therefore, the
baryon abundance strongly affects the resulting abundance of deuterium
and residual elements like lithium and $^3$He. Figure
\ref{fig:Abundances} shows the dependence of these primordial
abundances on the cosmic baryon abundance.

\begin{figure*}[hbtp]
\plotonebig{Copied_Figs/Abundances}
\caption{The BBN-predicted primordial abundances of deuterium (D),
  $^3$He, $^7$Li, and $^4$He (Y$_\textrm{P}$), as a function of the
  baryon abundance parameter $\eta_{10} \sim 274 \times \omegab
  h^2$. The width of the curves represents the uncertainties in
  various nuclear reaction rates. Figure taken from
  \citet[][Fig.~1]{Steigman2006}, another helpful overview of BBN physics.}
\label{fig:Abundances}
\end{figure*}    

%\afterpage{\clearpage}


The baryon abundance can be determined through observational
constraints of, for example, the deuterium abundance, which among the
byproducts of BBN depends most strongly on $\omegab h^2$. The
deuterium abundance has been inferred from variety of sources,
including from the Milky Way's ISM \citep{Linsky1993, Linsky1995},
absorption towards QSOs \citep{Tytler1996, Kirkman2003}, and even from
measurements of the composition of the Jovian atmosphere
\citep{Niemann1996}. All such methods have limitations, as deuterium
can be destroyed in stellar (and brown dwarf) cores, altering the
deuterium abundance slightly with time. \citet{Iocco2009} provides a
compilation of deuterium abundance measurements, placing a
constraint on the baryon abundance of $\omegab h^2 = 0.021 \pm
0.001$. Observed $^7$Li abundances are a factor of a few lower than
predicted from BBN, suggesting that there could be additional physics
responsible for destroying lithium \citep{Suzuki2000,
  Melendez2004}. This is known as the ``Lithium Problem,'' and is
still an unsolved problem in interpreting BBN.

A complementary method of inferring the baryon abundance at early
times is from measurements of the acoustic peaks in the Cosmic
Microwave Background (CMB) fluctuations power spectrum. The CMB power
spectrum contains large peaks, representing correlations in the CMB on
particular scales. In the first few hundred thousand years after the
big bang, the temperature of the universe was high enough to keep
atoms fully ionized into separate nuclei and electrons. These charged
particles were strongly coupled to the local photon field through
Thompson Scattering, so that the two components combined to form what
is called a photon-baryon fluid. This fluid fell towards the centers
of gravitational wells created by dark matter. However, unlike the
non-interacting dark matter, the photon-baryon fluid's pressure rose
when the density rose, stopping its infall. The pressure build-up
forced the fluid back out of the well until its pressure dropped
enough to allow gravity to draw it back in once again. These ongoing
fluctuations in the pressure and density of the fluid were frozen into
the CMB power spectrum as the acoustic peaks. This occurred when the
average temperature of the universe dropped sufficiently to allow
neutral atoms to form, and the CMB photons began streaming freely
through the universe, around redshift $z=1100$, when the universe's
average temperature was $T\approx 3000\textrm{K}$\footnote{For an
  introductory discussion on the concept of acoustic peaks in the CMB,
  see Chapter 9 of \citet{Ryden2003}.}.

The baryon abundance at this early time had several effects on the
acoustic peaks in the CMB, as did the cold dark matter abundance
($\omegac h^2$). Figure \ref{fig:CMB_Power} shows how changes in these
abundances are reflected in the acoustic peaks. The location of the
first peak (and all subsequent peaks) is determined by the sound speed
at the epoch of last scattering. This sound speed decreases when
$\omegab h^2$ increases \citep[][ch.~9.8]{Mukhanov2005}. The total
curvature of the universe (determined primarily at this time by
$\omegac h^2$) also shifts the location of each peak. The relative
heights of the acoustic peaks is a further diagnostic of the baryon
abundance. As seen in Figure \ref{fig:CMB_Power}, odd-numbered peaks
are higher than even-numbered peaks in a universe with high $\omegab
h^2$. This is because an increase in the amount of massive baryons
reduces the frequency of acoustic oscillations
\citep[][ch.~8.7.3]{Dodelson2003}. High $\omegac h^2$ decreases the
overall height of each peak. Finally, the power spectrum declines
towards higher multipoles ($l$) due to a process known as ``Silk
Damping''. This damping term is due to imperfections in the
photon-baryon coupling, and the characteristic damping scale is
influenced by $\omegab h^2$ \citep[][ch.~4.7]{Durrer2008}.

Through a combination of all the processes listed above, modern CMB
observations have been able to place strong constraints on both
$\omegab h^2$ and $\omegam h^2$. Two of the most noteworthy such
measurements come from the \textit{Wilkinson Microwave Anisotropy
  Probe} \citep[\WMAP{},][]{Bennett2003} and the \textit{Planck} Satellite
\citep{PlanckResultsI}. Different constraints are placed on these
parameters depending on what additional data\footnote{\Ho, BAO,
  polarization, etc.} is included in the analysis. We will use the
median value (from each paper) of parameters derived through these
various methods, and we use the systematic variance in the parameters as
the uncertainty, if it is larger than the statistical uncertainty
listed from the analysis.

From the 9-year data release of \WMAP{} \citep[\WMAP9,][]{Hinshaw2013},
we use $\omegab h^2 = 0.02229 \pm 0.00035$ and $\omegac h^2 =
0.1138 \pm 0.0032$. From the results paper of \Planck{}
\citep{PlanckResultsXVI}, we use $\omegab h^2 = 0.022115
\pm 0.00025$ and $\omegac h^2 = 0.11957 \pm 0.0025$. These values
allow us to constrain the cosmic baryon fraction as:
\begin{align}
\fb &= \frac{\omegab h^2}{\omegab h^2 + \omegac h^2} \\
&=0.164 \pm 0.004 \,\,(\WMAP9) \nonumber\\
&= 0.156 \pm 0.003\,\,(\Planck) \nonumber\\ 
\intertext{The uncertainty on \fb{} comes from the propogation of
  uncertainties on $\omegab h^2$ and $\omegac h^2$:}
\Delta\fb &=
\sqrt{\left(\frac{\partial\fb}{\partial\omegab h^2}\right)^2\Delta(\omegab h^2)^2 
+ \left(\frac{\partial\fb}{\partial\omegac h^2}\right)^2\Delta(\omegac h^2)^2} \nonumber \\
&= \sqrt{\left(\frac{\omegac h^2 \Delta(\omegab h^2)}{(\omegab h^2 + \omegac h^2)^2}\right)^2
+ \left(\frac{\omegab h^2 \Delta(\omegac h^2)}{(\omegab h^2 + \omegac h^2)^2}\right)^2} \nonumber \\
&= \frac{1}{(\omegam h^2)^2}\sqrt{\left[\omegac h^2 \Delta(\omegab h^2)\right]^2
+ \left[\omegab h^2 \Delta(\omegac h^2)\right]^2} 
\end{align}

It is commonly discussed \citeeg{Spergel2013} that the \Planck{}
results yield \omegam{} significantly higher and \Ho{} significantly
lower than other previous measurements, including \WMAP{}. This
results in the lower value of \fb{} as derived from \Planck{}. Earlier
estimates \citep[e.g.,\WMAP5,][]{Dunkley2009} placed \fb{} around
0.17, a canonical value which is often used in the literature. Here,
we will use the WMAP9 measurement of $\fb{}\sim0.164$ for our primary
comparisons with low-redshift baryon fractions, although we will
attempt to also compare to the Planck constraints whenever
possible. One limitation of this work is the relative uncertainty on
the ``cosmic'' baryon fraction due to the discrepancies in CMB
observations. We hope that subsequent analysis will resolve this
problem.

\begin{figure*}[hbt]
\plotonebig{Copied_Figs/CMB_Power}
\caption{The variation in the CMB power spectrum's acoustic peaks due
  to variations in cosmic abundance parameters. The thick black line
  represents a fiducial universe with $\omegam h^2 = 0.16$, $\omegab
  h^2 = 0.021$, and $\omegal = 0.7$. Other lines represent the results
  from changes to these parameters, notably an increase in peak height
  and rightward shift of peaks ($\Delta l > 0$) with an increase in
  $\omegab h^2$. A decrease in $\omegam h^2$ has similar (but
  distinguishable) effects. Figure from \citet[][Fig.~8.19]{Dodelson2003}.}
\label{fig:CMB_Power}
\end{figure*}    

%\afterpage{\clearpage}


\section{The Cluster Missing Baryon Problem}
\label{sec:Missing.Clusters}

Large \textit{galaxy clusters} represent the most massive dark matter
systems which have had adequate time to virialize since the big
bang. The richest clusters often contain hundreds of large galaxies,
while much smaller \textit{groups} may contain several to tens of
galaxies. The galaxies are bound to the gravitational well of the
combined mass of the cluster, which includes dark matter as well as
baryons. Some clusters show a steep pressure profile (and a flat
density profile) towards their centers, suggesting that they are
dynamically \textit{relaxed} and possess a \textit{cool core} (these
are known as cool core or CC clusters). Clusters with flattened
central pressure profiles (and cuspy density profiles) are
\textit{unrelaxed} or \textit{non-cool core} (NCC) clusters. Because
they are the largest bound objects in the universe, clusters offer one
of the best ways of measuring the baryon fraction at low-redshift:
they have collapsed from large volumes ($\sim8 \textrm{Mpc} h^{-1} in
radius$) and therefore represent the largest samples of the cosmic
dark matter and baryon densities that are not in dramatic dynamical
evolution. Additionally, their deep gravitational potential wells are
expected to prevent any significant fraction of baryons from escaping
the system through feedback effects, such as winds, supernovae, or AGN
activity.

The baryonic component of groups and clusters is not primarily stored in
galaxies, but in a hot ($\textrm{kT} \gsim 1$ keV), diffuse gas known
as the \textit{Intracluster Medium} (ICM). The ICM can be detected
through the X-ray Brehmsstrahlung emission of the ionized
plasma. X-ray telescopes such as \Chandra, \Rosat, \Suzaku, and \XMM{}
have been key in detecting the ICM and constraining its distribution
and total mass \citeeg{Vikhlinin2006,Eckert2012}. A complementary
method of detecting and measuring the intracluster gas is through the
thermal Sunyaev-Zeldovich effect \citep[SZ,][]{Sunyaev1972}, which
occurs when the energy spectrum of background CMB photons passing
through the cluster is altered due to inverse-Compton scattering with
the charged particles in the plasma.

The stars of individual galaxies make up the remainder of the baryonic
mass in groups and clusters. Stellar mass is derived from the
luminosity in starlight, which is converted into mass using the
observed colors or spectra of galaxies, or from the typical stellar
mass to light ratios derived for individual galaxies
\citep{Bahcall2014}.

The total mass of groups and clusters is derived through stacked
weak-lensing analysis, where the average distortion of background
galaxies is measured as a function of cluster-centric radius. These
distortions can be inverted to compute the distribution of mass in the
cluster which is responsible for the gravitational lensing
distortions. Recent weak-lensing analyses \citep{Mandelbaum2008,
  Sheldon2009a} yield accurate constraints on the total mass of
clusters, and show that the distribution of mass is fit well by the
Navarro-Frenk-White profile \citep[NFW,][]{Navarro1996}, a prediction
from N-body simulations of cold dark matter.

Before addressing the overall masses and sizes of galaxy clusters, it
is important to outline the definitions for these scales we will use
throughout this paper. Because dark matter halos are believed to be
self-similar (scaling only by overall mass or central density) their
sizes are often given relative to fixed overdensities $\Delta$, the
location where the density of matter interior is a particular multiple
of the critical density of the universe. For example, $\Delta = 200$
equates to the region in a system where
\begin{equation}
\rhom(<r) = \frac{3M(<r)}{4\pi{}r^3} = 200\times\rhocrit\;. 
\end{equation}
The mass and radius of a group or cluster is typically measured at a
characteristic overdensity. For example, many sources in the
literature list the group and cluster mass and radius as \Mfive{} and
\rfive{}, measured at an overdensity of $\Delta = 500$. As cluster
mass-density increases towards the center, $\Delta$ decreases towards
larger cluster-centric radii. The gravitational system virializes
around $\Delta = 100$ in the $\Lambda{}$CDM cosmology \citep{Eke1996},
so we take the characteristic scales of groups and clusters to be the
virial mass and radius, \Mvir{} $ \equiv $ M$_{100}$ and
\rvir{} $ \equiv $ r$_{100}$.

Through stacked weak lensing measurements, rich galaxy clusters have
been observed to have virial masses of $10^{14}\dash10^{15}~\Msun{}$
\citep{Mandelbaum2008}, and virial radii of around $1\dash3$ Mpc
\citep{Vikhlinin2006}. Lower-mass groups typically have masses
around $10^{13} \dash 10^{14}~\Msun{}$, and represent the intermediate
range between the most massive clusters and large galaxies. The
self-similarity of group and cluster halos results in a fairly
constant relation between masses or radii at various overdensities. For
example \citep{Rasheed2011}:
\begin{equation}
\rvir{} \approx 1.3\times{}\rtwo{} \approx
1.9\times{}\rfive{}.
\end{equation}
Using an approximate mass density profile of $\rhom \propto
\textrm{r}^{-2.5}$ in these outer cluster regions, we assume that the
total mass scales roughly as M $\propto \textrm{r}^{0.5}$, so that
\begin{equation}
\Mvir{} \approx 1.14\times{}\Mtwo{} \approx 1.38 \times{}\Mfive{}.
\end{equation}

Recent X-ray and SZ measurements have begun to illuminate the hot
intracluster plasma, allowing detailed study of these large reservoirs
of baryons in groups and clusters. X-ray surface brightness
observations, in particular, are able to directly measure the gas
density profile and therefore retrieve the mass. However, because
Brehmsstrahlung emission declines with the square of the gas density,
accurate X-ray measurements have typically been limited to the inner
regions of groups and clusters. Studies of this kind in clusters
include \citet[][with \textit{Chandra}]{Vikhlinin2006} and
\citet[][with \textit{XMM-Newton}]{Arnaud2007}, while \citet{Sun2009}
made similar measurements of the hot gas in groups using
\textit{Chandra}. Measuring the gas mass fraction (the internal gas
mass divided by the total mass) only out to \rfive{}, these
observations typically found $\fg{}(<\rfive{}) \approx 6\dash12\%$,
well below the cosmic baryon fraction, although with large variance
between clusters. Importantly, there is a clear trend towards lower
gas content in lower-mass groups, which show gas fractions below
10$\%{}$ at \rfive{}. The stellar fraction in clusters has been
observed to increase in smaller groups \citep{Giodini2009,
  Bahcall2014}. However, stars make up only a few percent of the total
mass even in groups, and represent $<2\%$ of the total mass budget
within \rfive{} in rich clusters. Combining the ICM gas and stellar
mass within \rfive{}, there is not enough baryonic matter to make up
the total baryon content expected from the cosmic fraction.

These discrepancies have been termed the ``Missing Baryon Problem'',
(Figure \ref{fig:Giodini_Missing}) because observations have not been
able to account for the expected abundance of baryons in galaxy
clusters. The problem is more severe in smaller clusters, likely
because baryons are less strongly bound to their shallow gravitational
wells. Many attempts have been made to find a solution to this problem
through theoretical and computational means. Possibly the most
predominant such explanation is that additional non-gravitational
energy is injected into the cluster, such as through shocks
\citep{Takizawa1998}, preheating \citep{Bialek2001}, or a number of
feedback mechanisms due to star formation or AGN activity
\citep{Metzler1994, McCarthy2007, Bode2009}. The net result of these
additional energy sources is to extend the distribution of hot gas in
the ICM, pushing more gas into the outskirts of the clusters while
leaving the central regions deficient in gas. These theories predict
that more sensitive analysis of the outskirts of groups and clusters
($\sim$\rvir{}) should recover the missing gas mass left unaccounted
for by observations of only the inner cluster
regions. \citet{Rasheed2011} used extrapolations of observed gas
density profiles to \rvir{}, showing that the missing gas mass is
indeed located in the outskirts of groups and clusters. Alternative
theories predict that the missing baryons could be residing in
additional phases, such as a cool diffuse gas phase, that have yet to
be identified observationally \citep{Afshordi2007, Bonamente2005}.
\begin{figure*}[btp]
\plotonebig{Copied_Figs/Giodini_Missing}
\caption{The problem of missing baryons, relative to mass, in groups
  and clusters. \textit{Upper}: shows the total baryon fraction within
  \rfive{} for groups and clusters binned by mass. The dark grey
  stripe represents the cosmic baryon fraction. All clusters are short
  of baryons, but low-mass systems are particularly
  deficient. \textit{Lower}: the gas and stellar fractions of the mass
  bins. Figure from \citet{Giodini2009}.}
\label{fig:Giodini_Missing}
\end{figure*}    

\afterpage{\clearpage}



\section{The Galaxy Missing Baryon Problem}
\label{sec:Missing.Galaxies}

Galaxies form in dark matter halos of a range of sizes. Galaxies like
the Milky Way and Andromeda (near the high-end of the galaxy mass
range) are commonly referred to as L$^*$ galaxies, and typically
reside in systems with total mass upwards of $10^{12}$ \Msun{}
\citep{Moster2010}. The virial radii of these systems are around 300
kpc \citep[][and refs.~therein]{Werk2014}. The baryonic components
which dominate the energy output in these systems (and are therefore
the easiest to detect through emission) include the stellar disk, the
gas and dust of the interstellar medium (ISM), and a hot, X-ray
emitting halo of gas. Yet the most recent estimates of stellar mass in
L$^*$ galaxies \citep{Behroozi2010} find that the stellar component
makes up only $\approx5\%$ of the total mass, far less than
anticipated if galactic systems also contain the cosmic baryon
fraction. Including the cold ISM gas measured by HI surveys and the
hot X-ray halo \citep[][respectively]{Martin2010, Gupta2012} only
increases the estimated baryon fraction in galaxies to around
$8\%$. Therefore a similar ``Missing Baryon Problem'' is discussed in
galactic halos because observations cannot account for the entire
anticipated baryon mass. 

Several models have been developed which attempt to solve this
``Galaxy Missing Baryon Problem,'' through, for example, gas
escaping galaxies through winds, jets, and outflows. Some theories
predict unseen components in the intergalactic medium (IGM) which act
as further reservoirs for baryons, but have yet to be accurately
measured. These intergalactic components include a highly-photoionized
Ly$\alpha$ forest \citep{Sargent1980, Cen1994} or the warm-hot
intergalactic medium \citep[WHIM][]{Cen1999, Dave1999}. 

One additional reservoir for galactic baryons, however, is a diffuse
region of gas within the galactic dark matter halo itself, known as
the circumgalactic medium \citep[CGM, first suggested
  by][]{Bahcall1969, Bergeron1985, Lanzetta1995}, which may contain a
large mass of baryons. Observationals have begun to probe the
CGM in the last few years through studies of QSO or galactic
sightlines which pass through the CGM of foreground
galaxies \citep{Steidel2010, Prochaska2011, Tumlinson2011,
  Werk2013}. When photons emitted by the background quasar or galaxy
pass through the CGM of the nearby galaxy, the gas (not energetic
enough to be detected in emission) absorbs characteristic spectral
lines from the background spectrum, depending on the chemical makeup
of the gas. The most common lines observed in absorption come from
neutral hydrogen -- hydrogen being the dominant source of baryonic
mass in the universe. However, absorption from more highly-ionized
species, such as \ion{Ca}{ii} \citep{Zhu2013}, \ion{Mg}{ii},
\ion{Si}{ii}, \ion{C}{ii}, and \ion{O}{vi} \citep{Tumlinson2011,
  Werk2014}, which are not significant contributors to baryonic mass,
can be used to characterize the ionization state of the CGM gas.

Knowledge of the ionization state can constrain the total hydrogen
mass, including both neutral \ion{H}{i} gas and the ionized
\ion{H}{ii}. The observed metal-line absorption in QSO sightlines and
the enrichment of the CGM also suggest that feedback effects are
substantial in galactic systems: metals created in the stellar disk
are expelled into the outer regions of the galactic system, or even
lost altogether, by feedback winds of $\gsim 10^2$ \kms{}
\citep{D'Odorico1991, Chen2010, Oppenheimer2012, Booth2012}. New
observations by the COS-Halos project show that the galaxy CGM indeed
contains a significant amount of extended gas (out to the virial
radius of 300 kpc) that could dramatically increase the baryon
fraction observed in galactic systems. 

\section{Our Investigation -- Where are the Baryons?}
\label{sec:investigation}

In this thesis, we examine the most up-to-date observations available
that can address the overall distribution of baryons relative to dark
matter. Previous works have focused on the narrow question of the
``Missing Baryon Problems'' described above, and this question is
starting to be answered. Generally, we wish to consider the larger
issues of the overall distribution and history of baryons in dark
matter potentials. This thesis addresses the questions: where are the
baryons in the universe, what forms are they in, and how are they correlated with dark matter
halos of all sizes?

Gas density profiles in groups and clusters are observed to be
shallower than the dark matter (NFW) profile, suggesting that there
should be large reservoirs of baryons in the outskirts
\citeeg{Rasheed2011}. We use published SZ and X-ray observations of
cluster outskirts -- along with extrapolations of the inner density
profiles when necessary -- to show that groups and clusters \textit{do
  contain the expected baryon gas fraction} when integrated to
approximately their virial radius. The baryons which originally fell
into groups and clusters have not been ejected from the potential, and
nearly the entire cosmic fraction of baryons remains within the systems.

Galaxies have significantly lower mass than clusters and their binding
potential is lower, making it easier to expel baryons from the galaxy
by feedback effects. Because of this, it is possible that the ``Galaxy
Missing Baryon Problem'' is a consequence of a majority of baryons
being removed from the galaxies altogether. However, the new
observations of the CGM allow powerful constraints of the mass-content
of the outer galactic regions. We study the distribution of baryons in
galaxies using available absorption measurements of the CGM, and find
the galaxy baryon fraction may in fact be consistent with the cosmic
baryon fraction; a large fraction of the baryons simply reside in the
hard-to-detect CGM. Including the gas mass of the CGM with the stellar
disk and ISM, we suggest that nearly the entire cosmic fraction of
baryons has also remained within galactic potentials.

Our investigation is organized as follows. In Chapter \ref{chap:Data},
we present the observations which constrain the baryonic mass within
galaxies, groups, and clusters. We also discuss the ways in which we
determine for the baryon fraction at the virial radius from
extrapolations of measurements in the more central regions. In Chapter
\ref{chap:Results}, we show our results and compare the observed
baryon fraction in galaxies, groups, and clusters to the cosmic baryon
fraction derived from BBN and the CMB. We demonstrate that current
evidence suggests that, averaged over sufficiently large scales of
$\sim\rvir$, all systems contain baryons in abundances consistent with
the cosmic baryon fraction. This is observed in a wide range of system
sizes, from galaxies to groups and clusters and large-scale
structure. In Chapter \ref{chap:Discussion}, we discuss the
observational limitations and potential biases of our findings,
compare them to simulations, and discuss the implications of our
results. In Chapter \ref{chap:Conclusions}, we present our conclusions
on the subject of the baryonic mass distribution, and highlight future
work that can be done to improve our findings.

Throughout this paper, we assume a cosmology of $\omegam = 0.3$,
$\omegal = 0.7$, $\Ho = 70\,\kmsMpc$ ($h = 0.7$). Subscripts typically
signify the value is evaluated within the indicated radius, such as
$\fgfive{} = \fg{}(<\rfive)$. The ``virial'' mass and radius are
defined to be at $\Delta = 100$. We adopt the term ``halo'' to refer
generically to gravitationally bound systems of dark matter and
baryons, which can take on a range of sizes from ``galaxy halos'' to
``group halos'' or ``cluster halos''.

