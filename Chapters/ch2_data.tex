%Senior Thesis Chapter 2
%Ben Cook '14 (bacook@)
%Adviser: Neta Bahcall
\chapter{Observations and Data Analysis}
\label{chap:Data}

\section{Total Mass in Groups and Clusters}
\label{sec:Mass}

The total mass in groups and clusters can be approximated in a number
of ways. The most direct method of measuring the mass profile of a
large halo is through gravitational lensing of the light of background
sources behind the halo. Strong gravitational lensing occurs when the
background object (ex: a high-redshift galaxy) is magnified and
severely warped by the gravitational potential of the lens (foreground
cluster). This allows a very accurate measurement of the mass of the
lens, but occurs only rarely when the background/foreground are in a
particular alignment. More commonly utilized is the technique of weak
gravitational lensing
\citeeg{Umetsu2009,Sheldon2009,VonderLinden2014}, where small
distortions of an immense number of background objects are used to
statistically determine the mass profile of a foreground halo.

Another common method of estimating the mass of groups and clusters is
through the assumption of \textit{hydrostatic equilibrium} (HSE). If
the gas in clusters is in HSE, then the pressure gradient offsets the
gravitational force, allowing the :
\begin{equation}
\frac{\textrm{dP}}{\textrm{dr}} 
\end{equation}

\section{Cluster Gas Mass Fraction}
\label{sec:Gas}
The baryonic content of galaxy groups and clusters is dominated by hot
gas in the intracluster medium (ICM). Until very recently, the most
sensitive X-ray and SZ observations were only able to constrain the
gas mass in the ICM in the inner regions of groups and clusters,
typically to around \rfive{} \citeeg{Vikhlinin2006, Arnaud2007,
  Sun2009}. \todo{Any more?} Because \rvir{} is about twice \rfive{},
these observations only probe the inner $\sim \frac{1}{8}$ of the
virial volume of group and cluster halos. In order to measure the
baryon fraction within groups and clusters, it is essential to
consider the gas within a volume substantially larger than that within
\rfive{}. Here, we describe the relevant observations of groups and
clusters which measure both the ICM and total mass to the outskirts of
the dark matter halo. Because very few telescopes retain the
sensitivity required to measure the gas density in the outskirts of
clusters, we also discuss a method of using observed gas density
profiles to extrapolate observed gas fractions to higher radii.

\subsection{Observations}
\label{sec:Gas.Observations}
\textbf{\citet{Vikhlinin2006}} derived the gas and total mass profiles
of 10 low-redshift (median redshift z$ = 0.06$) relaxed clusters using
long-exposure \textit{Chandra} observations. The clusters have a
median mass $\Mvir = 7.3 \power{14}$ \Msun, and range from $\Mvir =
1.1\power{14}$ -- $1.5 \power{15}$ \Msun. Temperatures range from $\textrm{kT}
= 2$ -- $9$ \keV. The authors measured X-ray temperature and surface
brightness profiles to approximately \rfive{}. They modeled the
surface brightness profile (which is proportional to n$_e$n$_p$) to
recover the gas particle density, $\rho_{gas}($r$)$. The total mass
(\Mfive) was derived by solving the equation of hydrostatic
equilibrium, using the observed density and temperature profiles, and
is well-fit by an NFW profile in most cases. The integrated gas
density and total mass profiles were used to derive the gas fraction
interior to \rfive{}, $\fg(<\rfive)$. This gas fraction ranges widely
from cluster to cluster, from $6\%$ to $14\%$, with median
$11\%$. These observations were also used to derive a useful scaling
relation between \Mfive{} and the X-ray temperature T:
\begin{equation}
\label{eq:M-T}
\Mfive{} = (2.97 \pm 0.15)\times10^{14}~\Msun~h_{70}^{-1}
\left(\frac{\textrm{T}}{5~\textrm{keV}}\right)^{1.58 \pm 0.11}.
\end{equation}

\textbf{\citet{Arnaud2007}} used very similar methods to derive the
gas and total mass profiles of 10 low-redshift (median redshift z$ =
0.09$) relaxed clusters from \textit{XMM-Newton} observations. The
clusters range in mass from $\Mvir = 1.2\power{14}$ -- $1.16 \power{15}$ \Msun,
with a median of $4.2 \power{14}$ \Msun, and temperatures vary from
$\textrm{kT} = 2$ -- $8$ \keV. The total mass also relies on
the assumption of hydrostatic equilibrium, and was extrapolated from
$\sim$r$_{700}$ to \rfive{} using an NFW profile.  \fg{} was derived
out to \rfive{} for these clusters, varying from $5.5\%$ to $16\%$,
with median $11\%$, similar to the \citet{Vikhlinin2006} measurements.

\textbf{\citet{Sun2009}} analyzed the gas fraction in 43 groups from
archival \textit{Chandra} observations. All the groups are at low
redshifts ($z \lsim 0.1$). Of these 43 observations, 11 were sensitive
enough to measure the X-ray surface brightness to \rfive{}, while an
additional 12 measured surface brightness to r$_{1000}$ and were
extrapolated to \rfive{}. The total mass of the 23 best-measured
groups ranges from $\Mvir = 2.0\power{13}$ -- $2.1 \power{14}$ \Msun, with a
median of $1.1 \power{14}$ \Msun, and ICM temperatures range from
$\textrm{kT} = 0.7$ -- $2.7 \keV$. The total mass (assuming hydrostatic
equilibrium) and gas mass were calculated using similar principles to
\citet{Vikhlinin2006}, with errors estimated by using 1000 artificial
profiles generated from Monte-Carlo simulations. $\fg(<\rfive)$ for
these 23 groups ranges from $5\%$ -- $11\%$, with a median of $8\%$,
lower than for the more massive clusters of \citet{Vikhlinin2006} and
\citet{Arnaud2007}.

The above three samples were combined in the analysis of
\textbf{\citet{Giodini2009}} (G09), which used all 10 clusters from
\citet{Vikhlinin2006}, all 10 clusters from \citet{Arnaud2007}, and 21
of the 23 best-measured groups from \citet{Sun2009} to study the
group/cluster gas mass fraction over a wide range of halo masses. The
authors bin the 42 groups and clusters logarithmically by mass,
highlighting that lower mass halos have significantly lower gas
fractions. The best-fit trend is:
\begin{equation}
\fg(<\rfive) = (9.3 \pm 0.2)\power{-2}~h_{70}^{-3/2}~
\left(\frac{\Mfive}{2\power{14}\Msun}\right)^{0.21 \pm 0.03}.
\end{equation}
Figure \ref{fig:Giodini_Fgas} shows the distribution of the observed
gas fractions, as a function of halo mass, measured by the three works
above. We will use the G09 bins as 5 independent samples of \fg{} for
different masses. 

\begin{figure*}[hbtp]
\plotonebig{Giodini_Fgas}
\caption{The dependence of $\fg(<\rfive)$ on \Mfive{} ($\sim
0.73\Mvir$), as presented in \citet{Giodini2009}. The light-grey
  points represent individual group/cluster observations from
  \citet{Vikhlinin2006}, \citet{Arnaud2007}, and \citet{Sun2009},
  while the dark points are the average gas fractions, binned
  logarithmically with mass. Lower-mass halos show significantly lower
gas fractions, with $\fg(<\rfive)$ scaling roughly as
$\Mfive^{0.21}$. }
\label{fig:Giodini_Fgas}
\end{figure*}    


\afterpage{\clearpage}

Recent results from the \Planck{} satellite detect the ICM using the
Thermal SZ effect, which measures the integrated line-of-sight gas
pressure. \textbf{\citet{PlanckIntV}} (PC13) derives a stacked pressure
profile for 62 massive clusters which have archival observations by
\XMM. The cluster sample \citep[detailed in][]{PlanckEarlyXI} includes
clusters of mass $\Mvir = 3.3\power{14}$ -- $2.7 \power{15}$ \Msun,
with median mass approximately $\Mvir = 8.70 \power{14}$ \Msun. X-ray
temperatures range from $\textrm{kT} = 3.4$ -- $13 \keV$. Total mass
(\Mfive) was derived from a scaling relation with the quantity
$\textrm{Y}_X = \textrm{M}_{gas}\textrm{T}_X$, an easily-observable
quantity that has been found to be a good mass proxy. The scaling
relation in question \citep{Arnaud2010} was calibrated against X-ray
derived hydrostatic masses, and so the total mass profile of the
stacked \Planck{} clusters assumes hydrostatic equilibrium. The total
mass beyond \rfive{} was calculated assuming an NFW profile. The
stacked pressure profile is measured to unprecedented scales ($3\rfive
\approx 1.6\rvir$), although the X-ray temperature profile measured by
\XMM{} only extends to \rfive{}, so the authors extrapolated the
observed temperature profile to $3\rfive{}$ to match the pressure
observations.

Assuming the ICM acts as an ideal gas ($\textrm{P} \propto
\textrm{n}_\textrm{e}\textrm{kT}$), the authors inverted the pressure
and temperature profiles to retrieve the gas density profile and
derive $\fg{}(\textrm{r})$ out to 3\rfive{}\footnote{The authors also
  derive the gas-fraction assuming a conservative case in which the
  ICM is isothermal beyond \rfive{}, resulting in lower \fg{}.}. The
reconstruction of the temperature profile was initially flawed, and
the correct gas fraction profile was given in a corrigendum,
\citet{PlanckIntVb}. \fg{} increases from \rfive{} to \rvir{}
\citep[as predicted by][see \ref{sec:Gas.Extrapolation}]{Rasheed2011},
reaching a peak of $\approx 15\pm2\%$ at $1.6\rvir{}$.

\textbf{\citet{Eckert2013b}} (E13) combined the stacked pressure profile
from \citet{PlanckIntV} with a stacked X-ray surface-brightness
profile that directly constrains the gas density to \rtwo{}. The X-ray
observations were performed with \Rosat{}, on a sample of 31 clusters
($z\lsim0.2$) of temperatures $\textrm{kT} = 2.5$ -- $9$ \keV, with
median $\textrm{kT} = 6.5$ \keV. The cluster masses range from $\Mvir
\approx 1.4 \power{14}$ to $1.0 \power{15} \Msun$, with median $\Mvir =
6.0\power{14} \Msun$\footnote{The authors do not give the masses of
  the clusters, so these values are taken from the M-T relation of
  \citet{Vikhlinin2006} (equation \ref{eq:M-T}).}. The \Planck{}
pressure profile, combined with the gas density profile, is used to
directly calculate the total mass, assuming hydrostatic
equilibrium. This is different from the method used by
\citet{PlanckIntV}, which assumed an NFW profile. However, both
estimates rely on the assumption of hydrostatic equilibrium either
explicitly or implicitly through calibration of the
$\textrm{Y}_\textrm{X}-\Mfive$ relation. 18
clusters are in common between the \Rosat{} and \Planck{} samples, and
the authors separate them into cool-core (CC, 6 clusters) and
non-cool core (NCC, 12 clusters) categories. The gas fraction profile
is calculated separately for the two categories, and the authors find
that NCC clusters have significantly higher gas fractions within
\rtwo{} ($0.169 \pm 0.010$) than relaxed, CC clusters do ($0.134\pm
0.011$), suggesting that the irregular, non-spherical morphologies of
the disturbed clusters may bias the gas fractions high. They also find
that \fg{} increases from \rfive{} to \rtwo{} ($\fg(<\rfive) \approx
0.12$ for CC clusters). 

\textbf{\citet{Umetsu2009}} (U09) observed the ICM of four very massive
($\Mvir \gsim 1\power{15}$ \Msun, $\textrm{kT} \approx 9$ -- $10$ \keV)
clusters using Thermal SZ measurements with the \textit{AMiBA} CMB
telescope. After deriving pressure profiles from the SZ effect, the
authors calculated the gas density profile using archival X-ray
temperature measurements and a theoretical temperature-profile
\citep{Komatsu2001}. The authors use \textit{Subaru} observations to
derive the cluster masses with weak-lensing analysis. The average gas
fraction is calculated to the limit of the SZ observations, \rtwo, and
is found to be $\fg(<\rfive) = 0.126 \pm 0.019 \pm 0.016$ within
\rfive{} and $\fg(<\rtwo) = 0.133 \pm 0.020 \pm 0.018$ within
\rtwo{}. The two uncertainties on each fraction are the statistical
error and cluster-to-cluster standard deviation, respectively. These
observations also find that \fg{} increases with radius beyond
\rfive{}, in agreement with \citet{PlanckIntV} and
\citet{Eckert2013b}. We emphasize that the total mass for these
clusters are \textit{not} dependent on the assumption of hydrostatic
equilibrium (because they were derived using weak-lensing
measurements), so comparisons of \fg{} from this sample to the same
value for samples which assume HSE can put constraints on the validity
of the HSE assumption.

Table \ref{tab:F_gas} lists the samples described above with the most
important characteristics of each sample, including median mass
($\Mvir$), whether that mass is derived assuming HSE, and the gas
fraction at all directly-observed radii.

\begin{table}[hbt]
\caption{Observed Gas Fraction Measurements in Groups/Clusters}
\scriptsize
\begin{tabular}{llccccc}
\hline \hline\\
\footnotesize \textbf{Reference} & \footnotesize \textbf{\#{} Clusters} & \footnotesize \textbf{$\left<\Mvir\right>$ (\Msun)} & \footnotesize \textbf{HSE?} & \footnotesize \textbf{$\fg{}_{,500}$} & \footnotesize \textbf{$\fg{}_{,200}$} & \footnotesize \textbf{$\fg{}_{,vir}$} \\
\footnotesize (1) & \footnotesize (2)& \footnotesize (3)& \footnotesize (4)& \footnotesize (5)& \footnotesize (6)& \footnotesize (7) \\\\
\hline
G09 Bin 1 & 2 & $2.9\power{13}$ & $\surd$ & $0.074 \pm 0.028$ & & \\
\phantom{G09} Bin 2 & 7  & $7.0\power{13}$ & $\surd$ & $0.068 \pm 0.005$ & & \\
\phantom{G09} Bin 3 & 17 & $1.7\power{14}$ & $\surd$ & $0.080 \pm 0.003$ & & \\
\phantom{G09} Bin 4 & 5 & $4.1\power{14}$ & $\surd$ & $0.103 \pm 0.008$ & & \\
\phantom{G09} Bin 5 & 10 & $9.8\power{14}$ & $\surd$ & $0.123 \pm 0.007$ & & \\
PC13 & 62 & $8.7\power{14}$ & $\surd$ & $0.125\pm0.005$ & $0.137\pm0.003$ & $0.145\pm0.01$\\
E13 - CC & 6 & $5.9\power{14}$ & $\surd$ & $0.115\pm0.010$ & $0.134\pm0.011$ & \\
E13 - NCC & 12 & $5.9\power{14}$ & $\surd$ & $0.128\pm0.010$ &
$0.169\pm0.010$ &\\
U09 & 4 & $7.6\power{14}$ & & $0.126\pm0.025$ & $0.133\pm0.027$ & \\
\hline
\end{tabular}
\caption*{\small{(1) G09, PC13, E13, and U09 stand for
    \citet{Giodini2009}, \citet{PlanckIntV}, \citet{Eckert2013b}, and
    \citet{Umetsu2009}, respectively. CC (NCC) represents the sample
    of cool-core (non-cool core) clusters.\\ (2) The number of
    clusters in each sample.  (3) The median virial mass of the
    clusters. \\ (4) $\surd$ marks that the total mass assumes
    hydrostatic equilibrium.\\ (5) $\fg(<\rfive)$ (6) $\fg(<\rtwo)$
    (7) $\fg(<\rvir)$\\ }}
\label{tab:F_gas}
\end{table}


\subsection{Extrapolation of Gas Density Profiles}
\label{sec:Gas.Extrapolation}
As seen above, very few observations retain the necessary sensitivity to
measure the gas density all the way to \rvir{}. Therefore, to
constrain the gas fraction within the entire halo, we can extrapolate
the observed gas mass profile (at \rtwo{} or \rfive{}) to higher
radius by assuming a power-law profile:
\begin{equation}
\rho_{gas}(\textrm{r}) \propto \textrm{r}^{-\alpha_g}, \nonumber
\end{equation}
where $\alpha{}_g$ is the slope of the gas density profile, which can in
general change as radius increases. The total
matter density can be similarly modeled,
\begin{equation}
\rho_{m}(\textrm{r}) \propto \textrm(r)^{-\alpha_{m}}, \nonumber
\end{equation}
with $\alpha{}_m$ the slope of the total mass density profile. At
large radii, the full equation for the gas fraction simplifies to approximately:
\begin{align}
\fg(<\textrm{r}) = \frac{\textrm{M}_{gas}}{\textrm{M}_{tot}} &=
\frac{\int_0^r 4\pi{}r'^2dr'\rho_{gas}(\textrm{r}')}{\int_0^r
  4\pi{}r'^2dr'\rho_{m}(\textrm{r}')} \nonumber \\ & \approx
\frac{\rho_{gas}(\textrm{r})}{\rho_{tot}(\textrm{r})} \nonumber\\ &\propto \textrm{r}^{\alpha_{m} - \alpha_{tot}}. \nonumber
\end{align}
Therefore, the gas fraction can be extrapolated to larger radii using
the difference in slopes between the gas density and total mass
density profiles.

\textbf{\citet{Rasheed2011}} (R11) used this approach to extrapolate
the gas fraction of the G09 cluster samples to \rvir{}. X-ray and SZ
observations show that the gas density decreases more slowly with
radius than the total mass density ($\alpha_{m} > \alpha_{gas}$),
suggesting that the gas fraction should increase when the cluster
outskirts are considered. The authors hoped to place constraints on
the amount of ``missing baryons'' within the virial volume of
clusters.

R11 used a large survey of the literature to recover X-ray
measurements which constrain the gas density slope out to
\rfive{}. These measurements include observations with \Rosat{},
\Chandra, \XMM, and \textit{Suzaku}, and cover a wide range of cluster
masses and temperatures. Averaging over the many observations, the
authors find that the gas density slope at \rfive{} steepens with more
massive clusters, with $\alpha_{gas}$ ranging from $\approx 1.8 \pm
0.2$ for poor clusters (G09 bin 2, $\Mvir \approx 7\power{13}\Msun$)
to $\approx 2.3 \pm 0.02$ for the most massive G09 bin ($\Mvir
\approx9.8\power{14}\Msun$). 

Compared to the gas density profile, the total density (NFW) profile
is significantly steeper in the outer regions of the halo. In the mass
range of the G09 groups and clusters, the NFW profile has a slope of
$\alpha_m = 2.6$ between \rfive{} and \rtwo{}, and steepens to
$\alpha_m = 2.7$ in the region \rtwo{} to \rvir{}. Therefore, R11
predicted that the gas fraction rises significantly above
\rfive{}. Because $\alpha_{gas}$ increases with cluster mass, the gas
fraction is predicted to rise more quickly with radius for groups and
poor clusters ($\fg\propto \textrm{r}^{0.8}$ for G09 bin 2) than for
rich clusters ($\fg \propto \textrm{r}^{0.3}$ for G09 bin 5). For
these two bins, this model predicts increases in \fg{} by a factor of
roughly $1.6$ and $1.2$, respectively, from \rfive{} to \rvir{}. This
offers an explanation for why the missing baryon problem is more
severe in lower-mass clusters: the shallower gas profile implies the
ICM is spread farther out in lower-mass halos than in very massive
ones.

We adopt R11's extrapolation model in order to approximate the gas
fraction at high radius in the samples which do not measure \fg{} to
\rvir{} (all except PC13). $\alpha_{gas}$ for each sample in Table
\ref{tab:F_gas} is taken from the temperature-slope relation in R11,
we do not extrapolate the gas profiles for any individual cluster
sample with coverage beyond \rfive{}. We assume $\alpha_m$ as
above for the NFW profile. We extrapolate \fg{} from the maximum
observed radius, r$_a$, to a larger radius r$_b$ using:
\begin{equation}
\fg(<\textrm{r}_b) =
\fg(<\textrm{r}_a)\left(\frac{\textrm{r}_b}{\textrm{r}_a}\right)^{\alpha_m - \alpha_{gas}}.
\end{equation}
 For example, extrapolating the gas fraction of G09's bin 5 from \rfive{} to \rtwo{}:
\begin{align}
\fg{}(<\rtwo) &=
\fg{}(<\rfive)\left(\frac{\rtwo}{\rfive}\right)^{\alpha_m -
  \alpha_{gas}} \nonumber\\
&\approx .103~(1.45)^{2.6 - 2.3} \nonumber \\
&\approx .115 \nonumber
\end{align}

To calculate the uncertainty on the extrapolated gas fraction, we
propagate the errors in $\fg(<\textrm{r}_a)$ (or $\fg{}_{,a}$) and in
$\alpha_{gas}$, assuming no significant uncertainty exists in
$\alpha_m$ or r$_b$/r$_a$. The fractional errors add in quadrature.
\begin{align}
\frac{\Delta\fg{}_{,b}}{\fg{}_{,b}} &=
\sqrt{\left(\frac{\Delta\fg{}_{,a}}{\fg{}_{,a}}\right)^2 +
  \left(\frac{\Delta\left(\textrm{r}_b/\textrm{r}_a\right)^{\alpha_m-\alpha_{gas}}}{\left(\textrm{r}_b/\textrm{r}_a\right)^{\alpha_m-\alpha_{gas}}}\right)^2}\nonumber\\
\intertext{The uncertainty in the right term is}
\Delta\left(\textrm{r}_b/\textrm{r}_a\right)^{\alpha_m-\alpha_{gas}}
& =
\left(\textrm{r}_b/\textrm{r}_a\right)^{\alpha_m-\alpha_{gas}}\ln{\left(\textrm{r}_b/\textrm{r}_a\right)}\Delta\alpha_{gas}\,,\nonumber\\
\intertext{yielding the final result:}
\frac{\Delta\fg{}_{,b}}{\fg{}_{,b}} &=
\sqrt{\left(\frac{\Delta\fg{}_{,a}}{\fg{}_{,a}}\right)^2 +
  \left(\ln{\left(\textrm{r}_b/\textrm{r}_a\right)}\Delta\alpha_{gas}\right)^2}
\end{align}

The gas density profile is expected to steepen at very large radii,
such that it eventually matches the NFW profile \citeeg{Umetsu2009},
which translates to the gas fraction asymptotically approaching a constant
value. At large enough radius, extrapolation of the gas fraction as
described above will, therefore, become invalid, as $\alpha_{gas}$
will not remain fixed. The range at which the gas density steepens
significantly is not known, however, as observational data does not
currently constrain $\alpha_{gas}$ far beyond \rfive. R11's assumption
that this slope remains constant to \rvir{} is therefore a
questionable one, but no simple alternatives exist. We also assume
$\alpha_{gas}$ remains constant to \rvir{} (and slightly beyond), and
emphasize that the our calculation of the gas fraction will be biased
high if the gas density slope steepens significantly beyond
\rfive{}. In Chapter \ref{chap:Results}, we discuss how our results
may be able to constrain the evolution of this density slope. 

\section{Cluster Stellar Mass Fraction}
\label{sec:Stellar}

\section{Galaxy Mass Fractions}
\label{sec:Galaxy}

\subsection{The Circumgalactic Medium}
\label{sec:Galaxy.CGM}

\subsection{Constraints on Other Galactic Components}
\label{sec:Galaxy.Components}

\subsection{Estimate of Galactic-Halo Baryon Fraction}
\label{sec:Galaxy.Fraction}
