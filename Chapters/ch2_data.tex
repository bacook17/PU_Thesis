%Senior Thesis Chapter 2
%Ben Cook '14 (bacook@)
%Adviser: Neta Bahcall
\chapter{Observations and Data Analysis}
\label{chap:Data}

\section{Total Mass in Groups and Clusters}
\label{sec:Mass}


\section{Cluster Gas Mass Fraction}
\label{sec:Gas}
The baryonic content of galaxy groups and clusters is dominated by hot
gas in the intracluster medium (ICM). Until very recently, the most
sensitive X-ray and SZ observations were only able to constrain the
gas mass in the ICM in the inner regions of groups and clusters,
typically to around \rfive{} \citeeg{Vikhlinin2006, Arnaud2007,
  Sun2009}. \todo{Any more?} Because \rvir{} is about twice \rfive{},
these observations only probe the inner $\sim \frac{1}{8}$ of the
virial volume of group and cluster halos. In order to measure the
baryon fraction within groups and clusters, it is essential to
consider the gas within a volume substantially larger than that within
\rfive{}. Here, we describe the relevant observations of groups and
clusters which measure both the ICM and total mass to the outskirts of
the dark matter halo. Because very few telescopes retain the
sensitivity required to measure the gas density in the outskirts of
clusters, we also discuss a method of using observed gas density
profiles to extrapolate observed gas fractions to higher radii.

\subsection{Observations}
\label{sec:Gas.Observations}
\textbf{\citet{Vikhlinin2006}} derived the gas and total mass profiles
of 10 low-redshift (median redshift z$ = 0.06$) relaxed clusters using
long-exposure \textit{Chandra} observations. The clusters have a
median mass $\Mvir = 7.3 \power{14}$ \Msun, and range from $\Mvir =
1.1\power{14}$ -- $1.5 \power{15}$ \Msun. Temperatures range from $\textrm{kT}
= 2$ -- $9$ \keV. The authors measured X-ray temperature and surface
brightness profiles to approximately \rfive{}. They modeled the
surface brightness profile (which is proportional to n$_e$n$_p$) to
recover the gas particle density, $\rho_{gas}($r$)$. The total mass
(\Mfive) was derived by solving the equation of hydrostatic
equilibrium, using the observed density and temperature profiles, and
is well-fit by an NFW profile in most cases. The integrated gas
density and total mass profiles were used to derive the gas fraction
interior to \rfive{}, $\fg(<\rfive)$. This gas fraction ranges widely
from cluster to cluster, from $6\%$ to $14\%$, with median
$11\%$. These observations were also used to derive a useful scaling
relation between \Mfive{} and the X-ray temperature T:
\begin{equation}
\label{eq:M-T}
\Mfive{} = (2.97 \pm 0.15)\times10^{14}~\Msun~h_{70}^{-1}
\left(\frac{\textrm{T}}{5~\textrm{keV}}\right)^{1.58 \pm 0.11}.
\end{equation}

\textbf{\citet{Arnaud2007}} used very similar methods to derive the
gas and total mass profiles of 10 low-redshift (median redshift z$ =
0.09$) relaxed clusters from \textit{XMM-Newton} observations. The
clusters range in mass from $\Mvir = 1.2\power{14}$ -- $1.16 \power{15}$ \Msun,
with a median of $4.2 \power{14}$ \Msun, and temperatures vary from
$\textrm{kT} = 2$ -- $8$ \keV. The total mass also relies on
the assumption of hydrostatic equilibrium, and was extrapolated from
$\sim$r$_{700}$ to \rfive{} using an NFW profile.  \fg{} was derived
out to \rfive{} for these clusters, varying from $5.5\%$ to $16\%$,
with median $11\%$, similar to the \citet{Vikhlinin2006} measurements.

\textbf{\citet{Sun2009}} analyzed the gas fraction in 43 groups from
archival \textit{Chandra} observations. All the groups are at low
redshifts ($z \lsim 0.1$). Of these 43 observations, 11 were sensitive
enough to measure the X-ray surface brightness to \rfive{}, while an
additional 12 measured surface brightness to r$_{1000}$ and were
extrapolated to \rfive{}. The total mass of the 23 best-measured
groups ranges from $\Mvir = 2.0\power{13}$ -- $2.1 \power{14}$ \Msun, with a
median of $1.1 \power{14}$ \Msun, and ICM temperatures range from
$\textrm{kT} = 0.7$ -- $2.7 \keV$. The total mass (assuming hydrostatic
equilibrium) and gas mass were calculated using similar principles to
\citet{Vikhlinin2006}, with errors estimated by using 1000 artificial
profiles generated from Monte-Carlo simulations. $\fg(<\rfive)$ for
these 23 groups ranges from $5\%$ -- $11\%$, with a median of $8\%$,
lower than for the more massive clusters of \citet{Vikhlinin2006} and
\citet{Arnaud2007}.

The above three samples were combined in the analysis of
\textbf{\citet{Giodini2009}} (G09), which used all 10 clusters from
\citet{Vikhlinin2006}, all 10 clusters from \citet{Arnaud2007}, and 21
of the 23 best-measured groups from \citet{Sun2009} to study the
group/cluster gas mass fraction over a wide range of halo masses. The
authors bin the 42 groups and clusters logarithmically by mass,
highlighting that lower mass halos have significantly lower gas
fractions. The best-fit trend is:
\begin{equation}
\fg(<\rfive) = (9.3 \pm 0.2)\power{-2}~h_{70}^{-3/2}~
\left(\frac{\Mfive}{2\power{14}\Msun}\right)^{0.21 \pm 0.03}.
\end{equation}
Figure \ref{fig:Giodini_Fgas} shows the distribution of the observed
gas fractions, as a function of halo mass, measured by the three works
above.
\begin{figure*}[hbt]
\plotonebig{Giodini_Fgas}
\caption{The dependence of $\fg(<\rfive)$ on \Mfive{} ($\sim
0.73\Mvir$), as presented in \citet{Giodini2009}. The light-grey
  points represent individual group/cluster observations from
  \citet{Vikhlinin2006}, \citet{Arnaud2007}, and \citet{Sun2009},
  while the dark points are the average gas fractions, binned
  logarithmically with mass. Lower-mass halos show significantly lower
gas fractions, with $\fg(<\rfive)$ scaling roughly as
$\Mfive^{0.21}$. }
\label{fig:Giodini_Fgas}
\end{figure*}    



Recent results from the \Planck{} satellite detect the ICM using the
Thermal SZ effect, which measures the integrated line-of-sight gas
pressure. \textbf{\citet{PlanckIntV}} derives a stacked pressure
profile for 62 massive clusters which have archival observations by
\XMM. The cluster sample \citep[detailed in][]{PlanckEarlyXI} includes
clusters of mass $\Mvir = 3.3\power{14}$ -- $2.7 \power{15}$ \Msun,
with median mass approximately $\Mvir = 8.70 \power{14}$ \Msun. X-ray
temperatures range from $\textrm{kT} = 3.4$ -- $13 \keV$. Total mass
(\Mfive) was derived from a scaling relation with the quantity
$\textrm{Y}_X = \textrm{M}_{gas}\textrm{T}_X$, an easily-observable
quantity that has been found to be a good mass proxy. The scaling
relation in question \citep{Arnaud2010} was calibrated against X-ray
derived hydrostatic masses, and so the total mass profile of the
stacked \Planck{} clusters assumes hydrostatic equilibrium. The total
mass beyond \rfive{} was calculated assuming an NFW profile. The
stacked pressure profile is measured to unprecedented scales ($3\rfive
\sim 1.6\rvir$), although the X-ray temperature profile measured by
\XMM{} only extends to \rfive{}, so the authors extrapolated the
observed temperature profile to $3\rfive{}$ to match the pressure
observations.

Assuming the ICM acts as an ideal gas ($\textrm{P} \propto
\textrm{n}_\textrm{e}\textrm{kT}$), the authors inverted the pressure
and temperature profiles to retrieve the gas density profile and
derive $\fg{}(\textrm{r})$ out to 3\rfive{}\footnote{The authors also
  derive the gas-fraction assuming a conservative case in which the
  ICM is isothermal beyond \rfive{}, resulting in lower \fg{}.}. The
reconstruction of the temperature profile was initially flawed, and
the correct gas fraction profile was given in a corrigendum,
\citet{PlanckIntVb}. \fg{} increases from \rfive{} to \rvir{}
\citep[as predicted by][see \ref{sec:Gas.Extrapolation}]{Rasheed2011},
reaching a peak of $\sim 15\pm2\%$ at $1.6\rvir{}$.

\textbf{\citet{Eckert2013b}} combined the stacked pressure profile
from \citet{PlanckIntV} with a stacked X-ray surface-brightness
profile. The X-ray observations were performed with \Rosat{}, on a
sample of 31 clusters ($z\lsim0.2$) of temperatures $\textrm{kT} =
2.5$ -- $9 \keV$, with median $\textrm{kT} = 6.5 \keV$. The cluster
masses range from $\Mvir \sim 1.4 \power{14}$ to $1.0 \power{15}
\Msun$, with median $\Mvir = 6.0\power{14} \Msun$\footnote{The authors
  do not give the masses of the clusters, so these values are taken
  from the M-T relation of \citet{Vikhlinin2006} (equation
  \ref{eq:M-T}).}. The \Rosat{} observations are used to derive the
gas density profile to \rtwo{}. The \Planck{} pressure profile,
combined with the gas density profile, is used to directly calculate
the total mass, assuming hydrostatic equilibrium. This is different
from the method used by \citet{PlanckIntV}, which assumed an NFW
profile, although both estimates rely on the assumption of
hydrostatic equilibrium. The authors separate their 

\subsection{Extrapolation of Gas Density Profiles}
\label{sec:Gas.Extrapolation}
Very few observations retain the necessary sensitivity to
measure the gas density all the way to \rvir{}, so in many cases we
extrapolate the observed 

\textbf{\citet{Rasheed2011}} 

\section{Cluster Stellar Mass Fraction}
\label{sec:Stellar}

\section{Galaxy Mass Fractions}
\label{sec:Galaxy}

\subsection{The Circumgalactic Medium}
\label{sec:Galaxy.CGM}

\subsection{Constraints on Other Galactic Components}
\label{sec:Galaxy.Components}

\subsection{Estimate of Galactic-Halo Baryon Fraction}
\label{sec:Galaxy.Fraction}
