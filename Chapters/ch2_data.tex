%Senior Thesis Chapter 2
%Ben Cook '14 (bacook@)
%Adviser: Neta Bahcall
\chapter{Observations and Data Analysis}
\label{chap:Data}

\section{Total Mass in Groups and Clusters}
\label{sec:Mass}

The total mass in groups and clusters can be measured in a number of
ways. The most direct method of determining the mass profile of a
large halo is through gravitational lensing of the light of background
sources behind the halo. Strong gravitational lensing occurs when the
background object (e.g., a high-redshift galaxy) is magnified and
severely warped by the gravitational potential of the lens (foreground
cluster). This allows a very accurate measurement of the mass of the
lens within the lensed arc radius, but occurs only rarely when the
background/foreground are aligned, and it provides the mass only
within the inner strongly lensed region.  More commonly utilized to
obtain the mass to larger radius is the technique of weak
gravitational lensing
\citeeg{Umetsu2009,Sheldon2009a,VonderLinden2014}, where small
distortions of an immense number of background objects are used to
statistically determine the mass profile of a foreground halo.

Another common method of estimating the mass of groups and clusters is
through the assumption of \textit{hydrostatic equilibrium} (HSE). If
the gas in clusters is in hydrostatic equilibrium with the cluster
potential, then the pressure gradient offsets the gravitational force:
\begin{align}
\frac{dP(\radius)}{d\radius} &=
-\frac{G~\Mass(<\radius)\rho_{gas}(\radius)}{\radius^2}
\end{align}
Therefore, assuming the ICM is in hydrostatic equilibrium with the
cluster potential, the total mass profile can be reconstructed from
the observed gas density and pressure profiles:
\begin{align}
 \Mass(<\radius) &=
 -\frac{\radius^2}{G~\rho_{gas}(\radius)}\frac{dP(<\radius)}{d\radius}\;.
 \intertext{Alternatively, the total mass can be obtained from the
   density and temperature profiles, assuming the ICM behaves as an
   ideal gas, $P(\radius)=n(\radius)kT(\radius)$:} 
\Mass(<\radius) &=
 -\frac{k~T(\radius)~\radius}{G}\left(\frac{d\log{\rho_{gas}(\radius)}}{d\radius}
 + \frac{d\log{T(\radius)}}{d\radius}\right)\;.
\end{align} 
Gas density is typically measured using X-ray observations, as the gas
density is easily determined from the X-ray surface
brightness. Temperature is determined from X-ray spectroscopy, and
pressure is measured directly through the thermal SZ effect.

Whether the total mass derived through the assumption of hydrostatic
equilibrium (or the ``hydrostatic mass'') is biased relative to the
true mass is highly debated, as sources of non-thermal pressure
(including cosmic rays, merger-induced shocks, and AGN feedback) can
invalidate the assumption of hydrostatic equilibrium. Comparing
hydrostatic masses to masses derived through weak lensing analysis can
help constrain the bias inherent in the HSE assumption. In Section
\ref{sec:Limitations.HSE}, we discuss observational and theoretical
constraints on the hydrostatic mass bias.

The total matter density distribution is often modeled by the NFW
profile, an analytic equation first proposed by \citet{Navarro1996}
to describe the ``universal density profile'' of simulated dark matter
halos, regardless of size. The NFW profile has the form:
\begin{equation}
\rho_m(\radius) = \frac{\delta_c\rhocrit}{(\textrm{r/r}_s)(1 + \textrm{r/r}_s)^2}\;.
\end{equation}
r$_s$ is a characteristic radius representing the central core of the
dark matter halo, $\rhocrit$ is the critical matter density, and
$\delta_c$ is a normalization constant which sets the characteristic
overdensity of the cluster. Halos which fit the NFW profile are
self-similar, in that $\radius_s$ and $\delta_c$ are determined
uniquely by the total halo mass. When $\radius \lsim \radius_s$, the
density profile decreases slowly ($\rho_m \propto \radius^{-1}$),
while at $\radius \gg \radius_s$ the profile steepens significantly ($\rho_m
\propto \radius^{-3}$). Observations of group and cluster halos
consistently find that the total mass profile follows the NFW profile
well \citeeg{Vikhlinin2006,Mandelbaum2008,Sheldon2009a,Umetsu2009}.

Several ``mass proxies'' have been used to estimate the total mass of
clusters. Mass proxies are typically easily-observed quantities that
are found to correlate strongly with the total cluster mass. Examples
include the X-ray temperature ($kT_X\approx1\dash10~\keV{}$ for groups
and clusters), the richness (number of bright galaxies observed within
the cluster), the X-ray luminosity, and $Y_X = \Mass_{gas}T_X$. Using
mass proxies allows observers to place loose constraints on the mass
of a cluster without requiring deep observations to recover the true
gas or mass density profiles. The assumption of hydrostatic
equilibrium can also affect the determination of mass through this
method, as many Mass-Proxy relations are calibrated against
hydrostatic masses of clusters \citeeg{Arnaud2007,Arnaud2010}.

\section{Cluster Gas Mass Fraction}
\label{sec:Gas}
The baryonic content of galaxy groups and clusters is dominated by hot
plasma in the intracluster medium (ICM). Until very recently, the most
sensitive X-ray and SZ observations were only able to constrain the
gas mass in the ICM in the inner regions of groups and clusters,
typically to around \rfive{} \citeeg{Vikhlinin2006, Arnaud2007,
  Sun2009}. Because \rvir{} is about twice \rfive{}, these
observations only probe the inner $\approx \sfrac{1}{8}$ of the virial
volume of groups and clusters. In order to measure the baryon
distribution within the entire halo of groups and clusters, it is
important to consider the gas within a volume substantially larger
than that within \rfive{}. Here, we describe the relevant observations
of groups and clusters which measure both the ICM and total mass of
the cluster. Because very few observations retain the sensitivity
required to measure the gas density in the outskirts of clusters, we
also discuss a method of using measured gas density profiles to
extrapolate observed gas fractions to larger radii.

\subsection{Observations}
\label{sec:Gas.Observations}
\textbf{\citet{Vikhlinin2006}} derived the gas and total mass profiles
of 10 low-redshift (median redshift $z = 0.06$) relaxed clusters using
long-exposure \textit{Chandra} observations. The clusters have a
median mass $\Mvir = 7.3 \power{14}~\Msun$, and range from $\Mvir =
1.1\power{14}\dash1.5 \power{15}~\Msun$. Temperatures range from
$kT = 2\dash9~\keV$. The authors measured X-ray temperature
and surface brightness profiles to approximately \rfive{}. They
modeled the surface brightness profile (which is proportional to
$n_en_p$) to recover the gas particle density,
$\rho_{gas}($r$)$. The total mass (\Mfive) was measured by solving the
equation of hydrostatic equilibrium, using the observed density and
temperature profiles, and is well-fit by an NFW profile in most
cases. The integrated gas density and total mass profiles were used to
derive the gas fraction interior to \rfive{}, $\fgfive$. This gas
fraction ranges widely from cluster to cluster, from $6\%$ to $14\%$,
with median $11\%$. These observations were also used to derive a
useful scaling relation between \Mfive{} and the X-ray temperature $T$:
\begin{equation}
\label{eq:M-T}
\Mfive{} = (2.97 \pm 0.15)\power{14}~\Msun~h_{70}^{-1}
\left(\frac{T}{5~\textrm{keV}}\right)^{1.58 \pm 0.11}.
\end{equation}

\textbf{\citet{Arnaud2007}} used very similar methods to derive the
gas and total mass profiles of 10 low-redshift (median $z = 0.09$)
relaxed clusters from \textit{XMM-Newton} observations. The clusters
range in mass from $\Mvir = 1.2\power{14}\dash1.16 \power{15}~\Msun$,
with a median of $4.2 \power{14}~\Msun$, and temperatures vary from
$kT = 2\dash8~\keV$. The total mass also relies on the
assumption of hydrostatic equilibrium, and was extrapolated from
$\approx$r$_{700}$ to \rfive{} using an NFW profile.  \fg{} was derived
out to \rfive{} for these clusters, varying from $5.5\%$ to $16\%$,
with median $11\%$, similar to the \citet{Vikhlinin2006} measurements.

\textbf{\citet{Sun2009}} analyzed the gas fraction in 43 groups from
archival \textit{Chandra} observations. All the groups are at low
redshifts ($z \lsim 0.1$). Of these 43 observations, 11 were sensitive
enough to measure the X-ray surface brightness to \rfive{}, while an
additional 12 measured surface brightness to r$_{1000}$ and were
extrapolated to \rfive{}. The total mass of the 23 best-measured
groups ranges from $\Mvir = 2.0\power{13}\dash2.1 \power{14}~\Msun$, with a
median of $1.1 \power{14}~\Msun$, and ICM temperatures range from
$kT = 0.7\dash2.7~\keV$. The total mass (assuming hydrostatic
equilibrium) and gas mass were obtained using similar principles to
\citet{Vikhlinin2006}, with errors estimated by using 1000 artificial
profiles generated from Monte-Carlo simulations. \fgfive{} for
these 23 groups ranges from $5\%$ -- $11\%$, with a median of $8\%$,
lower than for the more massive clusters of \citet{Vikhlinin2006} and
\citet{Arnaud2007}.

\begin{figure*}[hbt]
\plotonebig{Giodini_Fgas}
\caption{The dependence of $\fg(<\rfive)$ on \Mfive{} ($\sim
0.73\Mvir$), as presented in \citet{Giodini2009}. The light-grey
  points represent individual group/cluster observations from
  \citet{Vikhlinin2006}, \citet{Arnaud2007}, and \citet{Sun2009},
  while the dark points are the average gas fractions, binned
  logarithmically with mass. Lower-mass halos show significantly lower
gas fractions, with $\fg(<\rfive)$ scaling roughly as
$\Mfive^{0.21}$. }
\label{fig:Giodini_Fgas}
\end{figure*}    



The above three samples were combined in the analysis of
\textbf{\citet{Giodini2009}} (G09), which used all 10 clusters from
\citet{Vikhlinin2006}, all 10 clusters from \citet{Arnaud2007}, and 21
of the 23 best-measured groups from \citet{Sun2009} to study the
group/cluster gas mass fraction over a wide range of halo masses. The
authors binned the 41 groups and clusters logarithmically by mass,
highlighting that lower mass groups have significantly lower gas
fractions. The best-fit trend is:
\begin{equation}
\fg(<\rfive) = (9.3 \pm 0.2)\power{-2}~h_{70}^{-3/2}~
\left(\frac{\Mfive}{2\power{14}~\Msun}\right)^{0.21 \pm 0.03}.
\end{equation}
Figure \ref{fig:Giodini_Fgas} shows the distribution of the observed
gas fractions, as a function of virial mass, measured by the three
works above. We will use the G09 bins as 5 independent samples of
\fg{} for different masses.

Recent results from the \Planck{} satellite detected the ICM using the
thermal SZ effect, which measures the integrated line-of-sight gas
pressure. \textbf{\citet{PlanckIntV}} (hereafter PC13) derived a
stacked pressure profile for 62 massive clusters that have archival
observations with \XMM. The cluster sample \citep[detailed
  in][]{PlanckEarlyXI} includes clusters of mass $\Mvir =
3.3\power{14}\dash2.7 \power{15}~\Msun$, with median mass
approximately $\Mvir = 8.70 \power{14}~\Msun$. X-ray temperatures
range from $kT = 3.4\dash13~\keV$. Total mass (\Mfive) was derived
from a scaling relation with the quantity $Y_X =
\Mass_{gas}T_X$, an observable quantity that has
been found to be a good mass proxy. The scaling relation in question
\citep{Arnaud2010} was calibrated against X-ray derived hydrostatic
masses, and so the total mass of the stacked \Planck{}
clusters assumes hydrostatic equilibrium. The total mass beyond
\rfive{} was determined assuming an NFW profile. The stacked pressure
profile is measured to unprecedented scales ($3\rfive \approx
1.6\rvir$), although the X-ray temperature profile measured by \XMM{}
only extends to \rfive{}, so the authors extrapolated the observed
temperature profile to $3\rfive{}$ to match the pressure observations.

Assuming the ICM acts as an ideal gas ($P=n_ekT$), the authors
inverted the pressure and temperature profiles to retrieve the gas
density profile and derive $\fg{}(<\radius)$ out to
3\rfive{}\footnote{The authors also derive the gas fraction assuming a
  conservative case in which the ICM is isothermal beyond \rfive{},
  resulting in lower \fg{}.}. The reconstruction of the temperature
profile was initially flawed, and the correct gas fraction profile was
given in an erratum, \citet{PlanckIntVb}. The gas fraction increases
from \rfive{} to \rvir{} \citep[as indeed shown by][see
  \ref{sec:Gas.Extrapolation}]{Rasheed2011}, reaching a peak of
$\approx 15\pm2\%$ at $1.6\rvir{}$.

\textbf{\citet{Eckert2013b}} (E13) combined the stacked pressure
profile from \citet{PlanckIntV} with a stacked X-ray
surface-brightness profile that directly constrains the gas density to
\rtwo{}. The X-ray observations were performed with \Rosat{}, on a
sample of 31 clusters ($z\lsim0.2$) of temperatures $kT =
2.5\dash9~\keV$, with median $kT = 6.5~\keV$. The cluster masses range
from $\Mvir \approx 1.4 \power{14}$ to $1.0 \power{15}~\Msun$, with
median $\Mvir = 6.0\power{14}~\Msun$\footnote{The authors do not give
  the masses of the clusters, so these values are taken from the
  $\Mass{}\dash{}T$ relation of \citet{Vikhlinin2006} (our equation
  \ref{eq:M-T}).}. The \Planck{} pressure profile, combined with the
gas density profile, was used to determine the total mass, assuming
hydrostatic equilibrium. This is different from the method used by
\citet{PlanckIntV}, which used a scaling relation and assumed an NFW
profile. However, both estimates rely on the assumption of hydrostatic
equilibrium either explicitly or implicitly through calibration of the
$Y_X\dash\Mfive$ relation. 18 clusters are in common
between the \Rosat{} and \Planck{} samples, and the authors separated
them into cool-core (CC, 6 clusters) and non-cool core (NCC, 12
clusters) categories. The gas fraction profile was determined
separately for the two categories, and the authors found that NCC
clusters have significantly higher gas fractions within \rtwo{}
($0.169 \pm 0.010$) than relaxed, CC clusters do ($0.134\pm 0.011$),
suggesting that the irregular, non-spherical morphologies of the
disturbed clusters may bias the gas fractions high. They also found
that \fg{} increases from \rfive{} to \rtwo{} ($\fgfive \approx
0.12$ for CC clusters).

\textbf{\citet{Umetsu2009}} (U09) observed the ICM of four very
massive ($\Mvir \gsim 1\power{15}~\Msun$, $kT \approx 9\dash10~\keV$)
clusters using Thermal SZ measurements with the \textit{AMiBA} CMB
telescope. After deriving pressure profiles from the SZ effect, the
authors obtained the gas density profile using archival X-ray
temperature measurements and a theoretically-derived temperature
profile \citep{Komatsu2001}. The authors use \textit{Subaru}
observations to derive the cluster masses with weak lensing
analysis. The average gas fraction is determined to the limit of the
SZ observations, \rtwo, and is observed to be $\fg(<\rfive) = 0.126
\pm 0.019$ within \rfive{} and $\fg(<\rtwo) = 0.133 \pm 0.020$ within
\rtwo{}. These observations also find that \fg{} increases with radius
beyond \rfive{}, in agreement with \citet{Rasheed2011},
\citet{PlanckIntV}, and \citet{Eckert2013b}. We emphasize that the
total mass for these clusters is \textit{not} dependent on the
assumption of hydrostatic equilibrium (because they were derived using
weak lensing measurements). See our discussion of the hydrostatic mass
bias in Section \ref{sec:Limitations.HSE}.

Table \ref{tab:F_gas_obs} lists the observed data from the samples
described above, including the most important characteristics of each
sample: median mass ($\Mvir$), whether that mass is derived
assuming HSE, and the gas fraction at directly-observed radii.

\begin{table}[hbt]
\caption{Samples of Groups/Clusters: Relevant Observations}
\scriptsize
\begin{tabular}{llccccc}
\hline \hline\\
\footnotesize \textbf{Reference} & \footnotesize \textbf{Clusters} & \footnotesize \textbf{$\left<\Mvir\right>$ (\Msun)} & \footnotesize \textbf{HSE?} & \footnotesize \textbf{$\fg{}_{,500}$} & \footnotesize \textbf{$\fg{}_{,200}$} & \footnotesize \textbf{$\fg{}_{,vir}$} \\
\footnotesize (1) & \footnotesize (2)& \footnotesize (3)& \footnotesize (4)& \footnotesize (5)& \footnotesize (6)& \footnotesize (7) \\\\
\hline
G09 Bin 1 & 2 & $2.9\power{13}$ & $\surd$ & $0.074 \pm 0.028$ & & \\
\phantom{G09} Bin 2 & 7  & $7.0\power{13}$ & $\surd$ & $0.068 \pm 0.005$ & & \\
\phantom{G09} Bin 3 & 17 & $1.7\power{14}$ & $\surd$ & $0.080 \pm 0.003$ & & \\
\phantom{G09} Bin 4 & 5 & $4.1\power{14}$ & $\surd$ & $0.103 \pm 0.008$ & & \\
\phantom{G09} Bin 5 & 10 & $9.8\power{14}$ & $\surd$ & $0.123 \pm 0.007$ & & \\
PC13 & 62 & $8.7\power{14}$ & $\surd$ & $0.125\pm0.005$ & $0.137\pm0.003$ & $0.145\pm0.01$\\
E13 - CC & 6 & $5.9\power{14}$ & $\surd$ & $0.115\pm0.010$ & $0.134\pm0.011$ & \\
U09 & 4 & $1.2\power{15}$ & WL & $0.126\pm0.019$ & $0.133\pm0.020$ & \\
\hline
\end{tabular}
\caption*{\small{(1) G09, PC13, E13, and U09 stand for
    \citet{Giodini2009}, \citet{PlanckIntV}, \citet{Eckert2013b}, and
    \citet{Umetsu2009}, respectively. CC represents the sub-sample
    of cool-core clusters.\\ (2) The number of
    clusters in each sample.  (3) The median virial mass of the
    clusters. \\ (4) $\surd$ marks that the total mass assumes
    hydrostatic equilibrium. WL marks the mass was determined
    with weak lensing, and does not depend on the assumption of HSE.\\ (5) $\fg(<\rfive)$ (6) $\fg(<\rtwo)$
    (7) $\fg(<\rvir)$}}
\label{tab:F_gas_obs}
\end{table}


\subsection{Extrapolation of Gas Density Profiles}
\label{sec:Gas.Extrapolation}
As seen above, very few observations retain the sensitivity necessary
to measure the gas density to \rvir{}. Therefore, to constrain the gas
fraction within the virial radius, we extrapolate the observed gas
mass profile (at \rtwo{} or \rfive{}) to larger radius by assuming a
power-law profile:
\begin{equation}
\rho_{gas}(\textrm{r}) \propto \textrm{r}^{-\alpha_{gas}}\;, 
\end{equation}
where $\alpha{}_{gas}$ is the slope of the gas density profile, which can in
general change as radius increases. The total
matter density can be similarly modeled,
\begin{equation}
\rho_{m}(\radius) \propto (\radius)^{-\alpha_{m}}\;,
\end{equation}
with $\alpha{}_m$ the slope of the total mass density profile. At
large radii, the full equation for the gas fraction simplifies to approximately:
\begin{align}
\fg(<\radius) = \frac{\Mass_{gas}}{\Mass_{tot}} &=
\frac{\int_0^\radius 4\pi{}\radius'^2\rho_{gas}(\radius')d\radius'}{\int_0^\radius
  4\pi{}\radius'^2\rho_{m}(\radius')d\radius'} \\ & \approx
\frac{\rho_{gas}(\radius)}{\rho_{tot}(\radius)} \nonumber\\ &\propto \radius^{\alpha_{m} - \alpha_{tot}}\;.
\end{align}
Therefore, the gas fraction can be extrapolated to larger radii using
the difference in slopes between the gas density and total mass
density profiles.

\textbf{\citet{Rasheed2011}} (R11) used this approach to extrapolate
the gas fraction of the G09 cluster samples to \rvir{}. X-ray and SZ
observations show that the gas density decreases more slowly with
radius than the total mass density ($\alpha_{gas} < \alpha_{m}$),
suggesting that the gas fraction increases when the cluster outskirts
are considered. This is as expected if the gas is shock heated or
supported by feedback pressure, and thus extends to larger radius.

R11 surveyed the literature to recover X-ray measurements which
constrain the gas density slope nearly to \rvir{}. These measurements
include observations with \Rosat{}, \Chandra, \XMM, and
\textit{Suzaku}, and cover a wide range of cluster masses and
temperatures. Averaging over the many observations, the authors find
that the observed gas density slope at \rfive{} is steeper in with
more massive clusters, with $\alpha_{gas}$ ranging from $\approx 1.8
\pm 0.2$ for poor clusters (G09 bin 2, $\Mvir \approx
7\power{13}~\Msun$) to $\approx 2.3 \pm 0.02$ for the most massive G09
bin ($\Mvir \approx9.8\power{14}~\Msun$).

Compared to the gas density profile, the total density (NFW) profile
is significantly steeper in the outer regions of the halo. The NFW
profile has a slope of $\alpha_m = 2.6$ between \rfive{} and \rtwo{},
and steepens to $\alpha_m = 2.7$ in the region \rtwo{} to
\rvir{}. Therefore, R11 find that the gas fraction rises significantly
beyond \rfive{}. Because $\alpha_{gas}$ increases with cluster mass,
the gas fraction rises more quickly with radius in groups and poor
clusters ($\fg\propto \textrm{r}^{0.8}$ for G09 bin 2) than in rich
clusters ($\fg \propto \textrm{r}^{0.3}$ for G09 bin 5). For these two
bins, this yields an increase in \fg{} by a factor of $1.6$ and $1.2$,
respectively, from \rfive{} to \rvir{}. This offers an explanation for
the particular deficiency of baryons in low-mass clusters: the
shallower gas profile implies the ICM is spread more broadly than in
massive clusters, and the gas is extended further into the cluster
outskirts.

We adopt R11's extrapolation model in order to approximate the gas
fraction at \rvir{} in the samples which do not measure \fg{} to
these scales (all except PC13). $\alpha_{gas}$ for each sample in Table
\ref{tab:F_gas_obs} is taken from the temperature-slope relation in
R11; for consistency, we do not extrapolate the gas profiles for any
individual cluster sample with coverage beyond \rfive{}. We assume
$\alpha_m$ as above for the NFW profile ($\alpha_m = 2.6$ for
$\radius<\rtwo$, $2.7$ for $\radius>\rtwo$). We extrapolate \fg{} from the
maximum observed radius, r$_a$, to a larger radius r$_b$ using:
\begin{equation}
\fg{}_{,b} =
\fg{}_{,a}\left(\frac{\textrm{r}_b}{\textrm{r}_a}\right)^{\alpha_m - \alpha_{gas}}.
\end{equation}
 For example, extrapolating the gas fraction of G09's bin 5 from \rfive{} to \rtwo{}:
\begin{align}
\fgtwo &= \fgfive\left(\frac{\rtwo}{\rfive}\right)^{\alpha_m -
  \alpha_{gas}} \\ &\approx 0.103~(1.45)^{2.6 - 2.3} \nonumber
\\ &\approx 0.115 \nonumber
\end{align}

To calculate the uncertainty on the extrapolated gas fraction, we
propagate the errors in $\fg{}_{,a}$ and in $\alpha_{gas}$, making the
simplification that no significant uncertainty exists in $\alpha_m$ or
r$_b$/r$_a$. The fractional errors add in quadrature.
\begin{align}
\frac{\Delta\fg{}_{,b}}{\fg{}_{,b}} &=
\sqrt{\left(\frac{\Delta\fg{}_{,a}}{\fg{}_{,a}}\right)^2 +
  \left(\frac{\Delta\left(\textrm{r}_b/\textrm{r}_a\right)^{\alpha_m-\alpha_{gas}}}{\left(\textrm{r}_b/\textrm{r}_a\right)^{\alpha_m-\alpha_{gas}}}\right)^2}\;.\\
\intertext{The uncertainty in the right term is}
\Delta\left(\textrm{r}_b/\textrm{r}_a\right)^{\alpha_m-\alpha_{gas}}
& =
\left(\textrm{r}_b/\textrm{r}_a\right)^{\alpha_m-\alpha_{gas}}\ln{\left(\textrm{r}_b/\textrm{r}_a\right)}\Delta\alpha_{gas}\,,\\
\intertext{yielding the final result:}
\frac{\Delta\fg{}_{,b}}{\fg{}_{,b}} &=
\sqrt{\left(\frac{\Delta\fg{}_{,a}}{\fg{}_{,a}}\right)^2 +
  \left(\ln{\left(\textrm{r}_b/\textrm{r}_a\right)}\Delta\alpha_{gas}\right)^2}\;.
\end{align}

The gas density profile is expected to steepen at very large radii,
such that it eventually matches the NFW profile \citeeg{Umetsu2009},
which translates to the gas fraction asymptotically approaching a
constant value. At large enough radius, extrapolation of the gas
fraction as described above will, therefore, become invalid, as
$\alpha_{gas}$ will not remain fixed. The range at which the gas
density steepens significantly is not known, however, as observational
data does not currently constrain $\alpha_{gas}$ far beyond
\rtwo{}. We assume $\alpha_{gas}$ remains constant to
\rvir{} (and slightly beyond), and emphasize that the our estimate
of the gas fraction will be biased high if the gas density slope
steepens significantly beyond \rtwo{}. In Section
\ref{sec:Limitations.Slope}, we discuss current constraints on this
steepening of this slope.

\section{Cluster Stellar Mass Fraction}
\label{sec:Stellar}

The integrated stellar mass of groups and clusters is also an
important (although subdominant) reservoir of baryons in these large
systems. The stellar mass of clusters is thought to come almost entirely
from the stellar content of the individual cluster galaxies (plus an
estimated $\approx10\%$ from the diffuse intracluster light).

G09 estimated the stellar content of a large number ($>90$) of
groups and clusters from the \textit{COSMOS} survey. Using optical and
infrared observations from \textit{Subaru}, the authors fit a
broad-band spectrum to the detected galaxies in each cluster and used
their spectral energy distributions to derive photometric redshifts
for the sample. Converting the $Ks$-band (IR) luminosity of detected
galaxies to stellar mass, and accounting for the entire predicted
galactic mass function, G09 determined the stellar fraction in
clusters of masses $1\power{13}\dash1\power{15}~\Msun$. At \rfive{},
the stellar fraction was found to be significantly higher ($\approx
6\%$) for groups and poor clusters than for the most massive clusters
($\approx 2\%$). The authors derived a fit to stellar-fraction versus
mass of:
\begin{equation}
\textrm{f}_{*}(<\rfive{}) =
(5.0\pm0.1)\power{-2}~\left(\frac{\Mfive}{5\power{13}\Msun}\right)^{-0.37\pm0.04}
\;.
\end{equation}

The results of an extensive Sloan Digital Sky Survey (SDSS) weak
lensing study of stacked groups and clusters has placed unparalleled
constraints on the stellar fraction in groups and clusters. The MaxBCG
sample \citep{Sheldon2009a} contains $> 130,000$ groups and clusters
between redshifts $z = 0.1\dash0.3$. \citet{Sheldon2009b} binned the
clusters by richness and luminosity, and stacked by centering on the
brightest cluster galaxy (BCG). The mean weak lensing profile was
observed well into the surrounding large scale structure ($\gsim
15\dash20 \rvir$), as was the averaged optical surface brightness,
allowing the total mass to light ratio (M/L) to be
determined. \textbf{\citet{Bahcall2014}} (BK14) determined the
observed mass-to-light distribution as a function of environment and
scale. They combined this with the observed stellar mass to light
ratio of individual spiral and elliptical galaxies and their relative
abundance from the density-morphology relation, and used it to
determine the stellar mass fraction as a function of radius and for
groups and clusters of various masses.

Figure \ref{fig:BK14} shows the cumulative stellar fraction
distributions derived by BK14. The stellar fraction is determined at
$\radius{}_{200b}$\footnote{$\radius{}_{200b}$ represents where the
  density is 200 times the cosmic matter density, not the critical
  density, and corresponds roughly to our definition of \rvir{}.} as a
function of cluster mass. The authors find that the stellar fraction
at this fixed radius decreases with virial mass, in good agreement with
G09. The stellar fraction is also obtained as a function of radius,
divided into three richness bins, with mass ranges corresponding
approximately to $\Mvir{}< 2\power{13}~\Msun$, $2\power{13}~\Msun{} <
\Mvir{} < 1\power{14}~\Msun{}$, and $\Mvir{}>1\power{14}~\Msun{}$. The
cumulative stellar fraction decreases significantly from the centers
of groups and clusters, asymptotically approaching at high radius a
``cosmic stellar fraction'' of $\textrm{f}_{*} \approx 0.01 \pm 0.004$
in all richness bins. The local stellar fraction is approximately the
cosmic fraction at all scales above roughly $300$ kpc, the scale of
the BCG. The higher cumulative stellar fraction at any radius in
groups and poor clusters suggests that their stellar masses are more
dominated by the BCG than massive clusters.

\begin{figure*}[hbtp]
\plottwobig{Copied_Figs/BK14_FsvM}{Copied_Figs/BK14_FsvR}
\plotoneman{Copied_Figs/BK14_FsvR_local}{.49}
\caption{The stellar fraction from stacked optical and weak lensing
  observations, as presented in \citet{Bahcall2014}. \textit{Upper
    left}: The cumulative stellar fraction within $\radius{}_{200b}
  \approx \rvir{}$ in groups and clusters, as a function of total
  mass. f$_*$ decreases in more massive clusters, in agreement with
  observations collected by \citet{Giodini2009}. \textit{Upper right}:
  The cumulative stellar fraction as a function of halo-centric
  radius. The profiles are binned by richness, a proxy for mass. More
  massive clusters have lower stellar fraction at any given radius
  (lower mass halos are more dominated by their BCGs), but the stellar
  fraction tends towards a constant value at high radius irrespective
  of mass: the ``cosmic stellar fraction'' $\approx
  1\%$. \textit{Lower}: The local stellar fraction of all groups and
  clusters, regardless of richness, is approximately $1\%$ at all
  scales above the BCG.}
\label{fig:BK14}
\end{figure*}                
                                  
%\afterpage{\clearpage}


\section{Galaxy Mass Fractions}
\label{sec:Galaxy}

Studying the distribution of baryons in individual galaxies presents
many other challenges not faced in the investigation of
clusters. Because galaxy systems are significantly lower mass than
clusters ($\Mvir \approx 1\dash2\power{12}~\Msun{}$ for a typical
L$^*$ galaxy), their gravitational wells are shallower, and it is
easier for feedback effects to remove gas from the systems altogether
\citeeg{Oppenheimer2010}. The lower total mass also translates to a
much lower thermal temperature for the gas to be in hydrostatic
equilibrium. Therefore, the main diffuse gaseous component of galactic system
will primarily be too cool and low density to be detected in
emission. 

The most obvious and easily detected baryon reservoir in (L$^*$)
galaxies is the galactic disk itself, which contains the stellar
population as well as the gas and dust of the ISM. Baryons are also
stored in the ``circumgalactic medium'' (CGM), a large, diffuse region
surrounding galaxies with low temperatures and densities. Observations
are beginning, for the first time, to place reasonable constraints on
the total mass in this baryonic phase, allowing estimates of the
makeup, distribution, and total fraction of baryons in the low-mass
halos of galaxies.

\subsection{The Circumgalactic Medium}
\label{sec:Galaxy.CGM}

The existence of the CGM was first predicted by \citet{Bahcall1969} to
explain the presence of absorption features in the spectra of
quasars. The emission from quasars was known to come from high
redshift, while the absorption lines came from lower redshifts
(typically $z = 1\dash3$), had low dispersion velocities, and showed
the presence of many abundant elements and ionization states
(including absorption from \ion{H}{i}, \ion{C}{ii}, \ion{C}{iv},
\ion{Si}{ii}, and many others). The authors predicted that an extended
halo of gas ($\radius{} \approx 10^2$ kpc) surrounding normal galaxies
($\Mass{} \approx 10^{11}\dash10^{12}~\Msun$) could produce the observed
absorption profiles and dispersion velocities. Even at this early
stage, many separate phases of the CGM were predicted, as temperatures
ranging from $2\power{4} - 2\power{5}~\textrm{K}$ were required to
explain the variety of ionization states.

Very recently, observations have confirmed this hypothesis by
associating the absorption features in QSO sightlines to nearby
foreground galaxies with the expected redshifts. The CGM is a gaseous
reservoir in the dark matter halo around galaxies, which extends for
several hundred kiloparsecs, and likely as far as the virial radius
($\rvir{}\approx 300$ kpc for an L$^*$ galaxy). The CGM is also likely
the major source of pristine gas accreting onto the galactic disk.

The use of QSO and galaxy absorption spectroscopy remains the most
powerful tool in statistically characterizing the CGM. The high
spatial density of background galaxies and quasars results in a large
number of sightlines through the CGM at a wide variety of projected
distances. While it is very unlikely to have enough bright background
sources to resolve the distribution of the CGM in any individual
galaxy, it is possible to recover the average profile of gas
absorption in many galaxies as a function of radius by stacking many
sightlines together by the projected distance from their absorbing
galaxy.

\citet{Steidel2010} used a sample of 512 galaxy pairs, where both
foreground and background galaxies have spectroscopic redshifts and
are separated by less than 15" on the sky. At the median redshift of
the foreground absorbers ($z\approx 2$), this translates to projected
distances within 125 kpc. Absorption profiles (stacked into three
projected radius bins) were measured for Ly$\alpha$, \ion{C}{iv}
(1549\AA), \ion{C}{ii} (1334\AA), \ion{Si}{iv} (1393\AA), and
\ion{Si}{ii}(1260\AA{} and 1526\AA). Hydrogen, the most dominant
element by mass in the CGM, is measured in its neutral phase through
Ly$\alpha$ absorption, and is observed well beyond 100 kpc. Using an
approximate model for populating the CGM by galactic outflows, the
authors place a rough constraint on the CGM mass of $\Mass{}_{CGM}
\approx 3\power{10}~\Msun{}$, although we emphasize that this is only
an order-of-magnitude estimate. By this approximation, the CGM
contains roughly the same mass as the sum of stars and gas in the
stellar disk, and amounts to $\approx 3-4\%$ of the total mass of the
dark matter halo ($\approx 20\%$ of the expected baryonic mass,
assuming the halo contains the entire cosmic baryon fraction).

The COS-Halos survey \citep{Tumlinson2011} was designed specifically
to study the CGM in low-redshift galaxies. It uses far-UV spectroscopy
of background quasars behind 44 roughly L$^*$ galaxies, observed
with the Cosmic Origins Spectrograph instrument on the Hubble Space
Telescope. Measuring the column density in Ly$\alpha$, early results
\citep{Thom2012,Werk2013} placed a very conservative lower limit on
the CGM gas mass of $\Mass{}_{CGM} > 10^9 \Msun{}$. Later analysis by
\textbf{\citet{Werk2014}} (W14) developed a model for the ionization
state of the CGM, constrained by the observed absorption of low,
intermediate, and high-ionization transitions in metals. This model --
which indicates that the hydrogen in the CGM is increasingly ionized
at larger radii -- is used to greatly improve the limits on the CGM
mass. Integrating their best-fit gas density profile to the virial
radius (300 kpc), they find that the cool phase (T $\approx 10^4 -
10^5$ K) of the CGM in L$^*$ galaxies has a mass of $\Mass{}_{CGM}
\approx 7\power{10}\dash1.2\power{11}~\Msun{}$. This CGM mass is
larger than, although consistent with, the rough estimate made by
\citet{Steidel2010}, and is similar to the combined mass of the
stellar disk and ISM. The uncertainty in $\Mass{}_{CGM}$ is dominated
by saturation in a few Ly$\alpha$ absorption profiles. The minimum CGM
mass is $\Mass{}_{CGM} > 6.5\power{10} ~\Msun$, and could range as
high as $\Mass{}_{CGM} \approx 1.2\power{11}~\Msun{}$ if all saturated
column densities are 3 times higher than their lower limits. The total
virial mass of L$^*$ galaxies is approximately $1.6\power{12}~\Msun{}$,
so the authors conclude the CGM accounts for $25\dash45\%$ of the
expected baryonic mass in galaxies ($\Mass{}_{b} = \fb{}_{,cosmic}
\Mvir{} \approx 2.6\power{11} \Msun{}$).

\subsection{Estimate of The Galactic Baryon Fraction}
\label{sec:Galaxy.Fraction}

W14 combined their estimate of the cool gas mass in the CGM with
observations of other baryonic components of galactic systems to
constrain the total baryon fraction currently detected within the
virial radius. The median stellar mass of the COS-Halos galaxies is
$\left<\Mass{}_*\right> = 4\power{10}\Msun$, in agreement with
abundance matching estimates \citep{Behroozi2010}. The gaseous
component of the ISM, observed with HI surveys, can vary from very
little (for elliptical galaxies) to of order the stellar mass
\citep{McGaugh2010, Martin2010}. Together, for a star-forming L$^*$
galaxy, the mass of the stellar and gaseous disk (the normal galactic
component) is given as $7\power{10}\Msun{}$. Thus, the disk
contributes about $4-5\%$ of the total mass of the galactic system, or
about $27\%$ of the expected baryon content.

The presence of high-ionization absorption lines in CGM profiles
suggests that a warm ($>10^5~\textrm{K}$) CGM component exists. The
ionization model derived by W14 to fit the low-ionization transition
abundances severely underestimates the column density of
\ion{O}{vi}. The authors reference models which predict a mass of
warm CGM of at least $\Mass_{CGM, warm} \gsim 10^{10}~\Msun{}$ to
explain the observed \ion{O}{vi} absorption \citep{Peeples2014}. This
mass estimate is highly sensitive to the assumed metallicity in the
warm CGM phase, and decreasing Z from Z$_\odot$ to $0.1$Z$_\odot$
increases the predicted warm CGM mass by a factor of 10. The
constraints on the warm CGM fraction thus range from $0.6\%$ to $6\%$
of the total virial mass, or $4\dash40\%$ of the expected baryon fraction.

X-ray observations hint at the existence of a hot ($>10^6~\textrm{K}$)
gaseous reservoir outside large galaxies. Estimates of the total mass
of this X-ray component range from $10^9\dash10^{11}~\Msun{}$. W14
argues that, by extrapolating the observed mass of $\approx
10^9\Msun{}$, observed within 50 kpc of an L$^*$ galaxy by \Rosat{}
\citep{Anderson2013}, to 300 kpc could increase the mass by a factor
of $6-14$, although the density profile of this X-ray component is
unconstrained. W14 adopts the range of $10^9 \dash
1.4\power{10}~\Msun{}$ for the X-ray CGM from this work. This is
between $\approx 1\%$ to $5\%$ of the expected baryonic mass, or
$<1\%$ of the total mass budget. \citet{Gupta2012}, however, claims
that the Milky Way may have a hot, X-ray reservoir with mass as high
as $10^{11}~\Msun{}$, based on \ion{O}{vii} and \ion{O}{viii}
absorption observed with \XMM{} and emission from the soft X-ray
background. If this is true, the X-ray component of the CGM may be
substantially larger than observed by \citet{Anderson2013}.

Allow us to summarize the distribution of the expected baryonic mass
($\Mass{}_b\approx 2.6\power{11}~\Msun{}$) in galactic halos, as
described above and by \citet{Werk2014}. The gas in the cool CGM has
at least $25\%$ and as much as $50\%$ of $\Mass{}_{b}$. The stellar
disk contains $\approx25\%$, depending on the amount of gas in the
ISM. The warm CGM contains between $4\%$ and $40\%$, and the X-ray CGM
is thought to hold $1\% - 5\%$. W14 concludes that current
observations could be detecting the entire baryon fraction in galactic
clusters, with the primary uncertainty resting on the contents of the
warm CGM.

To fit a best estimate to the baryon fraction, we take the mean of the
lower and upper limits on each component's abundance, with the
exception of the stellar-disk component, which we take to be the upper
limit of $4.5\%$ of the total virial mass from stars and ISM. We take
one-half the range on the limits as the uncertainty:
\begin{align}
\textrm{f}_{disk}(<\rvir{})&= 0.045 \pm 0.013 \nonumber \\
\textrm{f}_{CGM, cool}(<\rvir) &= 0.060 \pm 0.013 \nonumber \\
\textrm{f}_{CGM, warm}(<\rvir) &= 0.035 \pm 0.025 \nonumber \\
\textrm{f}_{X-ray}(<\rvir) &= 0.005 \pm 0.004 \nonumber 
\intertext{Summing these components and propagating uncertainties,
  assuming they are independent, yields our best estimate of the
  baryon fraction in galactic systems:}
\textrm{\textbf{f}}_{\textbf{b}}(<\textbf{\rvir}) &= \textbf{0.145} \pm \textbf{0.030} \;.
\end{align}
If the errors are not independent (and add linearly, not in
quadrature), the true lower and upper limits on the baryon fraction in
galactic systems are $9.5\%$ and $19\%$, respectively.
