%Senior Thesis Chapter 5
%Ben Cook '14 (bacook@)
%Adviser: Neta Bahcall
\chapter{Summary and Conclusions}
\label{chap:Conclusions}

In this thesis, we present a synthesis of observational constraints on
the distribution and abundance of baryons in dark matter halos over a
wide range of masses. Baryons, while only $16\%$ of the total mass in
the universe, are the most easily detectable form of matter and are
incredibly important tools in understanding and observing the
formation of structure in the universe. Previous
observations suggest halos are deficient in baryons relative to the
cosmic fraction, a discrepancy known as the ``Missing Baryon
Problem''. We trace the baryon distribution in halos from galactic
scales ($\Mvir{} = 10^{12}~\Msun$) to groups ($\Mvir{}=10^{13}~\Msun$),
and clusters ($\Mvir{} = 10^{14}\dash10^{15}~\Msun$) to address the missing
baryon problem and to obtain a global picture of the distribution of
baryons relative to dark matter in the universe. 

Our group and cluster gas mass data is comprised of a collection of
X-ray and SZ measurements of the gas density, temperature, and
pressure profiles in the ICM. The total mass is derived either through
the assumption of hydrostatic equilibrium or by using weak
lensing. When the gas fraction is not measured to the virial radius,
we extrapolate the observed gas fraction using gas density profile
slopes appropriate to the given halo mass or ICM temperature
\citep{Rasheed2011}. Our galaxy cluster data comes from a compilation
by \citet{Werk2014}, and includes absorption measurements of the
multiphase CGM, stellar masses, and ISM masses from HI surveys.

Our main results are as follows:
\begin{enumerate}
\item Although the gas fraction within \rfive{} in clusters is
  significantly lower than the cosmic baryon fraction, the gas
  distribution is more extended than the dark matter distribution due
  to shock heating and other baryonic processes, such that the gas
  fraction increases significantly when integrated to the cluster
  outskirts.
\item The gaseous ICM is more extended in low-mass halos, explaining
  why the gas fraction is observed to be particularly low within
  \rfive{} in these halos. The gas fraction rises more steeply with
  radius in groups and poor clusters than in massive clusters, and our
  extrapolations predict that the gas fraction of all groups and
  clusters is consistent with cosmic fraction near \rvir{}.
\item The cluster stellar fraction at any given radius is higher in
  groups and poor clusters, although in clusters of all sizes the
  fraction asymptotically approaches the ``cosmic'' stellar fraction
  $0.01\pm0.004$ at high enough radius. 
\item Observational constraints on the gas content of galactic halos
  find the cool CGM can account for $25-50\%$ of the total cosmic
  baryonic mass fraction. Adding the baryons located in the stellar
  disk, ISM, and multiphase CGM result in a galactic baryon fraction
  for $\sim$L$^*$ galaxies that is nearly consistent with the cosmic
  fraction within the viral radius of $\sim300$ kpc around L$^*$
  galaxies.
\item Combining the observed baryonic components of galaxy, group, and
  cluster halos, we show that dark matter halos may contain the cosmic
  fraction of baryons within approximately the virial radius, across
  three orders of magnitude in halo mass.
\item The baryonic distribution in halos is not entirely
  self-similar. Baryons in less-massive halos are pushed farther into
  the outskirts of the shallow gravitational well by feedback
  mechanisms, not reaching the cosmic fraction until higher radii. The
  abundance of different baryonic components (particularly stellar
  mass and gas) also changes somewhat with cluster mass.
\item The consistent baryon fraction in galactic and cluster halos
  suggests that cluster dark matter and baryonic masses could be
  composed entirely of matter originally in galactic halos which fell
  into cluster halos. The dark matter of galactic potentials could
  have been stripped to form the group and cluster halo, and the
  galactic CGM gas, stripped and heated when falling into the
  potential, could be sufficient to explain the gas found in the ICM
  of groups and clusters. The stellar fraction of individual L$^*$
  galaxies agrees with the stellar fraction of groups, and while
  clusters have a lower stellar fraction, this could be due to a
  suppression of star formation in cluster galaxies rather than
  requiring a large additional source of gas and dark matter to fall
  into clusters.
\item The baryon distribution of the universe traces the dark matter
  distribution reasonably well, with no need for additional unseen or
  unbound reservoirs of baryons from halos. Averaged over scales
  larger than the virial radius, baryons map the total structure of
  the universe, and the baryonic mass of clusters is an effective
  proxy for the total mass.
\item The combination of these observations suggests that baryons,
  which originally fell with the cosmic fraction into halos, remain in
  the systems with approximately the same cosmic fraction. Despite
  baryon processes such as heating, infall, and feedback, which play
  an important role in expanding the hot gas distribution relative to
  the dark matter, the baryons remain largely within the dark matter
  halos of galaxies, groups, and clusters.
\end{enumerate}

Improvements in the precision of our results will come from better
constraints of the ICM temperature profile at large radii (in groups
and clusters) and of the mass in the warm CGM phase (in
galaxies). Additionally, further weak lensing calibrations are
required to constrain the magnitude of the hydrostatic equilibrium
bias on the total cluster mass.

The baryonic content of the universe, while energetically small, is
what makes up all physical objects we observe; it drives astrophysical
phenomena such as nucleosynthesis, chemistry, and planetary formation,
and is what eventually gave rise to life. Yet, until recently, baryons
were often thought to have been expelled from the gravitational
potentials of galaxies, groups, and clusters. Our results show that
baryons which fall into these systems remain within their potentials,
and can serve as useful tools to study the evolution of halos, the
cosmology of the low-redshift universe, and the growth of large-scale
structure.
